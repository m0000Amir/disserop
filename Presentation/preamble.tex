\begin{frame}[noframenumbering,plain]
    \setcounter{framenumber}{1}
    \maketitle
\end{frame}

\begin{frame}
    \frametitle{Положения, выносимые на защиту}
    \begin{enumerate} % https://tex.stackexchange.com/a/476052/104425
        \item математические модели в виде экстремальной комбинаторной задачи и
        задачи ЦЛП для оптимального размещения базовых станций при
        проектировании БШС с линейной топологией;
        \item специальный алгоритм МВ и Г для решения сформулированной
        экстремальной комбинаторной задачи;
        \item итерационная процедура нахождения последовательности лучших
        решений для задачи размещения базовых станций в рамках комплексного
        проектирования БШС с линейной топологией;
        \item математические модели для задач проектирования БШС с ячеистой
        топологией;
        \item имитационной модели многофазной сети массового обслуживания с
        зависимым временем обслуживания;
        \item модели прогнозирования оценок характеристик производительности сети с
        помощью методов машинного обучения.
      \end{enumerate}
\end{frame}
\note{
    Проговариваются вслух положения, выносимые на защиту
}

\begin{frame}
    \frametitle{Содержание}
    \tableofcontents
\end{frame}
\note{
    Работа состоит из четырёх глав.

    \medskip
    В первой главе \dots

    Во второй главе \dots

    Третья глава посвящена \dots

    В четвёртой главе \dots
}
