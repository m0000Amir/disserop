\begin{frame}[noframenumbering,plain]
    \setcounter{framenumber}{1}
    \maketitle
\end{frame}

\begin{frame}
    \frametitle{Положения, выносимые на защиту}
    
\begin{enumerate}
    \item Формулировка задачи оптимального размещения базовых станций при проектировании беспроводной широкополосной сети (БШС) вдоль протяденной транспортной магистрали в виде целочисленного линейного программирования (ЦЛП) и в виде комбинаторной модели в экстремальной форме.
   
    \item Специальный алгоритм типа ветвей и границ для решения сформулированной экстремальной комбинаторной задачи.
    \item Итерационная процедура нахождения последовательности лучших решений для задачи размещения базовых станций в рамках комплексного проектирования БШС вдоль протяженных транспортных магистралей.
    \item Математические модели для задач проектирования БШС с ячеистой топологией для обслуживания множества рассредоточенных объектов.

\end{enumerate}
\end{frame}
\note{
    Проговариваются вслух положения, выносимые на защиту
}

% \begin{frame}
%     \frametitle{Содержание}
%     \tableofcontents
% \end{frame}
% \note{
%     Работа состоит из четырёх глав.

%     \medskip
%     В первой главе \dots

%     Во второй главе \dots

%     Третья глава посвящена \dots

%     В четвёртой главе \dots
% }
