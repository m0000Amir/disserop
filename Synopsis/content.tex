\pdfbookmark{Общая характеристика работы}{characteristic}             % Закладка pdf
\section*{Общая характеристика работы}

\newcommand{\actuality}{\pdfbookmark[1]{Актуальность}{actuality}\underline{\textbf{\actualityTXT}}}
\newcommand{\progress}{\pdfbookmark[1]{Разработанность темы}{progress}\underline{\textbf{\progressTXT}}}

\newcommand{\objectresearch}{\pdfbookmark[1]{Объект исследования}{progress}\underline{\textbf{\objectresearchTXT}}}

\newcommand{\subjectresearch}{\pdfbookmark[1]{Предмет исследования}{progress}\underline{\textbf{\subjectresearchTXT}}}

\newcommand{\aim}{\pdfbookmark[1]{Цели}{aim}\underline{{\textbf\aimTXT}}}
\newcommand{\tasks}{\pdfbookmark[1]{Задачи}{tasks}\underline{\textbf{\tasksTXT}}}
\newcommand{\aimtasks}{\pdfbookmark[1]{Цели и задачи}{aimtasks}\aimtasksTXT}
\newcommand{\novelty}{\pdfbookmark[1]{Научная новизна}{novelty}\underline{\textbf{\noveltyTXT}}}
\newcommand{\influence}{\pdfbookmark[1]{Практическая значимость}{influence}\underline{\textbf{\influenceTXT}}}
\newcommand{\methods}{\pdfbookmark[1]{Методы исследования}{methods}\underline{\textbf{\methodsTXT}}}
\newcommand{\defpositions}{\pdfbookmark[1]{Положения, выносимые на защиту}{defpositions}\underline{\textbf{\defpositionsTXT}}}
\newcommand{\reliability}{\pdfbookmark[1]{Достоверность}{reliability}\underline{\textbf{\reliabilityTXT}}}
\newcommand{\probation}{\pdfbookmark[1]{Апробация}{probation}\underline{\textbf{\probationTXT}}}
\newcommand{\contribution}{\pdfbookmark[1]{Личный вклад}{contribution}\underline{\textbf{\contributionTXT}}}
\newcommand{\publications}{\pdfbookmark[1]{Публикации}{publications}\underline{\textbf{\publicationsTXT}}}


{\actuality} 
В настоящее время тенденция бурного развития информационных технологий во всех сферах деятельности человека оказывает весомое влияние на нефтегазовый сектор страны. Современные компании, представляющие собой сложные многоуровневые производственные системы, для своего устойчивого развития требуют постоянного развития информационных технологий.  Сегодня наблюдается  бурное развитие процесса «цифровизации» нефтегазовой отрасли. Крупные международные нефтегазовые компании имеют подразделения, задачами которых является разработка и реализация в дальнейшем принципов интеллектуального месторождения: «Умные месторождения» («Smart Fields») в компании Shell, «Месторождение будущего» («Field of the Future») в компании BP и «iFields» в компании Chevron и др \cite{Tcharo2018}. Развитие нефтегазового комплекса предусматривает переход к малолюдным системам управления добычей, транспортировкой и переработкой сырья. Основными современными информационными технологиями являются: большие данные (англ. Big Data), искусственные нейронные сети (англ. Artificial Neural Network – ANN), системы распределенного реестра (англ. Blockchain), промышленные интернет вещей (англ. Industrial internet of things – IIoT), технологии виртуальной и дополненной реальности (англ. Virtual Reality – VR), мониторинг распределенных объектов беспилотными летательными аппаратами БПЛА ( англ. Unmanned Aerial Vehicle – UAV). Необходимость в эффективной передаче больших объемов информации привела к развитию беспроводных технологий. Информационные  системы современных месторождений сегодня помимо данных первичного сбора и обработки информации технологических параметров основных производственных объектов содержат также колоссальный объем информации мультимедийного трафика. Сюда входят, например, данные по обнаружению утечек и разрушения трубопроводов; информация с камер видеонаблюдения, аналитики и т.д.

В настоящее время беспроводные технологии являются неотъемлемой частью процесса «цифровизации» месторождения. Активное использование беспроводных сетей основывается на ряде их преимуществ по сравнению с кабельными сетями:
\begin{itemize}
    \item возможность получения информации с любой точки контролируемой территории;
    \item быстрый ввод в эксплуатацию по системе подключение типа Plug-\&-Play;
    \item сокращение капитальных затрат на создание сети; 
    \item уменьшение затрат на эксплуатацию;
    \item высокая гибкость, мобильность, масштабируемость;
    \item упрощенные требования к обслуживанию оборудования.
\end{itemize}

В рамках этого процесса возникает актуальная научно - техническая проблема повышения качества проектирования беспроводной сети связи, осуществляющей сбор и передачу информации в центр  управления с множества контролируемых объектов на некоторой территории.   

% В совокупности со всеми вышеизложенными перспективными направлениями беспроводные технологии являются неотъемлемой частью «цифровизации» месторождения. Отсюда возникает научно - техническая проблема организации распределенной беспроводной сети связи, соответствующая реальным требованием современного производства.

% Большой объем передачи информации  привел к еще одной из наиболее интересных тенденций цифрового развития – внедрения беспроводных технологий. Современные месторождения сегодня, помимо данных первичного сбора и обработки информации технологических параметров основных производственных объектов содержат также колоссальный объем  информации мультимедийного трафика. Сюда входят данные БПЛА по обнаружению утечек и  разрушения трубопроводов; камер видеонаблюдений; а также большой поток данных цифровых двойников, аналитики и т.д. В совокупности со всеми вышеизложенными перспективными направлениями беспроводные технологии являются неотъемлемой частью «цифровизации» месторождения.



Процесс проектирования современной  беспроводных сетей связи (БШС) состоит из последовательного решения взаимосвязанных задач:

\begin{itemize}
    \item выбор типов технических средств и протоколов;
    \item выбор топологической структуры сети;
    \item анализ и оптимизация пропускной способности каналов связи, маршрутизация информационных потоков и др.
\end{itemize}

Задача синтеза топологии при комплексном проектировании БШС является основной проблемой исследования в данной работе.


{\progress} Создание и постоянное современной инфраструктуры передачи данных является одной из осноновных задач современного производства. Бурное развитие беспроводных сетей во всех областях деятельности человека обосновывает целесообразность их использования на нефтегазовых месторождениях. В настоящее время в России и за рубежом исследованию беспроводных сетей связи посвящен ряд работ, рассматривающие сети для контроля гражданских  и промашыленных объектов. Примерами таких объектов является жилые районы города, протяженные автомагистрали, железные дороги, трубопроводы и др. В частности, при исследовании проблемы синтеза топологии сети автор опирался на труды таких отечественных ученых как: В.М. Вишневский, А.К. Самуйлов, Ю.В. Гайдамака, О.Ю. Першин, О.В. Семенова, А.А. Ларионов, Д.В. Козырев и другие.
Наряду с отчественными работами диссертант обращался к трудам зарубежных авторов: Е.С. Кавальканте, Х. Лиу, А.Б. Рейз, Д.Ли, Д.П. Хейман, С. Шен, Д. Бендель, У. М. Амин, Б. Брахим, Х.Э. Кызылёз и другие. 

В работах указанных авторов рассматриваются задачи оптимального синтеза топологии сети и исследуются вопросы анализа сетей, в том числе рассматриваются оценки характеристик сетей с помощью стохастических моделей сетей массового обслуживания. 
% Таким образом актуальность задачи синтеза топологии сети в составе комплексного проектирования беспроводных сетей предопределили и положили начало целям и задачам данного диссертационного исследования. 

Исследования доведены до разработки алгоритмов и программ, применимых для решения практических задач. Проведены численные эксперименты, позволяющие оценить характеристики вычислительных методов.


{\objectresearch} БШС специальных типов, широко представленных на практике: БШС для контроля линейных траекторий и БШС с ячеистой топологией (mesh) для контроля объектов, рассредоточенных на некоторой территории.

{\subjectresearch} является синтез топологической структуры беспроводной широкополосной сети.

{\aim} состоит в разработке моделей и методов оптимального размещения базовых станций для БШС указанных типов, определяющего топологию таких сетей.

Для достижения поставленной цели были решены следующие {\tasks}:
\begin{enumerate}[beginpenalty=10000] % https://tex.stackexchange.com/a/476052/104425
  \item сделан анализ современного состояния и перспектив развития БШС для  обоснования  актуальности исследований в области оптимизации их топологии; 
  \item проанализирована методика проектирования современных БШС с целью определения технологических требований к решению задачи синтеза оптимальной топологии сети и предложены формулы расчета технологических параметров БШС, необходимых для решения задач размещения базовых станций;
  \item построены математические модели для задачи оптимального размещения базовых станций БШС с линейной топологией, разработан алгоритм типа метода ветвей и границ (МВиГ) для решения указанной задачи, предложена итерационная процедура нахождения последовательности лучших решений в размещении базовых станций в рамках комплексного проектирования сети;
  \item разработаны математические модели для проектирования и анализа БШС с ячеистой топологией;
  \item разработаны методы оценки характеристик производительности сети с помощью данных имитационного моделирования и методов машинного обучения. 
\end{enumerate}


{\novelty} результатов исследования заключается в следующем:
\begin{enumerate}[beginpenalty=10000] % https://tex.stackexchange.com/a/476052/104425
  \item построены математические модели в виде экстремальной комбинаторной задачи и задачи ЦЛП для оптимального размещения базовых станций при проектировании БШС с линейной топологией;  
  \item разработан специальный алгоритм МВиГ для решения сформулированной экстремальной комбинаторной задачи.;
  \item разработана итерационная процедура нахождения последовательности лучших решений для задачи размещения базовых станций в рамках комплексного проектирования БШС с линейной топологией;
  \item разработаны математические модели для задач проектирования БШС с ячеистой топологией;
%   \item  \fixme{разработаны алгоритмы для анализа выполнения технологических требований и оптимального размещения базовых станций для  БШС с ячеистой топологией};
%   \item работаны имитационные модели многофазной сети массового обслуживания с зависимым временем обслуживания;
  \item разработаны модели прогнозирования оценок характеристик производительности сети с помощью методов машинного обучения. 
\end{enumerate}

{\influence}. Разработанные модели и методы позволяют повысить качество и эффективность проектирования БШС для распространенных типов таких сетей.

% {\elaboration}. Исследования доведены до разработки алгоритмов и программ, применимых для решения практических задач. Проведены численные эксперименты, позволяющие оценить характеристики вычислительных методов.

{\methods} В работе использованы теория и методы оптимизации на конечных множествах и теории массового обслуживания

{\defpositions}
% \begin{enumerate}[beginpenalty=10000] % https://tex.stackexchange.com/a/476052/104425
%   \item математические модели линейной задачи и алгоритм решения линейной комбинаторной задачи методом ветвей и границ;
%   \item итерационная процедура построения последовательности топологий; 
%   \item математическая модель задачи покрытия множества рассредоточенных объектов; 
%   \item имитационная модель сети массового обслуживания с зависимым распределением времени обслуживания;
%   \item регрессионные модели характеристики производительности сети, полученные с помощью методов машинного обучения.
% \end{enumerate}

\begin{enumerate}[beginpenalty=10000] % https://tex.stackexchange.com/a/476052/104425
    \item математические модели в виде экстремальной комбинаторной задачи и
    задачи ЦЛП для оптимального размещения базовых станций при
    проектировании БШС с линейной топологией;
    \item специальный алгоритм МВ и Г для решения сформулированной
    экстремальной комбинаторной задачи;
    \item итерационная процедура нахождения последовательности лучших
    решений для задачи размещения базовых станций в рамках комплексного
    проектирования БШС с линейной топологией;
    \item математические модели для задач проектирования БШС с ячеистой
    топологией;
    \item имитационной модели многофазной сети массового обслуживания с
    зависимым временем обслуживания;
    \item модели прогнозирования оценок характеристик производительности сети с
    помощью методов машинного обучения.
  \end{enumerate}

% В папке Documents можно ознакомиться с решением совета из Томского~ГУ (в~файле \verb+Def_positions.pdf+), где обоснованно даются рекомендации по~формулировкам защищаемых положений.

% {\reliability} \fixme{полученных результатов обеспечивается \ldots \ Результаты находятся в соответствии с результатами, полученными другими авторами.}


{\probation}
Основные положения и результаты исследования представлены и обсуждены на научных конференциях «Губкинский университет в решении вопросов нефтегазовой отрасли России» (Москва, 17-21 сентября 2018); «13-е Всероссийское совещание по проблемам управления» (Москва, 17-20 июня 2019); «International Conference on Distributed Computer and Communication Networks: Control, Computation, Communications» (Москва, 22-27 сентября 2019), «Губкинский университет в решении вопросов нефтегазовой отрасли России» (Москва, 24-26 сентября 2019); «Управление развитием крупномасштабных систем» (Москва, 1-3 октября 2019); «Information and Telecommunication Technologies and Mathematical Modeling of High-Tech Systems» (Москва, 13-17 апреля 2020); «Computer-aided technologies in applied mathematics» (Томск, сентябрь 2020); «International Conference on Distributed Computer and Communication Networks: Control, Computation, Communications» (Москва, 14-18 сентября 2020); «Information and Telecommunication Technologies and Mathematical Modeling of High-Tech Systems» (Москва, 19-23 апреля 2021);


{\contribution} Основные результаты диссертации, выносимые на защиту получены автором самостоятельно.

\ifnumequal{\value{bibliosel}}{1}
{%%% Встроенная реализация с загрузкой файла через движок bibtex8. (При желании, внутри можно использовать обычные ссылки, наподобие `\cite{vakbib1,vakbib2}`).
    {\publications} Основные результаты по теме диссертации изложены в 12 печатных изданиях, 1 из которых издана в журнале, рекомендованных ВАК, 2 — в периодических научных журналах, индексируемых Web of Science и Scopus, 9 — в сборниках трудов конференции. 
}%
{%%% Реализация пакетом biblatex через движок biber
    \begin{refsection}[bl-author, bl-registered]
        % Это refsection=1.
        % Процитированные здесь работы:
        %  * подсчитываются, для автоматического составления фразы "Основные результаты ..."
        %  * попадают в авторскую библиографию, при usefootcite==0 и стиле `\insertbiblioauthor` или `\insertbiblioauthorgrouped`
        %  * нумеруются там в зависимости от порядка команд `\printbibliography` в этом разделе.
        %  * при использовании `\insertbiblioauthorgrouped`, порядок команд `\printbibliography` в нём должен быть тем же (см. biblio/biblatex.tex)
        %
        % Невидимый библиографический список для подсчёта количества публикаций:
        \printbibliography[heading=nobibheading, section=1, env=countauthorvak,          keyword=biblioauthorvak]%
        \printbibliography[heading=nobibheading, section=1, env=countauthorwos,          keyword=biblioauthorwos]%
        \printbibliography[heading=nobibheading, section=1, env=countauthorscopus,       keyword=biblioauthorscopus]%
        \printbibliography[heading=nobibheading, section=1, env=countauthorconf,         keyword=biblioauthorconf]%
        \printbibliography[heading=nobibheading, section=1, env=countauthorother,        keyword=biblioauthorother]%
        \printbibliography[heading=nobibheading, section=1, env=countregistered,         keyword=biblioregistered]%
        \printbibliography[heading=nobibheading, section=1, env=countauthorpatent,       keyword=biblioauthorpatent]%
        \printbibliography[heading=nobibheading, section=1, env=countauthorprogram,      keyword=biblioauthorprogram]%
        \printbibliography[heading=nobibheading, section=1, env=countauthor,             keyword=biblioauthor]%
        \printbibliography[heading=nobibheading, section=1, env=countauthorvakscopuswos, filter=vakscopuswos]%
        \printbibliography[heading=nobibheading, section=1, env=countauthorscopuswos,    filter=scopuswos]%
        %
        \nocite{*}%
        %
        {\publications} Основные результаты по теме диссертации изложены в~\arabic{citeauthor}~печатных изданиях,
        \arabic{citeauthorvak} из которых изданы в журналах, рекомендованных ВАК\sloppy%
        \ifnum \value{citeauthorscopuswos}>0%
            , \arabic{citeauthorscopuswos} "--- в~периодических научных журналах, индексируемых Web of~Science и Scopus\sloppy%
        \fi%
        \ifnum \value{citeauthorconf}>0%
            , \arabic{citeauthorconf} "--- в~сборниках трудов конференции.
        \else%
            .
        \fi%
        \ifnum \value{citeregistered}=1%
            \ifnum \value{citeauthorpatent}=1%
                Зарегистрирован \arabic{citeauthorpatent} патент.
            \fi%
            \ifnum \value{citeauthorprogram}=1%
                Зарегистрирована \arabic{citeauthorprogram} программа для ЭВМ.
            \fi%
        \fi%
        \ifnum \value{citeregistered}>1%
            Зарегистрированы\ %
            \ifnum \value{citeauthorpatent}>0%
            \formbytotal{citeauthorpatent}{патент}{}{а}{}\sloppy%
            \ifnum \value{citeauthorprogram}=0 . \else \ и~\fi%
            \fi%
            \ifnum \value{citeauthorprogram}>0%
            \formbytotal{citeauthorprogram}{программ}{а}{ы}{} для ЭВМ.
            \fi%
        \fi%
        % К публикациям, в которых излагаются основные научные результаты диссертации на соискание учёной
        % степени, в рецензируемых изданиях приравниваются патенты на изобретения, патенты (свидетельства) на
        % полезную модель, патенты на промышленный образец, патенты на селекционные достижения, свидетельства
        % на программу для электронных вычислительных машин, базу данных, топологию интегральных микросхем,
        % зарегистрированные в установленном порядке.(в ред. Постановления Правительства РФ от 21.04.2016 N 335)
    \end{refsection}%
    \begin{refsection}[bl-author, bl-registered]
        % Это refsection=2.
        % Процитированные здесь работы:
        %  * попадают в авторскую библиографию, при usefootcite==0 и стиле `\insertbiblioauthorimportant`.
        %  * ни на что не влияют в противном случае
        \nocite{vakbib2}%vak
        % \nocite{patbib1}%patent
        % \nocite{progbib1}%program
        \nocite{bib1}%other
        \nocite{confbib1}%conf
    \end{refsection}%
        %
        % Всё, что вне этих двух refsection, это refsection=0,
        %  * для диссертации - это нормальные ссылки, попадающие в обычную библиографию
        %  * для автореферата:
        %     * при usefootcite==0, ссылка корректно сработает только для источника из `external.bib`. Для своих работ --- напечатает "[0]" (и даже Warning не вылезет).
        %     * при usefootcite==1, ссылка сработает нормально. В авторской библиографии будут только процитированные в refsection=0 работы.
}


% При использовании пакета \verb!biblatex! будут подсчитаны все работы, добавленные
% в файл \verb!biblio/author.bib!. Для правильного подсчёта работ в~различных
% системах цитирования требуется использовать поля:
% \begin{itemize}
%         \item \texttt{authorvak} если публикация индексирована ВАК,
%         \item \texttt{authorscopus} если публикация индексирована Scopus,
%         \item \texttt{authorwos} если публикация индексирована Web of Science,
%         \item \texttt{authorconf} для докладов конференций,
%         \item \texttt{authorpatent} для патентов,
%         \item \texttt{authorprogram} для зарегистрированных программ для ЭВМ,
%         \item \texttt{authorother} для других публикаций.
% \end{itemize}
% Для подсчёта используются счётчики:
% \begin{itemize}
%         \item \texttt{citeauthorvak} для работ, индексируемых ВАК,
%         \item \texttt{citeauthorscopus} для работ, индексируемых Scopus,
%         \item \texttt{citeauthorwos} для работ, индексируемых Web of Science,
%         \item \texttt{citeauthorvakscopuswos} для работ, индексируемых одной из трёх баз,
%         \item \texttt{citeauthorscopuswos} для работ, индексируемых Scopus или Web of~Science,
%         \item \texttt{citeauthorconf} для докладов на конференциях,
%         \item \texttt{citeauthorother} для остальных работ,
%         \item \texttt{citeauthorpatent} для патентов,
%         \item \texttt{citeauthorprogram} для зарегистрированных программ для ЭВМ,
%         \item \texttt{citeauthor} для суммарного количества работ.
% \end{itemize}
% % Счётчик \texttt{citeexternal} используется для подсчёта процитированных публикаций;
% % \texttt{citeregistered} "--- для подсчёта суммарного количества патентов и программ для ЭВМ.

% Для добавления в список публикаций автора работ, которые не были процитированы в
% автореферате, требуется их~перечислить с использованием команды \verb!\nocite! в
% \verb!Synopsis/content.tex!.
 % Характеристика работы по структуре во введении и в автореферате не отличается (ГОСТ Р 7.0.11, пункты 5.3.1 и 9.2.1), потому её загружаем из одного и того же внешнего файла, предварительно задав форму выделения некоторым параметрам

%Диссертационная работа была выполнена при поддержке грантов \dots

%\underline{\textbf{Объем и структура работы.}} Диссертация состоит из~введения,
%четырех глав, заключения и~приложения. Полный объем диссертации
%\textbf{ХХХ}~страниц текста с~\textbf{ХХ}~рисунками и~5~таблицами. Список
%литературы содержит \textbf{ХХX}~наименование.

\pdfbookmark{Содержание работы}{description}                          % Закладка pdf
\section*{Содержание работы}
Во \underline{\textbf{введении}} обосновывается актуальность
исследований, проводимых в~рамках данной диссертационной работы,
приводится обзор научной литературы по~изучаемой проблеме,
формулируется цель, ставятся задачи работы, излагается научная новизна
и практическая значимость представляемой работы. В~последующих главах
сначала описывается общий принцип, позволяющий \dots, а~потом идёт
апробация на частных примерах: \dots  и~\dots.


\underline{\textbf{Первая глава}} посвящена анализу современного развития беспроводных сетей на месторождении. На сегодняшний день свое обширное примение нашли беспроводные MESH-сети, такие как Zigbee, WirelessHART и ISA100.11a. Данные протоколы ячеистой сети являются энергоэффективными, работающими на скорость до 250 кбит/с. Данной скорости вполне хватает для передачи информации о результатах периодического опроса датчиков \cite{Maltsev2017}. 

Анализ типовой архитектуры АСУ ТП позволяет выделить четыре зоны ответственности в плане реализации мероприятий безопасности беспроводных соединений \cite{Rimsha2018}:

\begin{enumerate}
    \item зона сбора и передачи данных c полевых измерительных приборов на основе беспроводной сенсорной сети;
    \item зона беспроводной передача данных между серверами ввода/вывода (SCADA) и ПЛК, использующие подключение через радиомодем или устройство широкополосного доступа;
    \item интерфейсную зону диспетчерского контроля и управления, где работают операторы и диспетчеры с целью наблюдения за ходом выполнения технологического процесса;
    \item зону выхода SCADA систем во внешнюю сеть для передачи данных в центральный офис.
\end{enumerate}

Первая зона представляет собой совокупность технических средств сбора информации, объединенных в ячеистую сеть. Как правило на данном уровне АСУТП расстояние между сенсорами не играет существенную роль, протоколы MESH-сетей, объединяя узлы в самоорганизующую сеть вполне справляются со своей задачей передачи низкоскоростного трафика. Во второй зоне отвественности система управления объединяет производственне объекты, расстояния между которыми могут достигать порядка нескольких километров. В каждом таком технологическом объекте присутсвует радиомодем, представляющий собой шлюз, в который поступает весь объем информации с сенсоров. Далее с радиомодема вся информация по дополнительному резервному каналу связи поступает далее на верхний уровень АСУ ТП. В качестве резервного канала используют беспроводную широкополсную сеть. Помимо низкоскоростного трафика, с учетом цифровизации месторождений на сегодяшний день растет доля мультимедийного трафика, который целесообразно передавать по протоколу семейства IEEE 802.11 в силу его высокой пропускной способности, легкой масшитабируемости и минимальных затрат на монтаж и эксплуатацию. 

В обоих случаев перед производством стоит научно-техническая проблема организации множества распределенных объектов в единую цифровую сеть для мониторинга и контроля за производством. При проектировании такой сети Здесь одной из главных задач является задача синтеза топологии беспроводной сети.



\underline{\textbf{Вторая глава}} посвящена исследованию

\underline{\textbf{Третья глава}} посвящена исследованию

Можно сослаться на свои работы в автореферате. Для этого в файле
\verb!Synopsis/setup.tex! необходимо присвоить положительное значение
счётчику \verb!\setcounter{usefootcite}{1}!. В таком случае ссылки на
работы других авторов будут подстрочными.
Изложенные в третьей главе результаты опубликованы в~\cite{vakbib1, vakbib2}.
Использование подстрочных ссылок внутри таблиц может вызывать проблемы.

В \underline{\textbf{четвертой главе}} приведено описание

\FloatBarrier
\pdfbookmark{Заключение}{conclusion}                                  % Закладка pdf
В \underline{\textbf{заключении}} приведены основные результаты работы, которые заключаются в следующем:
%% Согласно ГОСТ Р 7.0.11-2011:
%% 5.3.3 В заключении диссертации излагают итоги выполненного исследования, рекомендации, перспективы дальнейшей разработки темы.
%% 9.2.3 В заключении автореферата диссертации излагают итоги данного исследования, рекомендации и перспективы дальнейшей разработки темы.

\begin{enumerate}
    \item Проведен анализ методики проектирования современных беспроводных широкополосных сетей. В рамках такой методики были исследованы проблемы синтеза топологии беспроводных сетй вдоль протяженных транспортных магистралей: автомобильные дороги, трубопроводные магистрали, лини метрополитена, железные дороги. 
    \item Предложена новая математическая модель в виде задачи целочисленного линейного программирования оптимального размещения базовых станций с линейной топологией.
    \item Представлена новая математическая модель задачи оптимального размещения БС в виде комбинаторной модели в экстремальной форме. 
    % Данная модель учитывает специфику задачи для нахождения оптимального решения.
    \item Для комбинаторной модели разработан новый специальный алгоритм типа ветвей и границ, учитывающий специфике решения задачи размещения базовых станций широкополосной сети вдоль протежянных транспортных магистралей.  
    
    % Представлены результаты сравнения поиска оптимального решения с помощью МВиГ с алгоритмами решения задачи в общем виде.
    \item В рамках комплексного проектирования БШС представлена новая итерационная процедура нахождения последовательности лучших решений задачи оптимального размещения базовых станций для случая, когда найденное оптимальное решение построения топологии сети не удовлетворяет  критериями функционирования БШС, проверяемых на следующих этапах проектирования.
    \item Предложена новая математическая модель виде задачи частично целочисленного линейного программирования оптимального размещения базовых станций для покрытия множества рассредоточенных объектов. 
    \item Разработан программный комплекс для расчета задачи оптимального размещения базовых станций с помощью нового алгоритма типа ветвей и границ.
    \item Представлены результаты численных экспериментов, доказывающие эффективность предложенных моделей и методов для решения задачи синтеза топологии при проектировании БШС.
\end{enumerate}

% \begin{enumerate}
%     \item Был проведен анализ методики проектирования современных БШС. В рамках такой методики были исследованы проблемы синтеза топологии БШС вдоль протяженных участков: трубопроводные магистрали, протяженные автомобильные дороги. 
%     \item Была предложена математическая модель в виде задачи ЦЛП размещения БС с линейной топологией.
%     \item Была представлена математическая модель экстремальной задачи оптимального размещения БС в комбинаторной форме. 
%     % Данная модель учитывает специфику задачи для нахождения оптимального решения.
%     \item Для комбинаторной модели был разработан специальный алгоритм типа ветвей и границ. Представлены результаты сравнения поиска оптимального решения с помощью МВиГ с алгоритмами решения задачи в общем виде.
%     \item В рамках комплексного проектирования была представлена итерационная процедура нахождения последовательности лучших решений задачи оптимального размещения для случая, когда найденное оптимальное решение построение топологии сети не удовлетворяет некоторым критериями функционирования БШС, проверяемых на этапе моделирования процесса передачи данных.
%     \item Предложена математическая модель оптимального размещения БС для покрытия множества рассредоточенных объектов.
%     \item Разработан программный комплекс для расчета задачи оптимального размещения БС  с помощью предложенного алгоритма типа ветвей и границ.
% \end{enumerate}




% \begin{enumerate}
%   \item построены математические модели в виде экстремальной комбинаторной задачи и задачи ЦЛП для оптимального размещения базовых станций при проектировании БШС с линейной топологией;
%   \item представлен алгоритм метода ветвей и границ для задачи размещения базовых станций с линейной топологией; 
%   \item разработана итерационная процедура нахождения последовательности лучших решений для задачи размещения базовых станций в рамках комплексного проектирования БШС с линейной топологией;
%   \item разработаны математические модели для задач проектирования БШС для покрытия множества рассредоточенныз объектов;
%   \item разработаны модели прогнозирования оценок характеристик производительности сети с помощью методов машинного обучения.
% \end{enumerate}


\pdfbookmark{Литература}{bibliography}                                % Закладка pdf
При использовании пакета \verb!biblatex! список публикаций автора по теме
диссертации формируется в разделе <<\publications>>\ файла
\verb!common/characteristic.tex!  при помощи команды \verb!\nocite!

\ifdefmacro{\microtypesetup}{\microtypesetup{protrusion=false}}{} % не рекомендуется применять пакет микротипографики к автоматически генерируемому списку литературы
\urlstyle{rm}                               % ссылки URL обычным шрифтом
\ifnumequal{\value{bibliosel}}{0}{% Встроенная реализация с загрузкой файла через движок bibtex8
    \renewcommand{\bibname}{\large \bibtitleauthor}
    \nocite{*}
    \insertbiblioauthor           % Подключаем Bib-базы
    %\insertbiblioexternal   % !!! bibtex не умеет работать с несколькими библиографиями !!!
}{% Реализация пакетом biblatex через движок biber
    % Цитирования.
    %  * Порядок перечисления определяет порядок в библиографии (только внутри подраздела, если `\insertbiblioauthorgrouped`).
    %  * Если не соблюдать порядок "как для \printbibliography", нумерация в `\insertbiblioauthor` будет кривой.
    %  * Если цитировать каждый источник отдельной командой --- найти некоторые ошибки будет проще.
    %

    %% authorvak
    \cite{IvanovVAK2019}%
    %
    %% authorwos
    % \nocite{wosbib1}%
    %
    %% authorscopus
    % \nocite{scbib1}%
    \nocite{Ivanov2019}%
    
    \nocite{Mukhtarov2020}%
    %
    %% authorpathent
    % \nocite{patbib1}%
    %
    %% authorprogram
    % \nocite{progbib1}%
    %
    %% authorconf
    \nocite{VishnevskyMukhtarovPershinDCCN2020_RSCI}
    \nocite{LazarevaLarionovMukhtarovITTMM2020_RSCI}
    \nocite{MukhtarovIvanovPershinDCCN2019_RSCI}
    \nocite{MukhtarovPershinVSPU2019_RSCI}
    \nocite{MukhtarovPershinMLSD2019materials_RSCI}
    \nocite{MukhtarovPershinMLSD2019works_RSCI}
    %
    %% authorother
    \nocite{VishnevskyLarionovMukhtarovICAM2020_RSCI}
    \nocite{MukhtarovPershinGUBKIN2019_RSCI}
    \nocite{MukhtarovPershinGUBKIN2018_RSCI}%

    \ifnumgreater{\value{usefootcite}}{0}{
        \begin{refcontext}[labelprefix={}]
            \ifnum \value{bibgrouped}>0
                \insertbiblioauthorgrouped    % Вывод всех работ автора, сгруппированных по источникам
            \else
                \insertbiblioauthor      % Вывод всех работ автора
            \fi
        \end{refcontext}
    }{
        \ifnum \totvalue{citeexternal}>0
            \begin{refcontext}[labelprefix=A]
                \ifnum \value{bibgrouped}>0
                    \insertbiblioauthorgrouped    % Вывод всех работ автора, сгруппированных по источникам
                \else
                    \insertbiblioauthor      % Вывод всех работ автора
                \fi
            \end{refcontext}
        \else
            \ifnum \value{bibgrouped}>0
                \insertbiblioauthorgrouped    % Вывод всех работ автора, сгруппированных по источникам
            \else
                \insertbiblioauthor      % Вывод всех работ автора
            \fi
        \fi
        %  \insertbiblioauthorimportant  % Вывод наиболее значимых работ автора (определяется в файле characteristic во второй section)
        \begin{refcontext}[labelprefix={}]
            \insertbiblioexternal            % Вывод списка литературы, на которую ссылались в тексте автореферата
        \end{refcontext}
        % Невидимый библиографический список для подсчёта количества внешних публикаций
        % Используется, чтобы убрать приставку "А" у работ автора, если в автореферате нет
        % цитирований внешних источников.
        \printbibliography[heading=nobibheading, section=0, env=countexternal, keyword=biblioexternal, resetnumbers=true]%
    }
}
\ifdefmacro{\microtypesetup}{\microtypesetup{protrusion=true}}{}
\urlstyle{tt}                               % возвращаем установки шрифта ссылок URL
