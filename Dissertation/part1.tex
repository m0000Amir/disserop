\chapter{Определение технологических параметров БШС, необходимых для решения задач размещения базовых станций}\label{ch:ch1}

\fixme{
  \begin{itemize}
    \item про БШС на промылсе;
    \item Про проектироваение БШС
      \begin{itemize}
        \item каждый этап
        \item ----
        \item ----
        \item Моделирование. Пару слов сказать про простейший поток и аналитический расчет времени межконцевой задержки.
      \end{itemize}
    \item Расчет радиуса связи
  \end{itemize}
}



Современные беспроводные широкополосные сети связи (БШС), обладая рядом преимуществ, нашли свое широкое применение в задачах мониторинга и управления различных   производственных или гражданских объектов, технологических установок, движущихся транспортных средств и т.п. К ряду таких преимуществ можно отнести возможность получения информации с любой точки контролируемой территории, быстрый ввод в эксплуатацию, сокращение капитальных затрат на создание и эксплуатацию сети, высокая гибкость, мобильность и масштабируемость. 


Нефтегазовые объекты часто расположены в труднодоступной местности на обширной территории в несколько киллометров. Данный фактор является ключевым преимуществом беспроводных технологий для рпазвертывания по сравнению с кабельными коммуникациями.

Беспроводная связь в автоматизации промышленного производства способствует осуществлению производственных процессов более экономически эффективно, гибко и надежно, а также позволяет реализовывать новые концепции автоматизации \cite{Gost62657}.

Не маловажную роль на месторождениях играет безопасность. Технологические объекты на нефтяных или газовых месторождениях, оснащенных широкополосным подключением, позволяют соответствовать современным концепциям и требованиям в сфере безопасности персонала и безопасности имущества, включая охрану с использованием беспроводных камер видеонаблюдения. Для предоставления доступа к объекту могут использоваться дополнительные возможности, такие как считывание номерных знаков и распознавание лиц. А благодаря использованию тепловых камер можно контролировать риски отключения и перегрузки даже с помощью периодических снимков оборудования на промысле. 

Внедрение БШС особенно хорошо подходят для систем видеонаблюдения, поскольку позволяют расположить камеры там, где они нужны, а не там, где удобно для подключения к проводной сети, при этом не нужно постоянно платить за трафик.


Для обеспечения высокого качества беспроводной связи необходимо проводить грамотное проектирование БШС. Существуют различные подходы к проектированию беспроводных сетей. Для одних задачей является максимальная зона покрытия, для других -- достижения максимальной производительности передачи данных, для третьих -- нахождения баланса между зоной охвата и производительностью \cite{Proletarsky}. В диссертации будут предложены модели и методы оптимального размещения БС БШС, целью которых является максимальная зона охвата.  Процесс проектирования современной БШС, как правило, для такого подхода имеет следующие основные этапы (Рис. \cref{fig:part1_design_stages}):

\begin{figure}[h!]
  \centering
   \includegraphics[width=0.8\textwidth]{design_stages.png}
\caption{Этапы проектирования БШС.}
\label{fig:part1_design_stages}
\end{figure}

Любое проектирование БШС всегда начинается с первоначального обследования местности. В данный этап входят задачи радиобследования и радиопланирования. оценки реальных размеров области контроля, наличие стационарных инженерно-технических соооружений, мешающих передачи сигнала, такими как металлические конструкции, перекрытия, стены и т.д. При развертывании БШС в открытой местности также немаловажную роль играет наличие перепада высот. В ходе выполнения комплекса работ на местности, определяются возможные точки размещения оборудования \cite{Dunaitsev2017}. На основе результатов данного этапа проводится выбор типо моделй оборудования для дальнейшего их размещения и организации сети.

Производительность и дальность действия беспроводных сетей не безграничны. При их проектировании стоит учитывать множество параметров: частота, скорость, мощность излучения \cite{Proletarsky}. На этапе выбора оборудования необходимо определиться с протоколом будущей БШС и  подготовить необходимый комплекс технических средств для развертывания будущей сети. БС является основопологающим устройством будущей сети, которая отвечает за покрытие задданной области. Покрытие в свою очередь зависит от мощности передатчика устройства, усиления антенн, чувствительности приемного устройства.


После определения множества возможных точек размещения БС на этапе обследования местности и выборе возможных типов и моделей оборудований можно переходить непосредственно к размещению БС и определению топологической структуры сети. Этап выбора топологической структуры будущей сети является ключевой проблемой данной диссертации. В рамках данной проблемы будут предложены модели и методы оптимального размещения БС для организации БШС.

Проектирования БШС требует учета всего комплекса критериев качества и ограничений для проектируемой беспроводной сети \cite{Vasiliev2015}. После решения задачи синтеза топологии, для полученного размещения решаются задачи оцненки качества. Такими задачами являются расчет надежности всех элементов сети \cite{Wankpo2020, Krishnamoorthy2021, Kozyrev2019}, оценка характеристик качества канала, вероятности потери пакетов, пропускной способности, времени доставки сообщений в сети \cite{Gorbunova2020, Larionov2019, Vishnevsky2016_Methods_of_performance, Vishnevsky2016_Review_of_methodology, Wang2017, Sandmann2012, Baumann2017}. Оценка межкоцневой задержки сети \cite{Wang2017, Sandmann2012}

ТМО и МО \cite{Lovas2021, SatyaHermanto2018}

\section{Расчет дальности действия связи}


\fixme{Перед тем как приступить к задаче ЦЛП необходимо рассчитать характеристики станции: радиус связи $R_{jq}$ и радиус покрытия $r_j$.}

\fixme{При развертывания сети необходимо обеспечить максимальное покрытие данного участка связь между шлюзами через систему размещенных базовых станций беспроводной широкополосной сети}.

Для оценки производительности канала связи воспользуемся уравнением энергетического потенциала. Полное уравнение можно записать следующим образом:

% It is essential during deployment to provide maximum coverage of a given area and ensure communication between the placed base stations in the wireless broadband network. 

% Link Budget is a way of estimation of communication link's performance while accounting for the system's power, gains, and losses for both the transmitter and receiver. The complete equation can be written as follows:

\begin{equation}
  \label{eq:part3_link_budget}
  P_{tr} - L_{tr} + G_{tr} - L_{fs} + G_{recv} - L_{recv} = SOM + P_{recv},
\end{equation}
где:

\begin{itemize}

  \item $P_{tr}$ -- мощность передатчика, дБм;

  \item $L_{tr}$ -- потери сигнала на антенном кабеле и разъемах передающего тракта, дБ;

  \item $G_{tr}$ -- усиление антенны передатчика, дБ;

  \item $L_{fs}$ -- потери в свободном пространстве, дБ;

  \item $G_{recv}$ -- усиление антенны приемника, дБ;

  \item $L_{recv}$ -- потери сигнала на антенном кабеле и разъемах приемного тракта, дБ;

  \item $P_{recv}$ -- чувствительность приемника, дБм;
  
  \item $SOM$ -- запас на замирание сигнала, дБ.

\end{itemize}

Усиление антенны описывает фокусирование переданного или полученного сигнала. Значения даны относительно полуволнового диполя или теоеритического изотропного излучателя \cite{Gost62657}.

Запас на замирание сигнала, SOM,  учитывает все возможные факторы отрицательно влияющие на дальность связи. К таким факторам относятся:

\begin{itemize}
  \item температурный дрейф чувствительности приемника и выходной мощности передатчика;
  \item влияние погодных условий на передачу сигнала: туман, снег, дождь;
  \item  потери в антенно-фидерном тракте, возникающие из-за рассогласования фидера и антенны.
\end{itemize}

Минимальнная значения величины запаса на замирание  $SOM$ должна быть не меньше  10 дБ. Считается, что 10-ти децибельный запас по усилению достаточен для инженерного расчета, но на практике зачастую используют значение $20 ... 30$ дБ \cite{Proletarsky}.

\fixme{Энергетический потенциал указывает на качество канала передачи радиосигналов}.

Мощность принимаемой антенны рассчитывается из уравнения передачи Фрииса:

\begin{displaymath}
  \label{eq:part3_Friis}
  \frac{P_{recv}}{P_{tr}} = G_{tr}G_{recv}\left(\frac{c}{4\pi R f} \right)^2,
\end{displaymath}
где
$c$ --  скорость света,
$f$ -- частота, 
$R$ рассточние между приемной и передающей антенной.

% The Free Space Path Loss ($ FSPL $) equation defines the propagation signal loss between two antennas through free space (air):

Уравнение потерь в свободном пространстве (Free Space Path Loss, $FSPL $) определяет потерю сигнала при распространении между двумя антеннами в свободном пространстве (в воздухе):

\begin{equation}
  \label{eq:part3_FSPL}
  FSPL = \left(\frac{4\pi R f}{c} \right)^2.
\end{equation}

Формула \cref{eq:part3_FSPL}, выраженная в децибеллах будет выражаться как

\begin{equation}
  \label{eq:part3_L_fs}
  L_{fs} = 20 \lg{F} + 20\lg{R} + K,
  \end{equation}
где $F$ -- центральная частота, на котором работает канал связи, $R$ -- рассточние между приемной и передающей антенной и $K$ -- константа.

Константа $K$ зависит от размерностей частоты и расстояния:

\begin{itemize}
  \item для чистоты, выраженной в ГГц, и рассчтояния, выраженная в км, константа $K$ равна 92.45;
  \item для чистоты, выраженной в МГц, и рассчтояния, выраженная в км, константа $K$ равна 32.4;
  \item для чистоты, выраженной в МГц, и рассчтояния, выраженная в м, константа $K$ равна -27.55.
\end{itemize} 

Потерия $L_{fs}$ выразим из формулы \cref{eq:part3_link_budget} как:

\begin{equation}
  \label{eq:part3_L_fs_from_link_budget}
  L_{fs} = P_{tr} - L_{tr} + G_{tr} + G_{recv} - L_{recv} - SOM - P_{recv}.
\end{equation}

Радиус связи получаем из уравнений \cref{eq:part3_L_fs} и \cref{eq:part3_L_fs_from_link_budget}:

\begin{equation}
  \label{eq:part3_D}
  R = 10^{\left(\frac{L_{fs} - 20\lg{F} - K}{20}\right)}.
\end{equation}

Используя формулу \cref{eq:part3_D} и \cref{eq:part3_L_fs_from_link_budget}, мы можем расчитать теоретическое максимальную дальность связи $ R_{jq}$ между базовыми станциями и радиусом покрытия $ r_j $ с предположением об отсутствии препятствий, отражений, влияния контуров местности и т. д. Это допущение приемлемо для нашего случая с открытой местностью.

\begin{figure}[h!]
  \centering
   \includegraphics[width=0.8\textwidth]{link_distance.pdf}
\caption{Соеденинение между станциями.}
\label{fig:part3_link_distance}
\end{figure}

Для расчета дальности связи $R_{jq}$ (Рис. \cref{fig:part3_link_distance}), базовые станции $s_j$ и $s_q$ будут рассматриваться как станции \textit{передатчик} и \textit{приемник}, соответственно. Будем считать, что станции обрудованы направленными антеннами с усилениями $G_{tr}^{R}$ и $G_{recv}^{R}$.

\begin{figure}[h!]
  \centering
   \includegraphics[width=0.8\textwidth]{coverage.pdf}
\caption{Покрытие станции}
\label{fig:part3_coverage}
\end{figure}

Каждая базовая станция оснащена всенаправленной антенной с заданным усилением антенны $G_ {tr}^{r}$. Данная антенн необходимо для покрытия заданной области.

% Each base station is equipped with an omnidirectional antenna with given gain antenna $G_{tr}^{r}$. A station uses this antenna to cover a given area.

При вычислении радиуса покрытия $r_j$ (Рис.  \cref{fig:part3_coverage}) базовая станция будем считать \textit{передатчиком} а пользовательское устройство \textit{приемником}.

\section{Расчет межконцевой задержки}\label{part4_e2e_delay_section}

Одной из основных характеристик проектируемой сети является ее межконцевая задержка. Рассмотрим беспроводную сеть как сеть массового обслуживания (СеМО) с кросс-трафиком и с узлами $M/M/1$. По теореме Бурке \cite{Burke1956} на выходе узла $M/M/1$, а значит на входе каждой последующей фазы тоже пуассоновский поток. Интенсивность на выходе каждой фазы равна суммарной интенсивности всех входящих потоков с интенсивностями $\lambda$.

По формуле Литтла \cite{Little1961} можно рассчитать время задержки на фазе. Интенсивность времени обслуживания рассчитывается по формуле: 

\begin{displaymath}
    \mu_j = p_j / w,
\end{displaymath}
где: $p_j$ - пропускная спобоность $j$-ой станции, Мбит/с; $w$ - средний размер пакета, Мбит.

Для каждой станции коэффициент загрузки равен:


\begin{displaymath}
\rho_j= \frac{\sum{\lambda}}{\mu_j} = \frac{q \cdot \lambda}{\mu_j} <1,
\end{displaymath}

где $q$ -- число входящих потоков. Условие $\rho_j<1$ является необходимым и достаточным условием существования стационарного режима функционирования \fixme{СеМО}.

Тогда среднее время задержки по времени на каждой станции:

\begin{displaymath}
    \overline{T_j} = \frac{\rho_j}{1 - \rho_j} \cdot \frac{1}{q \cdot \lambda}.
\end{displaymath}

Тогда межконцевая задержки в сети равна

\begin{equation}
    \label{eq:end_to_end_delay}
    T^{e2e}= \sum{\overline{T_j}}.
\end{equation}

\section{Выводы по главе \cref{ch:ch1}}

% \section{Форматирование текста}\label{sec:ch1/sec1}

% Мы можем сделать \textbf{жирный текст} и \textit{курсив}.

% \section{Ссылки}\label{sec:ch1/sec2}

% Сошлёмся на библиографию.
% Одна ссылка: \cite[с.~54]{Sokolov}\cite[с.~36]{Gaidaenko}.
% Две ссылки: \cite{Sokolov,Gaidaenko}.
% Ссылка на собственные работы: \cite{vakbib1, confbib2}.
% Много ссылок: %\cite[с.~54]{Lermontov,Management,Borozda} % такой «фокус»
% %вызывает biblatex warning относительно опции sortcites, потому что неясно, к
% %какому источнику относится уточнение о страницах, а bibtex об этой проблеме
% %даже не предупреждает
% \cite{Lermontov, Management, Borozda, Marketing, Constitution, FamilyCode,
%     Gost.7.0.53, Razumovski, Lagkueva, Pokrovski, Methodology, Berestova,
%     Kriger}%
% \ifnumequal{\value{bibliosel}}{0}{% Примеры для bibtex8
%     \cite{Sirotko, Lukina, Encyclopedia, Nasirova}%
% }{% Примеры для biblatex через движок biber
%     \cite{Sirotko2, Lukina2, Encyclopedia2, Nasirova2}%
% }%
% .
% И~ещё немного ссылок:~\cite{Article,Book,Booklet,Conference,Inbook,Incollection,Manual,Mastersthesis,
%     Misc,Phdthesis,Proceedings,Techreport,Unpublished}
% % Следует обратить внимание, что пробел после запятой внутри \cite{}
% % обрабатывается ожидаемо, а пробел перед запятой, может вызывать проблемы при
% % обработке ссылок.
% \cite{medvedev2006jelektronnye, CEAT:CEAT581, doi:10.1080/01932691.2010.513279,
%     Gosele1999161,Li2007StressAnalysis, Shoji199895, test:eisner-sample,
%     test:eisner-sample-shorted, AB_patent_Pomerantz_1968, iofis_patent1960}%
% \ifnumequal{\value{bibliosel}}{0}{% Примеры для bibtex8
% }{% Примеры для biblatex через движок biber
%     \cite{patent2h, patent3h, patent2}%
% }%
% .

% \ifnumequal{\value{bibliosel}}{0}{% Примеры для bibtex8
% Попытка реализовать несколько ссылок на конкретные страницы
% для \texttt{bibtex} реализации библиографии:
% [\citenum{Sokolov}, с.~54; \citenum{Gaidaenko}, с.~36].
% }{% Примеры для biblatex через движок biber
% Несколько источников (мультицитата):
% % Тут специально написано по-разному тире, для демонстрации, что
% % применение специальных тире в настоящий момент в biblatex приводит к непоказу
% % "с.".
% \cites[vii--x, 5, 7]{Sokolov}[v"--~x, 25, 526]{Gaidaenko}[vii--x, 5, 7]{Techreport},
% работает только в \texttt{biblatex} реализации библиографии.
% }%

% Ссылки на собственные работы:~\cite{vakbib1, confbib1}.

% Сошлёмся на приложения: Приложение~\cref{app:A}, Приложение~\cref{app:B2}.

% Сошлёмся на формулу: формула~\cref{eq:equation1}.

% Сошлёмся на изображение: рисунок~\cref{fig:knuth}.

% Стандартной практикой является добавление к ссылкам префикса, характеризующего тип элемента.
% Это не является строгим требованием, но~позволяет лучше ориентироваться в документах большого размера.
% Например, для ссылок на~рисунки используется префикс \textit{fig},
% для ссылки на~таблицу "--- \textit{tab}.

% В таблице \cref{tab:tab_pref} приложения~\cref{app:B4} приведён список рекомендуемых
% к использованию стандартных префиксов.

% \section{Формулы}\label{sec:ch1/sec3}

% Благодаря пакету \textit{icomma}, \LaTeX~одинаково хорошо воспринимает
% в~качестве десятичного разделителя и запятую (\(3,1415\)), и точку (\(3.1415\)).

% \subsection{Ненумерованные одиночные формулы}\label{subsec:ch1/sec3/sub1}

% Вот так может выглядеть формула, которую необходимо вставить в~строку
% по~тексту: \(x \approx \sin x\) при \(x \to 0\).

% А вот так выглядит ненумерованная отдельностоящая формула c подстрочными
% и надстрочными индексами:
% \[
%     (x_1+x_2)^2 = x_1^2 + 2 x_1 x_2 + x_2^2
% \]

% Формула с неопределенным интегралом:
% \[
%     \int f(\alpha+x)=\sum\beta
% \]

% При использовании дробей формулы могут получаться очень высокие:
% \[
%     \frac{1}{\sqrt{2}+
%         \displaystyle\frac{1}{\sqrt{2}+
%             \displaystyle\frac{1}{\sqrt{2}+\cdots}}}
% \]

% В формулах можно использовать греческие буквы:
% %Все \original... команды заранее, ради этого примера, определены в Dissertation\userstyles.tex
% \[
%     \alpha\beta\gamma\delta\originalepsilon\epsilon\zeta\eta\theta%
%     \vartheta\iota\kappa\varkappa\lambda\mu\nu\xi\pi\varpi\rho\varrho%
%     \sigma\varsigma\tau\upsilon\originalphi\phi\chi\psi\omega\Gamma\Delta%
%     \Theta\Lambda\Xi\Pi\Sigma\Upsilon\Phi\Psi\Omega
% \]
% \[%https://texfaq.org/FAQ-boldgreek
%     \boldsymbol{\alpha\beta\gamma\delta\originalepsilon\epsilon\zeta\eta%
%         \theta\vartheta\iota\kappa\varkappa\lambda\mu\nu\xi\pi\varpi\rho%
%         \varrho\sigma\varsigma\tau\upsilon\originalphi\phi\chi\psi\omega\Gamma%
%         \Delta\Theta\Lambda\Xi\Pi\Sigma\Upsilon\Phi\Psi\Omega}
% \]

% Для добавления формул можно использовать пары \verb+$+\dots\verb+$+ и \verb+$$+\dots\verb+$$+,
% но~они считаются устаревшими.
% Лучше использовать их функциональные аналоги \verb+\(+\dots\verb+\)+ и \verb+\[+\dots\verb+\]+.

% \subsection{Ненумерованные многострочные формулы}\label{subsec:ch1/sec3/sub2}

% Вот так можно написать две формулы, не нумеруя их, чтобы знаки <<равно>> были
% строго друг под другом:
% \begin{align}
%     f_W & =  \min \left( 1, \max \left( 0, \frac{W_{soil} / W_{max}}{W_{crit}} \right)  \right), \nonumber \\
%     f_T & =  \min \left( 1, \max \left( 0, \frac{T_s / T_{melt}}{T_{crit}} \right)  \right), \nonumber
% \end{align}

% Выровнять систему ещё и по переменной \( x \) можно, используя окружение
% \verb|alignedat| из пакета \verb|amsmath|. Вот так:
% \[
% |x| = \left\{
% \begin{alignedat}{2}
%     &&x, \quad &\text{eсли } x\geqslant 0 \\
%     &-&x, \quad & \text{eсли } x<0
% \end{alignedat}
% \right.
% \]
% Здесь первый амперсанд (в исходном \LaTeX\ описании формулы) означает
% выравнивание по~левому краю, второй "--- по~\( x \), а~третий "--- по~слову
% <<если>>. Команда \verb|\quad| делает большой горизонтальный пробел.

% Ещё вариант:
% \[
%     |x|=
%     \begin{cases}
%         \phantom{-}x, \text{если } x \geqslant 0 \\
%         -x, \text{если } x<0
%     \end{cases}
% \]

% Кроме того, для  нумерованных формул \verb|alignedat| делает вертикальное
% выравнивание номера формулы по центру формулы. Например, выравнивание
% компонент вектора:
% \begin{equation}
%     \label{eq:2p3}
%     \begin{alignedat}{2}
%         {\mathbf{N}}_{o1n}^{(j)} = \,{\sin} \phi\,n\!\left(n+1\right)
%         {\sin}\theta\,
%         \pi_n\!\left({\cos} \theta\right)
%         \frac{
%         z_n^{(j)}\!\left( \rho \right)
%         }{\rho}\,
%         &{\boldsymbol{\hat{\mathrm e}}}_{r}\,+   \\
%         +\,
%         {\sin} \phi\,
%         \tau_n\!\left({\cos} \theta\right)
%         \frac{
%         \left[\rho z_n^{(j)}\!\left( \rho \right)\right]^{\prime}
%         }{\rho}\,
%         &{\boldsymbol{\hat{\mathrm e}}}_{\theta}\,+   \\
%         +\,
%         {\cos} \phi\,
%         \pi_n\!\left({\cos} \theta\right)
%         \frac{
%         \left[\rho z_n^{(j)}\!\left( \rho \right)\right]^{\prime}
%         }{\rho}\,
%         &{\boldsymbol{\hat{\mathrm e}}}_{\phi}\:.
%     \end{alignedat}
% \end{equation}

% Ещё об отступах. Иногда для лучшей <<читаемости>> формул полезно
% немного исправить стандартные интервалы \LaTeX\ с учётом логической
% структуры самой формулы. Например в формуле~\cref{eq:2p3} добавлен
% небольшой отступ \verb+\,+ между основными сомножителями, ниже
% результат применения всех вариантов отступа:
% \begin{align*}
%     \backslash!             & \quad f(x) = x^2\! +3x\! +2         \\
%     \mbox{по-умолчанию}     & \quad f(x) = x^2+3x+2               \\
%     \backslash,             & \quad f(x) = x^2\, +3x\, +2         \\
%     \backslash{:}           & \quad f(x) = x^2\: +3x\: +2         \\
%     \backslash;             & \quad f(x) = x^2\; +3x\; +2         \\
%     \backslash \mbox{space} & \quad f(x) = x^2\ +3x\ +2           \\
%     \backslash \mbox{quad}  & \quad f(x) = x^2\quad +3x\quad +2   \\
%     \backslash \mbox{qquad} & \quad f(x) = x^2\qquad +3x\qquad +2
% \end{align*}

% Можно использовать разные математические алфавиты:
% \begin{align}
%     \mathcal{ABCDEFGHIJKLMNOPQRSTUVWXYZ} \nonumber  \\
%     \mathfrak{ABCDEFGHIJKLMNOPQRSTUVWXYZ} \nonumber \\
%     \mathbb{ABCDEFGHIJKLMNOPQRSTUVWXYZ} \nonumber
% \end{align}

% Посмотрим на систему уравнений на примере аттрактора Лоренца:

% \[
% \left\{
% \begin{array}{rl}
%     \dot x = & \sigma (y-x)  \\
%     \dot y = & x (r - z) - y \\
%     \dot z = & xy - bz
% \end{array}
% \right.
% \]

% А для вёрстки матриц удобно использовать многоточия:
% \[
%     \left(
%         \begin{array}{ccc}
%             a_{11} & \ldots & a_{1n} \\
%             \vdots & \ddots & \vdots \\
%             a_{n1} & \ldots & a_{nn} \\
%         \end{array}
%     \right)
% \]

% \subsection{Нумерованные формулы}\label{subsec:ch1/sec3/sub3}

% А вот так пишется нумерованная формула:
% \begin{equation}
%     \label{eq:equation1}
%     e = \lim_{n \to \infty} \left( 1+\frac{1}{n} \right) ^n
% \end{equation}

% Нумерованных формул может быть несколько:
% \begin{equation}
%     \label{eq:equation2}
%     \lim_{n \to \infty} \sum_{k=1}^n \frac{1}{k^2} = \frac{\pi^2}{6}
% \end{equation}

% Впоследствии на формулы~\cref{eq:equation1, eq:equation2} можно ссылаться.

% Сделать так, чтобы номер формулы стоял напротив средней строки, можно,
% используя окружение \verb|multlined| (пакет \verb|mathtools|) вместо
% \verb|multline| внутри окружения \verb|equation|. Вот так:
% \begin{equation} % \tag{S} % tag - вписывает свой текст
%     \label{eq:equation3}
%     \begin{multlined}
%         1+ 2+3+4+5+6+7+\dots + \\
%         + 50+51+52+53+54+55+56+57 + \dots + \\
%         + 96+97+98+99+100=5050
%     \end{multlined}
% \end{equation}

% Уравнения~\cref{eq:subeq_1,eq:subeq_2} демонстрируют возможности
% окружения \verb|\subequations|.
% \begin{subequations}
%     \label{eq:subeq_1}
%     \begin{gather}
%         y = x^2 + 1 \label{eq:subeq_1-1} \\
%         y = 2 x^2 - x + 1 \label{eq:subeq_1-2}
%     \end{gather}
% \end{subequations}
% Ссылки на отдельные уравнения~\cref{eq:subeq_1-1,eq:subeq_1-2,eq:subeq_2-1}.
% \begin{subequations}
%     \label{eq:subeq_2}
%     \begin{align}
%         y & = x^3 + x^2 + x + 1 \label{eq:subeq_2-1} \\
%         y & = x^2
%     \end{align}
% \end{subequations}

% \subsection{Форматирование чисел и размерностей величин}\label{sec:units}

% Числа форматируются при помощи команды \verb|\num|:
% \num{5,3};
% \num{2,3e8};
% \num{12345,67890};
% \num{2,6 d4};
% \num{1+-2i};
% \num{.3e45};
% \num[exponent-base=2]{5 e64};
% \num[exponent-base=2,exponent-to-prefix]{5 e64};
% \num{1.654 x 2.34 x 3.430}
% \num{1 2 x 3 / 4}.
% Для написания последовательности чисел можно использовать команды \verb|\numlist| и \verb|\numrange|:
% \numlist{10;30;50;70}; \numrange{10}{30}.
% Значения углов можно форматировать при помощи команды \verb|\ang|:
% \ang{2.67};
% \ang{30,3};
% \ang{-1;;};
% \ang{;-2;};
% \ang{;;-3};
% \ang{300;10;1}.

% Обратите внимание, что ГОСТ запрещает использование знака <<->> для обозначения отрицательных чисел
% за исключением формул, таблиц и~рисунков.
% Вместо него следует использовать слово <<минус>>.

% Размерности можно записывать при помощи команд \verb|\si| и \verb|\SI|:
% \si{\farad\squared\lumen\candela};
% \si{\joule\per\mole\per\kelvin};
% \si[per-mode = symbol-or-fraction]{\joule\per\mole\per\kelvin};
% \si{\metre\per\second\squared};
% \SI{0.10(5)}{\neper};
% \SI{1.2-3i e5}{\joule\per\mole\per\kelvin};
% \SIlist{1;2;3;4}{\tesla};
% \SIrange{50}{100}{\volt}.
% Список единиц измерений приведён в таблицах~\cref{tab:unit:base,
%     tab:unit:derived,tab:unit:accepted,tab:unit:physical,tab:unit:other}.
% Приставки единиц приведены в~таблице~\cref{tab:unit:prefix}.

% С дополнительными опциями форматирования можно ознакомиться в~описании пакета \texttt{siunitx};
% изменить или добавить единицы измерений можно в~файле \texttt{siunitx.cfg}.

% \begin{table}
%     \centering
%     \captionsetup{justification=centering} % выравнивание подписи по-центру
%     \caption{Основные величины СИ}\label{tab:unit:base}
%     \begin{tabular}{llc}
%         \toprule
%         Название  & Команда                 & Символ         \\
%         \midrule
%         Ампер     & \verb|\ampere| & \si{\ampere}   \\
%         Кандела   & \verb|\candela| & \si{\candela}  \\
%         Кельвин   & \verb|\kelvin| & \si{\kelvin}   \\
%         Килограмм & \verb|\kilogram| & \si{\kilogram} \\
%         Метр      & \verb|\metre| & \si{\metre}    \\
%         Моль      & \verb|\mole| & \si{\mole}     \\
%         Секунда   & \verb|\second| & \si{\second}   \\
%         \bottomrule
%     \end{tabular}
% \end{table}

% \begin{table}
%     \small
%     \centering
%     \begin{threeparttable}% выравнивание подписи по границам таблицы
%         \caption{Производные единицы СИ}\label{tab:unit:derived}
%         \begin{tabular}{llc|llc}
%             \toprule
%             Название       & Команда                 & Символ              & Название & Команда & Символ \\
%             \midrule
%             Беккерель      & \verb|\becquerel| & \si{\becquerel}     &
%             Ньютон         & \verb|\newton| & \si{\newton}                                      \\
%             Градус Цельсия & \verb|\degreeCelsius| & \si{\degreeCelsius} &
%             Ом             & \verb|\ohm| & \si{\ohm}                                         \\
%             Кулон          & \verb|\coulomb| & \si{\coulomb}       &
%             Паскаль        & \verb|\pascal| & \si{\pascal}                                      \\
%             Фарад          & \verb|\farad| & \si{\farad}         &
%             Радиан         & \verb|\radian| & \si{\radian}                                      \\
%             Грей           & \verb|\gray| & \si{\gray}          &
%             Сименс         & \verb|\siemens| & \si{\siemens}                                     \\
%             Герц           & \verb|\hertz| & \si{\hertz}         &
%             Зиверт         & \verb|\sievert| & \si{\sievert}                                     \\
%             Генри          & \verb|\henry| & \si{\henry}         &
%             Стерадиан      & \verb|\steradian| & \si{\steradian}                                   \\
%             Джоуль         & \verb|\joule| & \si{\joule}         &
%             Тесла          & \verb|\tesla| & \si{\tesla}                                       \\
%             Катал          & \verb|\katal| & \si{\katal}         &
%             Вольт          & \verb|\volt| & \si{\volt}                                        \\
%             Люмен          & \verb|\lumen| & \si{\lumen}         &
%             Ватт           & \verb|\watt| & \si{\watt}                                        \\
%             Люкс           & \verb|\lux| & \si{\lux}           &
%             Вебер          & \verb|\weber| & \si{\weber}                                       \\
%             \bottomrule
%         \end{tabular}
%     \end{threeparttable}
% \end{table}

% \begin{table}
%     \centering
%     \begin{threeparttable}% выравнивание подписи по границам таблицы
%         \caption{Внесистемные единицы}\label{tab:unit:accepted}

%         \begin{tabular}{llc}
%             \toprule
%             Название        & Команда                 & Символ          \\
%             \midrule
%             День            & \verb|\day| & \si{\day}       \\
%             Градус          & \verb|\degree| & \si{\degree}    \\
%             Гектар          & \verb|\hectare| & \si{\hectare}   \\
%             Час             & \verb|\hour| & \si{\hour}      \\
%             Литр            & \verb|\litre| & \si{\litre}     \\
%             Угловая минута  & \verb|\arcminute| & \si{\arcminute} \\
%             Угловая секунда & \verb|\arcsecond| & \si{\arcsecond} \\ %
%             Минута          & \verb|\minute| & \si{\minute}    \\
%             Тонна           & \verb|\tonne| & \si{\tonne}     \\
%             \bottomrule
%         \end{tabular}
%     \end{threeparttable}
% \end{table}

% \begin{table}
%     \centering
%     \captionsetup{justification=centering}
%     \caption{Внесистемные единицы, получаемые из эксперимента}\label{tab:unit:physical}
%     \begin{tabular}{llc}
%         \toprule
%         Название                & Команда                 & Символ                 \\
%         \midrule
%         Астрономическая единица & \verb|\astronomicalunit| & \si{\astronomicalunit} \\
%         Атомная единица массы   & \verb|\atomicmassunit| & \si{\atomicmassunit}   \\
%         Боровский радиус        & \verb|\bohr| & \si{\bohr}             \\
%         Скорость света          & \verb|\clight| & \si{\clight}           \\
%         Дальтон                 & \verb|\dalton| & \si{\dalton}           \\
%         Масса электрона         & \verb|\electronmass| & \si{\electronmass}     \\
%         Электрон Вольт          & \verb|\electronvolt| & \si{\electronvolt}     \\
%         Элементарный заряд      & \verb|\elementarycharge| & \si{\elementarycharge} \\
%         Энергия Хартри          & \verb|\hartree| & \si{\hartree}          \\
%         Постоянная Планка       & \verb|\planckbar| & \si{\planckbar}        \\
%         \bottomrule
%     \end{tabular}
% \end{table}

% \begin{table}
%     \centering
%     \begin{threeparttable}% выравнивание подписи по границам таблицы
%         \caption{Другие внесистемные единицы}\label{tab:unit:other}
%         \begin{tabular}{llc}
%             \toprule
%             Название                  & Команда                 & Символ             \\
%             \midrule
%             Ангстрем                  & \verb|\angstrom| & \si{\angstrom}     \\
%             Бар                       & \verb|\bar| & \si{\bar}          \\
%             Барн                      & \verb|\barn| & \si{\barn}         \\
%             Бел                       & \verb|\bel| & \si{\bel}          \\
%             Децибел                   & \verb|\decibel| & \si{\decibel}      \\
%             Узел                      & \verb|\knot| & \si{\knot}         \\
%             Миллиметр ртутного столба & \verb|\mmHg| & \si{\mmHg}         \\
%             Морская миля              & \verb|\nauticalmile| & \si{\nauticalmile} \\
%             Непер                     & \verb|\neper| & \si{\neper}        \\
%             \bottomrule
%         \end{tabular}
%     \end{threeparttable}
% \end{table}

% \begin{table}
%     \small
%     \centering
%     \begin{threeparttable}% выравнивание подписи по границам таблицы
%         \caption{Приставки СИ}\label{tab:unit:prefix}
%         \begin{tabular}{llcc|llcc}
%             \toprule
%             Приставка & Команда                  & Символ      & Степень &
%             Приставка & Команда                  & Символ      & Степень   \\
%             \midrule
%             Иокто     & \verb|\yocto|  & \si{\yocto} & -24     &
%             Дека      & \verb|\deca|  & \si{\deca}  & 1         \\
%             Зепто     & \verb|\zepto|  & \si{\zepto} & -21     &
%             Гекто     & \verb|\hecto|  & \si{\hecto} & 2         \\
%             Атто      & \verb|\atto|  & \si{\atto}  & -18     &
%             Кило      & \verb|\kilo|  & \si{\kilo}  & 3         \\
%             Фемто     & \verb|\femto|  & \si{\femto} & -15     &
%             Мега      & \verb|\mega|  & \si{\mega}  & 6         \\
%             Пико      & \verb|\pico|  & \si{\pico}  & -12     &
%             Гига      & \verb|\giga|  & \si{\giga}  & 9         \\
%             Нано      & \verb|\nano|  & \si{\nano}  & -9      &
%             Терра     & \verb|\tera|  & \si{\tera}  & 12        \\
%             Микро     & \verb|\micro|  & \si{\micro} & -6      &
%             Пета      & \verb|\peta|  & \si{\peta}  & 15        \\
%             Милли     & \verb|\milli|  & \si{\milli} & -3      &
%             Екса      & \verb|\exa|  & \si{\exa}   & 18        \\
%             Санти     & \verb|\centi|  & \si{\centi} & -2      &
%             Зетта     & \verb|\zetta|  & \si{\zetta} & 21        \\
%             Деци      & \verb|\deci| & \si{\deci}  & -1      &
%             Иотта     & \verb|\yotta| & \si{\yotta} & 24        \\
%             \bottomrule
%         \end{tabular}
%     \end{threeparttable}
% \end{table}

% \subsection{Заголовки с формулами: \texorpdfstring{\(a^2 + b^2 = c^2\)}{%
%         a\texttwosuperior\ + b\texttwosuperior\ = c\texttwosuperior},
%     \texorpdfstring{\(\left\vert\textrm{{Im}}\Sigma\left(
%             \protect\varepsilon\right)\right\vert\approx const\)}{|ImΣ (ε)| ≈ const},
%     \texorpdfstring{\(\sigma_{xx}^{(1)}\)}{σ\_\{xx\}\textasciicircum\{(1)\}}
% }\label{subsec:with_math}

% Пакет \texttt{hyperref} берёт текст для закладок в pdf-файле из~аргументов
% команд типа \verb|\section|, которые могут содержать математические формулы,
% а~также изменения цвета текста или шрифта, которые не отображаются в~закладках.
% Чтобы использование формул в заголовках не вызывало в~логе компиляции появление
% предупреждений типа <<\texttt{Token not allowed in~a~PDF string
%     (Unicode):(hyperref) removing...}>>, следует использовать конструкцию
% \verb|\texorpdfstring{}{}|, где в~первых фигурных скобках указывается
% формула, а~во~вторых "--- запись формулы для закладок.

% \section{Рецензирование текста}\label{sec:markup}

% В шаблоне для диссертации и автореферата заданы команды рецензирования.
% Они видны при компиляции шаблона в режиме черновика или при установке
% соответствующей настройки (\verb+showmarkup+) в~файле \verb+common/setup.tex+.

% Команда \verb+\todo+ отмечает текст красным цветом.
% \todo{Например, так.}

% Команда \verb+\note+ позволяет выбрать цвет текста.
% \note{Чёрный, } \note[red]{красный, } \note[green]{зелёный, }
% \note[blue]{синий.} \note[orange]{Обратите внимание на ширину и расстановку
%     формирующихся пробелов, в~результате приведённой записи (зависит также
%     от~применяемого компилятора).}

% Окружение \verb+commentbox+ также позволяет выбрать цвет.

% \begin{commentbox}[red]
%     Красный текст.

%     Несколько параграфов красного текста.
% \end{commentbox}

% \begin{commentbox}[blue]
%     Синяя формула.

%     \begin{equation}
%         \alpha + \beta = \gamma
%     \end{equation}
% \end{commentbox}

% \verb+commentbox+ позволяет закомментировать участок кода в~режиме чистовика.
% Чтобы убрать кусок кода для всех режимов, можно использовать окружение
% \verb+comment+.

% \begin{comment}
% Этот текст всегда скрыт.
% \end{comment}

% \section{Работа со списком сокращений и~условных обозначений}\label{sec:acronyms}

% С помощью пакета \texttt{nomencl} можно создавать удобный сортированный список
% сокращений и условных обозначений во время написания текста. Вызов
% \verb+\nomenclature+ добавляет нужный символ или сокращение с~описанием
% в~список, который затем печатается вызовом \verb+\printnomenclature+
% в~соответствующем разделе.
% Для того, чтобы эти операции прошли, потребуется дополнительный вызов
% \verb+makeindex -s nomencl.ist -o %.nls %.nlo+ в~командной строке, где вместо
% \verb+%+ следует подставить имя главного файла проекта (\verb+dissertation+
% для этого шаблона).
% Затем потребуется один или два дополнительных вызова компилятора проекта.
% \begin{equation}
%     \omega = c k,
% \end{equation}
% где \( \omega \) "--- частота света, \( c \) "--- скорость света, \( k \) "---
% модуль волнового вектора.
% \nomenclature{\(\omega\)}{частота света\nomrefeq}
% \nomenclature{\(c\)}{скорость света\nomrefpage}
% \nomenclature{\(k\)}{модуль волнового вектора\nomrefeqpage}
% Использование
% \begin{verbatim}
% \nomenclature{\(\omega\)}{частота света\nomrefeq}
% \nomenclature{\(c\)}{скорость света\nomrefpage}
% \nomenclature{\(k\)}{модуль волнового вектора\nomrefeqpage}
% \end{verbatim}
% после уравнения добавит в список условных обозначений три записи.
% Ссылки \verb+\nomrefeq+ на последнее уравнение, \verb+\nomrefpage+ "--- на
% страницу, \verb+\nomrefeqpage+ "--- сразу на~последнее уравнение и~на~страницу,
% можно опускать и~не~использовать.

% Группировкой и сортировкой пунктов в списке можно управлять с~помощью указания
% дополнительных аргументов к команде \verb+nomenclature+.
% Например, при вызове
% \begin{verbatim}
% \nomenclature[03]{\( \hbar \)}{постоянная Планка}
% \nomenclature[01]{\( G \)}{гравитационная постоянная}
% \end{verbatim}
% \( G \) будет стоять в списке выше, чем \( \hbar \).
% Для корректных вертикальных отступов между строками в описании лучше
% не~использовать многострочные формулы в~списке обозначений.


% \nomenclature{%
%     \( \begin{rcases}
%         a_n \\
%         b_n
%     \end{rcases} \)%
% }{коэффициенты разложения Ми в дальнем поле соответствующие электрическим и
%     магнитным мультиполям}
% \nomenclature[a\( e \)]{\( {\boldsymbol{\hat{\mathrm e}}} \)}{единичный вектор}
% \nomenclature{\( E_0 \)}{амплитуда падающего поля}
% \nomenclature{\( j \)}{тип функции Бесселя}
% \nomenclature{\( k \)}{волновой вектор падающей волны}
% \nomenclature{%
%     \( \begin{rcases}
%         a_n \\
%         b_n
%     \end{rcases} \)%
% }{и снова коэффициенты разложения Ми в дальнем поле соответствующие
%     электрическим и магнитным мультиполям. Добавлено много текста, так что
%     описание группы условных обозначений значительно превысило высоту этой
%     группы...}
% \nomenclature{\( L \)}{общее число слоёв}
% \nomenclature{\( l \)}{номер слоя внутри стратифицированной сферы}
% \nomenclature{\( \lambda \)}{длина волны электромагнитного излучения в вакууме}
% \nomenclature{\( n \)}{порядок мультиполя}
% \nomenclature{%
%     \( \begin{rcases}
%         {\mathbf{N}}_{e1n}^{(j)} & {\mathbf{N}}_{o1n}^{(j)} \\
%         {\mathbf{M}_{o1n}^{(j)}} & {\mathbf{M}_{e1n}^{(j)}}
%     \end{rcases} \)%
% }{сферические векторные гармоники}
% \nomenclature{\( \mu \)}{магнитная проницаемость в вакууме}
% \nomenclature{\( r, \theta, \phi \)}{полярные координаты}
% \nomenclature{\( \omega \)}{частота падающей волны}

% С помощью \verb+nomenclature+ можно включать в~список сокращения,
% не~используя их~в~тексте.
% % запись сокращения в список происходит командой \nomenclature,
% % а не употреблением самого сокращения
% \nomenclature{FEM}{finite element method, метод конечных элементов}
% \nomenclature{FIT}{finite integration technique, метод конечных интегралов}
% \nomenclature{FMM}{fast multipole method, быстрый метод многополюсника}
% \nomenclature{FVTD}{finite volume time-domain, метод конечных объёмов
%     во~временной области}
% \nomenclature{MLFMA}{multilevel fast multipole algorithm, многоуровневый
%     быстрый алгоритм многополюсника}
% \nomenclature{BEM}{boundary element method, метод граничных элементов}
% \nomenclature{CST MWS}{Computer Simulation Technology Microwave Studio
%     программа для компьютерного моделирования уравнен Максвелла}
% \nomenclature{DDA}{discrete dipole approximation, приближение дискретиных
%     диполей}
% \nomenclature{FDFD}{finite difference frequency domain, метод конечных
%     разностей в~частотной области}
% \nomenclature{FDTD}{finite difference time domain, метод конечных разностей
%     во~временной области}
% \nomenclature{MoM}{method of moments, метод моментов}
% \nomenclature{MSTM}{multiple sphere T-Matrix, метод Т-матриц для множества
%     сфер}
% \nomenclature{PSTD}{pseudospectral time domain method, псевдоспектральный метод
%     во~временной области}
% \nomenclature{TLM}{transmission line matrix method, метод матриц линий передач}

\nomenclature{БШС}{беспроводная широкополосная сеть}
\nomenclature{БС}{базовая станция}
\nomenclature{RSU}{Roadside Unit, придорожные стационарные объекты телекоммуникационной связи}
\nomenclature{СеМО}{Сеть массового обслуживания}
\FloatBarrier
