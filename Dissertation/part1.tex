\chapter{Синтез топологической структуры беспроводных широкополосных сетей}\label{ch:ch1}

Современные беспроводные широкополосные сети (БШС) связи , обладая рядом преимуществ, нашли свое широкое применение в задачах мониторинга и управления различных   производственных или гражданских объектов, движущихся транспортных средств и т.д. К основным преимуществ развертывания БШС можно отнести:

\begin{itemize}
  \item  возможность получения информации с любой точки контролируемой территории,
  \item быстрый ввод в эксплуатацию,
  \item сокращение капитальных затрат на создание и эксплуатацию сети,
  \item высокая гибкость, мобильность и масштабируемость. 
\end{itemize}


Одним и ключевых направлений цифровой трансформации является внедрений БШС на автомобильных дорогах в рамках интеллектуальных транспортных систем (ИТС). Создание современной инфраструктуры передачи мультимедийной информации  вдоль протяженных магистралей является одной из важнейших проблем при создании новых и функционировании существующих транспортных магистралей. Особенно актуально решение этой проблемы для стран с обширной территорией, к числу которых относится Российская Федерация. \cite{Vish2015}. Создание такой инфраструктуры позволит обеспечить качественной и высокоскоростной связью для передачи данных в реальном времени. 





Цифровая трансформация также затронула основной элемент российской экономики -- нефтегазовый комплекс, являясь первостепенной задачей  для развития  нефтегазового сектора страны. Она позволит бизнесу в рамках перехода производства к новой модели управления <<Индустрия 4.0>> сохранить лидирующие позиции и не утратить производственного потенциала  \cite{Eremin2020, Dmitrievskiy2020}. Неотъемлемой частью такой трансформации является внедрение беспроводных технологий \cite{Eremin2020, Dmitrievskiy2020, Eremin2019, Hiriyannaiah2020, Munirathinam2020, CarmenLucas-Estan2018, Lu2019}. Нефтегазовые объекты часто расположены в труднодоступной местности на обширной территории в несколько километров. Данный фактор является ключевым преимуществом беспроводных технологий для развертывания по сравнению с кабельными коммуникациями.

% Беспроводная связь в автоматизации промышленного производства способствует осуществлению производственных процессов более экономически эффективно, гибко и надежно, а также позволяет реализовывать новые концепции автоматизации \cite{Gost62657}. 

% Для управления большими объемами на месторождениях применение применение беспроводных каналов связи в режиме 60/24/7, позволяюшая операторам принимать обоснованные управленческие решения, своевременно диагностировать и устранять неисправности, а также оптимизировать работу АСУ ТП в целом \cite{Dmitrievsky2021}.

% Не маловажную роль на месторождениях играет безопасность. Технологические объекты на нефтяных или газовых месторождениях, оснащенных широкополосным подключением, позволяют соответствовать современным концепциям и требованиям в сфере безопасности персонала и безопасности имущества, включая охрану с использованием беспроводных камер видеонаблюдения. Для предоставления доступа к объекту могут использоваться дополнительные возможности, такие как считывание номерных знаков и распознавание лиц. А благодаря использованию тепловых камер можно контролировать риски отключения и перегрузки даже с помощью периодических снимков оборудования на промысле. 

% Внедрение БШС особенно хорошо подходят для систем видеонаблюдения, поскольку позволяют расположить камеры там, где они нужны, а не там, где удобно для подключения к проводной сети.

В данном исследовании в рамках цифровой трансформации представлены модели и методы оптимизации задачи синтеза топологии БШС, являющейся частью комплексного проектирования беспроводных сетей. 
 (Рисунок \cref{fig:industry4}).

\begin{figure}[h!]
  \centering
   \includegraphics[width=1\textwidth]{industry4.png}
\caption{Задача синтеза топологии при проектировании БШС в рамках цифровой трансформации <<Индустрия 4.0>>.}
\label{fig:industry4}
\end{figure}

В первой главе представлены основные этапы проектирования БШС и место задачи синтеза ее топологии в ней. Рассмотрены основные протоколы передачи данных БШС и анализ их структуры для решения задачи оптимального размещения базовых станций (БС) этой сети. В диссертации качестве входных параметров задач размещения БС принимаются величины дальности телекоммуникационной связи станций. Для расчета данных параметров в главе представлены основные модели распространения радиосигнала на открытой местности. В диссертации предлагается новый алгоритм размещения БС, учитывающий ограничение на время передачи пакетов в сети. Для ее оценки в главе исследованы модели многофазных сетей массового обслуживания.

% В диссертационной работе представлены формулировки задачи размещения БС в виде целочисленного линейного программирования и в комбинаторной форме. Предложен новый алгоритм размещения БС для комбинаторной задачи, позволяющий учитывать специфику задачи, а именно размещения БС вдоль протяженных объектов, и использовать в качестве ограничения задачи оценку характеристику производительности БШС. 

% В качестве такой оценки рассмотрена величина межконцевой задержки в БШС. Для рассмотренных протоколов передачи данных проанализированы структуры кадров и организация передачи пакетов в канале. Проанализирована аппарат теории массового обслуживания для оценки межконцевой задержки сети. 


% === === === ===
% В первой главе представлены основные этапы проектирования БШС и место задачи синтеза ее топологии в ней. Рассмотрены основные протоколы передачи данных БШС и анализ их структуры для решения задачи оптимального размещения узлов (БС) этой сети. 
% В качестве входных параметров задач размещения БС принимаются величины дальности телекоммуникационной связи станций. Для расчета данных параметров в главе представлены основные модели распространения радиосигнала на открытой местности. 
% В диссертационной работе представлены формулировки задачи размещения БС в виде целочисленного линейного программирования и в комбинаторной форме. Предложен новый алгоритм размещения БС для комбинаторной задачи, позволяющий учитывать специфику задачи, а именно размещения БС вдоль протяженных объектов, и использовать в качестве ограничения задачи оценку характеристику производительности БШС. 
% В качестве такой оценки рассмотрена величина межконцевой задержки в БШС. Для рассмотренных протоколов передачи данных проанализированы структуры кадров и организация передачи пакетов в канале. Проанализирована аппарат теории массового обслуживания для оценки межконцевой задержки сети. 

\section{Этапы проектирования БШС}

Для обеспечения высокого качества беспроводной связи необходимо проводить грамотное проектирование БШС. Существуют различные подходы к проектированию беспроводных сетей. Для одних задачей является максимальная зона покрытия, для других -- достижения максимальной производительности передачи данных, для третьих -- нахождения баланса между зоной охвата и производительностью \cite{Proletarsky}. В диссертации будут предложены модели и методы оптимального размещения БС БШС, целью которых является максимальная зона охвата.  Процесс проектирования современной БШС, как правило, для такого подхода имеет следующие основные этапы (Рисунок \cref{fig:part1_design_stages}):

\begin{figure}[h!]
  \centering
   \includegraphics[width=0.8\textwidth]{design_stages.png}
\caption{Этапы проектирования БШС.}
\label{fig:part1_design_stages}
\end{figure}

Любое проектирование БШС всегда начинается с первоначального обследования местности. В данный этап входят задачи радиообследования и радиопланирования, оценки реальных размеров области контроля, наличие стационарных инженерно-технических сооружений, мешающих передачи сигнала, такими как металлические конструкции, перекрытия, стены и т.д. При развертывании БШС в открытой местности также немаловажную роль играет наличие перепада высот. В ходе выполнения комплекса работ на местности, определяются возможные точки размещения оборудования \cite{Dunaitsev2017}. На основе результатов данного этапа проводится выбор типов моделей оборудования для дальнейшего их размещения и организации сети.

Производительность и дальность действия беспроводных сетей небезграничны. При их проектировании стоит учитывать множество параметров: частота, скорость, мощность излучения \cite{Proletarsky}. На этапе выбора оборудования необходимо определиться с протоколом будущей БШС и  подготовить необходимый комплекс технических средств для развертывания будущей сети. БС является основополагающем устройством будущей сети, которая отвечает за телекоммуникационное покрытие заданной области. Покрытие в свою очередь зависит от мощности передатчика устройства, усиления антенн, чувствительности приемного устройства.


После определения множества возможных точек размещения БС на этапе обследования местности и выборе возможных типов и моделей оборудований можно переходить непосредственно к размещению БС и определению топологической структуры сети. Этап выбора топологической структуры будущей сети является ключевой проблемой диссертации. В рамках данной проблемы будут предложены модели и методы оптимального размещения БС для организации БШС.

После решения задачи синтеза топологии, для полученного размещения решаются задачи оценки характеристик производительности БШС. Для расчета оценок широко применяется аппарат теории массового обслуживания (ТМО). Примерами таких задач являются расчет надежности всех элементов сети \cite{Wankpo2020, Krishnamoorthy2021, Kozyrev2019}, оценка характеристик качества канала, вероятности потери пакетов, пропускной способности,  среднего времени доставки сообщений в сети \cite{Larionov2019, Vishnevsky2016_Methods_of_performance, Vishnevsky2016_Review_of_methodology, Wang2017, Sandmann2012, Baumann2017}. В работе \cite{Eremenko2013} рассматривают стохастическую модель марковской цепи для оценки качества передачи данных автоматизированных систем управления технологическим процессом  в условиях помех и прерываний. Одним из современных направлений в исследовании характеристик производительности БШС является использование ТМО в совокупности с методами машинного обучения (МО) \cite{Lovas2021, SatyaHermanto2018}.

Описанная процедура проектирования БШС является общей для большинства внедрения беспроводных коммуникационных сетей. В зависимости от конкретных целей, которые преследуют проектировщики, план работ может требовать содержание конкретных этапов и подзадач проектирования. В общем же случае проектирование БШС будет происходить согласно представленной последовательности этапов. В изложенной концепции важным является представление места результатов исследования диссертации в глобальной задаче комплексного проектирования.


\section{Анализ современных беспроводных широкополосных технологий 
передачи данных}

% Написать про Wi-Fi, LTE, 5G.

% Далее расписать про Vanet в целом 

% \begin{itemize}
%   \item Скорее всего придется удалить инфу про сенсорные сети;
% \end{itemize}

% Анализ современных беспроводных сетей.
% \begin{itemize}
%   \item Wi-Fi
%   \item LTE
%   \item 5G
%   \item Wi-Max
% \end{itemize}

% Сети VANET
% \begin{itemize}
%   \item 802.11p;
%   \item решения на основе сотовых сетей.
% \end{itemize}

Сегодня БШС являются одним из основных направлений развития телекоммуникационной индустрии. Существуют множество различных протоколов БШС, каждая имеет свои преимущества для конкретной задачи. Исходя из специфики таких задач, можно выделить два больших класса. Первый класс -- это сети малого радиуса действия, объединяющие в себя множество сенсоров. Для таких сетей характерны низкоскоростной трафик, малое энергопотребление, высокие требования к надежности передачи. К ним относятся ZigBee, Bluetooth, промышленные сети WirelessHART и ISA 100.11a, и т.д.  Вторым классом сетей являются БШС, для которых  характерны высокая скорость передачи, большая дальность передачи сигнала от несколько десятков метров до несколько километров: Wi-Fi, WiMAX; сотовые сети 3G, LTE и новое поколение сетей 5G NR.

При проектировании БШС необходимо определиться с выбором протокола передачи данных, который будет удовлетворять всем требованиям проектировщиков. Исследования в диссертационной работе главным образом касаются развертыванию высокоскоростной сети передачи данных на открытой местности. В связи с этим в данной работе будет рассмотрен второй класс сетей. 

На сегодняшний день основными стандартами являются семейство протоколов IEEE 802.11, сотовые сети LTE и новая мобильная связь 5G NR. В параграфе будет представлен анализ структуры и возможных топологий развертывания БС данных стандартов. Изложенная информация будет необходимо при постановке задач синтеза топологии будущей БШС.


% Примерами таких сетей являются сети второго уровня, объединяющие кластеры сенсорных сетей для передачи данных в центр управления. Также примером является БШС сбора мультимедийного трафика с беспроводных камер видеонаблюдения. 


% Существуют множество различных протоколов БШС, используемых на месторождении. Любая сеть имеет свои преимущества для конкретной задачи. Исходя из специфики таких задач, можно выделить два больших класса. Первый класс -- это сети, узлами которого являются измерительные устройства (сенсоры) АСУ ТП. Такие сети называют  ячеистые сенсорные сети. Для таких сетей характерны низкоскоростной трафик, малое энергопотребление, высокие требования к надежности передачи, а также надежности, непосредственно,  оборудования. Вторым классом сетей являются БШС, для которых  характерны высокая скорость передачи, большая дальность передачи сигнала от несколько десятков метров до несколько сотен метров. Примерами таких сетей являются сети второго уровня, объединяющие кластеры сенсорных сетей для передачи данных в центр управления. Также примером является БШС сбора мультимедийного трафика с беспроводных камер видеонаблюдения. 

% \subsection{Сети малого радиуса действия c низкоскоростным трафиком}
% \fixme{Нижний уровень АСУ ТП, включающий измерительные приборы технологических параметров, исполнительные механизмы, коммутируемые с устройствами сопряжения, объединяются в беспроводную сенсорную сеть для эффективного мониторинга, контроля и управления технологическим объектом \cite{Florencio2020}.}


% Ячеистая сеть (Mesh) является одной эффективных технологий, позволяющая работать в неблагоприятных климатических условиях \cite{Sidney2015, Tagirov2013}. Ячеистые сети имеют повышенную надежность благодаря тому, что каждое устройство является узлом такой сети, через который помимо сигнал с самого устройства передается сигнал смежных узлов сети.
% Сеть на основе ячеистой топологии надежна, энергоэффективна. Высокая надежность обеспечивается наличием резервных маршрутов передачи данных. При прекращении функционирования одного из устройств (датчиков), передача сигнала будет происходить в обход устройства, по резервному пути. Ключевым недостатком MESH сетей является -- низкая скорость передачи данных. К ключевым преимуществом сенсорных сетей можно также отнести автономность питания, позволяющее установку узлов (устройств) сети на удалении от проводных сетей и общего питания. 


% Благодаря невысокому энергопотреблению устройств ячеистые сенсорные сети могут работать продолжительное время от одного элемента питания. Это осуществляется за счет управления временем опроса данных в диапазоне от одной секунды до одного часа. Такой подход позволяет  экономить заряд батареи устройства и увеличивает продолжительность жизни устройства.

% Широкое распространение в промышленности получили протоколы сенсорных сетей: ISA100.11a, ZigBee, WirelessHART и набирающий свою популярность стандарт LoRaWAN. Производители сенсорных сетей Emerson, Honeywell, Yokagawa, Simens и др. обеспечивают высокую надежность узлов таких сетей, так как работа оборудования осуществляется в условиях пыли, влаги, вибраций и высоких температур.

% \subsubsection{ISA 100.11a}
% Данный протокол является одним из распространенных протоколов передачи сигнала на промысле. Так компания Honeywell Process Solutions приняла решение разрабатывает оборудование, совместимое с ISA 100 Wireless \cite{Sidney2015}.
% Топология, используемая в ISA 100.11a, основана на соединении типа "точка -- точка". Такая сеть представляет собой централизованную систему, где каждое полевое устройство связано непосредственно с главным устройством (шлюзом). 

% Протокол ISA 100.11a определяют следующие методы: расширенный спектр со скачкообразной перестройкой частоты (FHSS), множественный доступ с временным разделением каналов (TDMA) и множественный доступ с контролем несущей (CSMA).

% Как и любая MESH сеть, данная технология энергоэффективной, т.е. с меньшим энергопотреблением для мощности передачи сигнала. Данный фактор вносит существенную проблему, каждое устройство должно быть в непосредственной близости от шлюза и иметь прямую видимость для передачи данных. Для решения этой проблемы, в стандарт была добавлена поддержка Mesh. Потеря связи и плохой сигнал сказываются не только на целостности и своевременности получаемой информации, но и на жизненном цикле батареи питания \cite{Tagirov2013}.

% \subsubsection{ZigBee}
% Технология Zigbee основана на протоколе IEEE 802.15.4 LR PAN (Low Rate Personal Area Network) для низкоскоростных сетей с малым энергопотреблением. Этот стандарт описывает физический и канальный уровни со скоростью обмена данными до 250 Кбит/с и вариантами топологии "звезда","точка-точка" и Мesh. Частотный диапазон сети для России 2400 - 2483 МГц (16 каналов). Дальность связь -- 200 м. Стандарт поддерживает до 255 одновременно подключенных устройств. Ключевым преимуществом ZigBee является низкое энергопотребление. Исторически технологии ZigBee развивалась не в направлении повышения скорости передачи данных, а по пути улучшения алгоритмов сетевого взаимодействия для обеспечения длительного время работы без замены электропитания.  В сетях ZigBee различают два типа устройств -- полнофункциональные устройства FFD (Full-Function Device) и устройства с сокращенным набором функций  RFD (Reduced-function Device). Устройство FFD выступает в роли координатора, то есть организующего сеть и ретранслирующего сообщения. Устройства RFD не могут обмениваться данными друг с другом, а только через координатор FFD.

% Сеть ZigBee строится на базе трех основных устройств: координатор, оконечное
% устройство и маршрутизатор. Координатор выполняет основную роль в сети: формирует,
% запускает сеть, участвует в построении сети, управляет ею, выполняет функции
% маршрутизатора и доверительного центра (trust-центра) – устанавливает настройки
% безопасности, задает настройки в процессе присоединения устройств к сети, управляет
% ключами безопасности [3]. 

% \subsubsection{WirelessHART}


% Протокол WirelessHART разработан для мониторинга на базе проводного протокола HART с выходным сигналом 4 -- 20 мА. Верхний стек WirelessHART соответствует верхним уровням HART и Modbus. Первой компанией, которая предоставила оборудование, поддерживающее WirelessHART была Emerson. Технология также, как и ZigBee, соответствует стандарту IEEE 802.15.4, работающая в частотах не лицензируемого диапазона ISM (промышленность, наука и медицина) —- 2400–2483,5 МГц.

% Координация коммуникации в сети c одноуровневым кодированием осуществляется посредством метода множественного доступа с временным разделением каналов (Time Division Multiple Access – TDMA), который синхронизирует радиостанции с периодом 10 мс \cite{Chupaev2018}. Шлюз системы WirelessHART поддерживает до 250 полевых устройств. 

% В WiressHART используется широкополосная модуляция методом прямой последовательности DSSS. Для предотвращения интерференции с сетями, работающими на этой же частоте, предусмотрена технология скачкообразной смены несущей частоты (FHSS) \cite{Tagirov2013}

% \subsubsection{LoRaWAN}
% LoRaWAN -- современный стандарт интернета вещей (IoT) с большим потенциалом. Применяется для передачи низкоскоростного трафика. Данная технология только начала осваивать промышленный сектор. К преимуществам LoRaWAN относят очень высокую чувствительность приемника до -148 дБм, большая дальность связи 10-15 км и низкое энергопотребление у конечных устройств. По сравнению с другими энергоэффективными сетями LoRaWan имеет колоссальную дальность коммуникационной связи. Демодулятор LoRa может работать при входном сигнале, ниже уровня собственных шумов, вплоть до -20 дБ \cite{Yang2020}. 

% Низкое энергопотребление осуществляется за счет передачу небольших пакетов данных. Это в свою очередь приводит к относительно низкой пропускной способности. Скорость передачи в зависимости от используемой технологии передачи данных на физическом уровне варьируется от нескольких сотен бит/с до нескольких десятков Кбит/с.


% \subsection{Сети дальнего радиуса действия с высокоскоростным трафиком}

% Ячеистые сенсорные сети давно и прочно вошли как неотъемлемая часть АСУ ТП на производстве. Тем не менее они имеют один существенный недостаток -- малый радиус действия связи. Для решения такой проблемы, кластеры ячеистых сетей объединяет между собой в иерархическую структуру, в которой шлюзы объединяются в сети, на базе протоколов дальнего радиуса действия: Wi-Fi, WiMAX \cite{Savazzi2013}. Такие технологии организуют второй уровень, после уровня сбора и обработки информации с полевых устройств, для передачи данных в центр управления. Еще одним современным направлением является внедрений сетей телекоммуникаций для обеспечения связи для мобильных устройств обходчиков и сбора данный с беспроводных камер видеонаблюдения \cite{Sidney2015}.







% \subsubsection{WiMAX}
% \fixme{удалить}
% Технология WiMAX основана на стандарте IEEE 802.16. Разработанный стандарт беспроводной широкополосного доступа рассчитан на внедрение в городских распределенных региональных беспроводных сетях в диапазонах до 66 ГГц. Стандарт предусматривает 5 режимов работ. Режим магистральной передачи WirelessMAN-SC работает в диапазоне 10-66 ГГц со скоростями до 120 Мбит/с и шириной канала порядка 25 МГц. Применяемые топологии -- "точка-точка" и "точка - много точек" \cite{Vishnevsky2009}. Остальные режимы WirelessMAN-SCa, WirelessMAN-OFDM, WirelessMAN-OFDMA и  WirelessHUMAN работают на частотах 2-11 Ггц. Режим WirelessMAN-OFDM поддерживает архитектуру Mesh. Дальность связи для WiMAX технологии достигает 20 км для прямой видимости и хорошего состояния радиоканала.



% Технология WiMAX успела получить свою популярность в России и занять определенную нишу в организации телекоммуникационной связи. Исторически WiMAX всегда рассматривался как альтернатива сотовым сетям связи. С резким скачком популярности LTE и появлением сетей пятого поколение WiMAX перестал быть конкурентноспособным стандартом на рынке.

\subsection{Архитектура сетей Wi-Fi}

Семейство протоколов IEEE 802.11 или Wi-Fi является одним из популярных протоколов среди беспроводных сетей широкополосного доступа. Стандарт определяет два типа топологий. Первая топология в стандарте называется независимой (Independent Basic Service Set, IBSS) на основе <<точка -- точка>>. Такую сеть также называют Ad-Hoc сетью \cite{SmelyanskyBook}. Передача в сети производится на основе децентрализованного алгоритма доступа. Хоть и не существует главной базовой станции, в сети Ad-Hoc в каждый момент времени любой узел может взять ведущую роль базовой станции. Следующий тип -- централизованная топология. Такой тип сети называют Infrastracture  Basic Service Set или просто BSS \cite{Olifer2020}. Узлы сети могут обмениваться только с базовой станцией, обеспечивающее взаимодействие между всеми конечным узлами сети. 
 
Стек протоколов стандарта IEEE 802.11 соответствует общей структуре 802 (Рисунок \cref{fig:part1_wifi_osi_stack}). Физический уровень практически соответствует физическому уровню в модели OSI. Канальный уровень во всех протоколах 802.х разбит на два или более подуровня.  Подуровень управления доступа к среде обеспечивает адресацию и механизмы управления доступом к каналам. Подуровень MAC выступает в качестве интерфейса между физическим уровнем и подуровнем управления логическим соединением (Logical Link Control, LLC). Задача LLC состоит в том, чтобы сделать различия стандартов 802.11x невидимыми для сетевого уровня \cite{Tanenbaum2022}. На уровне MAC протокола 802.11 определяются два типа доступа к среде передачи данных: функция распределенной координации (Distributed Coordination Function, DCF) и функция централизованной координации (Point Coordination function, PCF), работающий только в режиме IBSS \cite{802.11WG-WirelessLANWorkingGroup2020}.


\begin{figure}[h!]
  \centering
   \includegraphics[width=0.8\textwidth]{wifi_osi_stack.png}
\caption{Часть стека протоколов 802.11}
\label{fig:part1_wifi_osi_stack}
\end{figure}

\paragraph{Развитие стандарта.}
Первый стандарт в 1997 году поддерживал скорость передачи до 1,2 Мбит/c с рабочем диапазоном частот 2,4 ГГц. Уже в 1999 году стандарт 802.1b скорость передачи достигала 11 Мбит/с. В тот же 1999 год одновременно вышел 802.11a, поддерживающий способ модуляции OFDM -- мультиплексирование с ортогональным частотным разделением каналов. Данный метод модуляции пришел на смену методов прямой последовательности для расширения спектра (Direct Sequence Spread Spectrum, DSSS) и псевдослучайная перестройка рабочей частоты (Frequency-Hopping Spread Spectrum, FHSS). Протокол работает в диапазоне частот 5ГГц. Данное введение позволило увеличить скорость до 54 Мбит/с. В 2003 году появился 802.11g, в котором OFDM стал совместим с 802.11b DSSS.  В 2009 году, в рамках стандарта 802.11n была завершена работа над методами передачи данных SU-MIMO (Single User -- Multiple Input Multiple Output), которые одновременно используют несколько антенн на приемнике и передатчике, что дает очередной выигрыш в скорости. Благодаря четырем антеннам и более широким каналам стандарт 802.11 теперь определяет скорости до 600 Мбит/с \cite{Tanenbaum2022, Coleman2018}. В 2013 году появился стандарт 802.11ac с максимальной теоретической скоростью беспроводного соединения 3,47 Гбит/с. В стандарте представлена технология одновременная передача по нескольким антеннам сразу четырем клиентам MU-MIMO (Multi User –- Multiple Input Multiple Output). В 2021 была представлена сеть шестого поколения Wi-Fi 6, стандарт получил название IEEE 802.11ax. Скорость передачи достигает 9608 Мбит/с. Стандарт работает на частотах 2,4 ГГц и 5 ГГц. Реализована новый тип модуляции -- 1024-QAM. Теперь MU-MIMO работает для одновременной передачи до восьми клиентам.  Кроме того, Wi-Fi 6 обеспечивает эффективную работу в сетях IoT, существенно повышая эффективность работы с малыми объемами данных и эффективно используя батарею маломощных измерительных устройств за счет сокращения обмена служебной информацией.

\paragraph{Беспроводной широкополосный доступ вдоль протяженных автомобильных дорог.}
Тенденция цифровизации затронула транспортную отрасль. Одним из путей ее развития является создание современной инфраструктуры передачи мультимедийной информации вдоль протяженных автомагистралей \cite{Vish2015}. Для реализации такой сети разработан стандарт IEEE 802.11p/WAVE. Протокол разработан для обмена данными между высокоскоростными транспортными средствами, а также между транспортными средствами и придорожной инфраструктурой. Такие сети называются V2X (Vehicle-to-everything) сетями. Стандарт 802.11p работает в диапазоне 5,85–5,925 ГГц для движущихся объектов на скорости до 200 км/ч мимо неподвижных базовых станций на расстоянии до 1 км. Протокол IEEE 802.11p входит в состав стандарта Wireless Access in Vehicular Environ (WAVE) и является своего рода интерфейсом для связи с IEEE 1609. В совокупности два этих стандарта определяют интерфейс и набор дополнительных функций для обеспечения связи между движущимися объектами \cite{IEEE802.11p}.

\paragraph{Ячеистая топология стандарта 802.11.} Стандарт поддерживает Mesh топологию. Все нововведения относятся к МАС уровню и рассматриваются вопросы маршрутизации пакетов в рамках Mesh-сети. Фактически в стандарте рассматриваются сетевой и транспортный уровень модели OSI. Пакет уровня MAC содержит дополнительный Mesh - заголовок в отличие от стандартных заголовка пакетов 802.11. В заголовке указывается время жизни пакеты при пересылке через несколько узлов сети \cite{Vishnevsky2009}. Такой подход помогает решить проблему наличия циклов при выборе пути передачи. Протоколы маршрутизации делят на два класса: проактивный -- каждый узел строит свою таблицу маршрутизации и делится информацией об изменении топологии сети со своими соседями и реактивный --  таблицы маршрутизации не строятся, маршрут составляется по мере необходимости. Стандарт 802.11s определяет гибридный протокол маршрутизации по умолчанию -- HWMP (Hybrid Wireless Mesh Protocol). Также стандарт позволяет разработчикам оборудования использовать альтернативные методы маршрутизации. Режим энергосбережения в 802.11s является опциональным \cite{802.11WG-WirelessLANWorkingGroup2020}. Точки доступа Mesh-сети (Mesh Access Point, MAP) всегда активны, так как в любой момент времени могут обратиться устройства, не поддерживающие непосредственно 802.11s. Необходимость возникает в случае IoT, когда всегда устройства имеют автономное питание (измерительные устройства, датчики) \cite{Vishnevsky2009}.


При внедрении беспроводных технологий необходимо учитывать специфику выполняемых задач будущей сети. Для каждого конкретной цели существуют свои требования к скорости передачи данных, дальности связи, потребляемой мощности, помехозащищённости, надежности и т.д. Чтобы учесть специфику задач разработано множество беспроводных решений, охватывающее дальность связи от несколько сантиметров до десятков километров и скоростей передачи от единиц Кбит/с до сотен Мбит/с.

\subsection{Архитектура сетей LTE}
% В рамках задачи синтеза топологии, в диссертационной работе интересно рассмотреть уровень радиодоступа E-UTRAN. 

Перейдем теперь к широкополосным мобильным сетям. На сегодня основной технологией является технология LTE (Long-Term Evolution). Сеть LTE состоит из двух компонентов:
\begin{itemize}
    \item сети радиодоступа E-UTRAN (Evolved Universal Terrestrial Radio Access Network);
    \item архитектура ядра сети SAE (System Architecture Evolution)
\end{itemize}
Сеть радиодоступа E-UTRAN состоит из базовых станций сети eNodeB (или eNB, evolved Node B).
Основным компонентом архитектуры SAE является Evolved Packet Core (EPC), в состав которого входят следующие логические элементы \cite{Tikhvinsky2014}:

\begin{enumerate}
  \item MME (Mobility Management Entity) -- узел управления мобильностью абонентского терминала, взаимодействует с базовыми станциями с помощью протоколов плоскости управления C-plane через интерфейс S1-C;
  \item UPE (User Plane Entity) -- выполняет сжатие заголовков IP-протоколов, шифрование потоков данных, терминацию пакетов данных плоскости пользователя, коммутацию пакетов данных при обеспечении мобильности пользователя. Передача данных пользователей происходит согласно протоколам плоскости пользователя U-plane, взаимодействие с eNode происходит с помощью интерфейса S1-U.
\end{enumerate}

Архитектура сети LTE представлена на рисунке \cref{fig:part1_lte_architecture}. Интерфейс S1 необходим для соединения базовой станции eNodeB с узлами MME/UPE.  Интерфейс X2 позволяет соединить узлы eNodeB по принципу <<каждый с каждым>> для переключения устройств между базовыми станциями. 

% Обмен с узлом управления мобильностью (ММЕ — mobility management entity) происходит по интерфейсу S1-MME, а c обслуживающим шлюзом (SGW — Serving Gateway) — по интерфейсу S1-U. Интерфейс S1 поддерживает связи типа множество-множество между MME, SGW и eNB.

\begin{figure}[h!]
  \centering
   \includegraphics[width=.8\textwidth]{lte_architecture.png}
\caption{Архитектура сети LTE.}
\label{fig:part1_lte_architecture}
\end{figure}

Базовые станции eNB выполняют функции управления радиоресурсами (Radio Resource Managment, RRM): управление радиоканалами (Radio Bearer Control), управление доступом (Radio Admission Control), управление мобильностью (Connection Mobility Control), динамическое распределение ресурсов (Dynamic Resource Allocation).

LTE в России работает на частотах в диапазоне 800, 1800 и 2600 МГц. В нижних диапазонах, где всё отлично с покрытием, проблема найти полосу достаточной для полноценного LTE ширины. В верхних обычно хорошо с частотным ресурсом, но площадь покрытия в 5-6 раз меньше при прочих равных. 

В 2010-е годы переход к 4G LTE-A (LTE-Advanced) предоставил возможность передачи на скоростях в нисходящем канале до 100 Мбит/с для мобильных и 1 Гбит/с для малоподвижных абонентов. Такая скорость дала возможность передачи высокоскоростного мультимедийного трафика. В сетях LTE-Advanced из архитектуры сети LTE исчезло понятие контроллера радиосети (Radio Network Controller, RNC), который выполнял функцию по управлению коммуникационными ресурсами. Управляющие функции перешли к базовым станциям, которые теперь, помимо обслуживания радиочасти, стали принимать решение о маршрутизации всего поступившего к ним трафика от абонентов. Данное введение позволяет организовывать соединения между близкими абонентами напрямую, минуя ядро сети, что существенно разгружает межузловую нагрузку на опорной сети. 


\subsection{Архитектура сетей  5G NR}
Сеть 5G -- новое поколение мобильной связи, действующее на основе стандартов телекоммуникаций 5G/IMT-2020. Обозначение <<5G>>, обычно отображающее технологический этап мобильных сетей, в данном моменте на международном уровне де-факто отражает новую эру сетей связи и сервисов в целом \cite{Koucheryavy}. Принято выделять три основные области применения для сетей 5G:

\begin{enumerate}
  \item eMBB (enhanced Mobile BroadBand) -- предоставление усовершенствованного широкополосного мобильного доступа для сценариев, где приоритетом являются пользовательская скорость передачи данных, трафик на единицу площади, мобильность, энергоэффективность и эффективность использования спектра;
  \item mMTC (massive Machine-Type Communication) -- развертывание крупномасштабной системы межмашинной связи с высокой плотностью соединений, характеризуется большим количеством подключенных устройств, передающих низкоскоростной трафик, который нечувствителен к задержкам;
  \item URLLC (Ultra-Reliable and Low-Latency Communication) -- предоставление высоконадежного соединения с очень низкой задержкой передачи данных.
\end{enumerate}


Международный союз электросвязи МСЭ-R определил показатели главных
эксплуатационных характеристик сети, достижение которых позволяет отнести ее к сети 5-го поколения. Требования к ключевым показателям эффективности сетей 5G/IMT-2020 определены в 3GPP TR 38.913 \cite{3GPP_TR_38_913}:
\begin{itemize}
  \item пиковая скорость передачи данных  в канале вниз (DL) 20 Гбит/с и в канале вверх (UL) 10 Гбит/с;
  \item пиковая спектральная эффективность для DL – 30 бит/с/Гц и для UL – 15 бит/с/Гц;
  \item сквозная задержка на уровне управления (control plane latency) 10 мс, на уровне пользователя (user plane latency) для услуг типа eMBB 4 мс в UL/DL и  для услуг типа URLLC 0,5 мс в UL/DL;  
  \item вероятность потери пакета $10^{-5}$ для пакета размеров 32 байт со сквозной задержкой в сети в 1 мс для услуг URLLC;
  \item  максимально возможное ослабление радиосигнала между передатчиком и приемником (Maximum Coupling Loss, MaxCL), при котором данные могут быть успешно приняты 164 дБ при скорости передачи данных равной 160 бит/с; 
  \item плотность подключенных устройств: 1 млн на 1 $\text{км}^2$;
  \item срок работы батареи устройства: 10 – 15 лет для устройств IoT/М2М/D2D; 
  \item скорость передвижения абонентского устройства – до 500 км/ч и беспрерывное обслуживание при хэндовере.
\end{itemize}

По спецификациям 3GPP диапазоны 5G/IMT-2020 разделены на два поддиапазона (Frequency Range) FR1 (450–6000 МГц) и FR2 (24,25 – 52,6 ГГц). Также документами ассоциации GSMA частоты делятся на 3 группы: ниже 1 ГГц (частотный диапазон FR1), в полосе 1 – 6 ГГц (частотный диапазон FR1) и свыше 6 ГГц (частотный диапазон FR2) \cite{prikaz_minkomsvyazi_5G_IMT-2020}.

В отличие от предыдущих поколений, где для каждого нового поколения сетей связи (2G, 3G, 4G) разрабатывался новый радиоинтерфейс, для технологии 5G/IMT-2020 планируется применять как новый радиоинтерефейс (New Radio или NR согласно спецификациям 3GPP), так и эволюцию стандарта LTE-A. Стандарт LTE-A продолжает эволюционировать, приближаясь по своим характеристикам к возможностям стандарта NR. В стандарте LTE-A уже реализована поддержка активных антенных систем, использование более эффективного кодирования и модуляции, агрегация нескольких частотных каналов, уменьшенная задержка на уровне радиоинтерфейса. Новый радиоинтерфейс NR изначально разрабатывался для обеспечения более высоких скоростей передачи данных и меньших задержек, более эффективного использования частотного ресурса \cite{prikaz_minkomsvyazi_5G_IMT-2020}.

Чтобы обеспечить интеграцию с существующими устаревшими системами 4G и обеспечить независимое развертывание 5G RAN и 5G Core, 3GPP определил набор параметров архитектуры.

Для обеспечения интеграцию с существующими сетями 4G/LTE 3GPP определил несколько вариантов развёртывания (Deployment Options) сетей 5G NR.  Существует варианты совместного развёртывания с существующими сетями LTE-A -- класс Non-Standalone (NSA). Внедрение новых полноценных сетей 5G NR относятся к классу Standalone (SA). Большинство операторов не смогут единовременно перевести все имеющиеся сети LTE на 5G  из-за огромных затрат и связанных с этим проблем обратной/прямой совместимости между системами \cite{Teral2019,Kibria2018,Liu2020, ElRhayour2019}.  Варианты архитектуры определяются в соответствии с различными сценариями развертывания сетей. 

\begin{figure}[h!]
  \centering
   \includegraphics[width=.8\textwidth]{5g_deployment_options.png}
\caption{Сценарии развертывания 5G NR}
\label{fig:part1_5g_deployment_options}
\end{figure}

На рисунке \cref{fig:part1_5g_deployment_options} представлены варианты развёртывания сетей 5G. Все варианты размещения содержат различные типы базовых станций: eNB -- базовая станция существующей сети LTE, gNB -- базовая станция новой сети 5G NR,  en-gNB и ng-eNB -- базовые станции, которые обеспечивает стандарт сети LTE для оконечных пользователей и подключены к ядру 5GC новой сети 5G NR. Здесь NG-C, NG-U -- протоколы интерфейса между gNB/eLTE eNB и функциями  плоскости управления (Control Plain, CP) и плоскости пользователя (User Plane, UP), соответственно. Интерфейсы S1-C (плоскость управления) и S1-U (плоскость пользователя) -- интерфейсы сетей LTE для связи базовой станции с EPC. Xx -- межсетевой интерфейс между узлом LTE eNB и узлом gNB, и интерфейс Xn -- межсетевой интерфейс между двумя узлами gNB.

Варианты развертывания 5G :
\begin{itemize}
  \item Вариант 2. Сценарий развертывания новой сети SA NR, наиболее привлекателен в областях, где нет устаревшей системы LTE и требуется развертывание полноценной системы доступа 5G NR. В SA NR базовые станции gNB (Next Generation NodeB) подключаются к ядру 5GC (5G core);
  \item Вариант 3/3A/3x. Сценарий развертывания NSA NR в существующий EPC сети LTE. Предполагается, что данный сценарий будет самым распространенным на практике \cite{EthemAlpaydn2019}. Передача пользовательских данных сети NR от базовой станции en-gNB будет происходить непосредственно к EPC или через базовую станцию LTE eNB;
  \item Вариант 4/4A. Сценарий развертывания сети NSA NR, в которой имеющиеся базовые станции LTE eNB усовершенствованы до ng-eNB и вместо ядра EPC используется 5GC. Якорной точкой для терминации NG-C являются базовые станции сети радиодоступа NR (gNb). Данная архитектура является финальной версией перехода к сценарию комбинированного использования сети 5G/LTE;
  \item Вариант 5. Сценарий развертывания особенно подходит для областей, где нет устаревшей системы LTE и развернуты развитые системы доступа E-UTRA. При данном сценарии развертывания базовые станции ng-eNB подключаются к 5GC;
  \item Вариант 7/7A. Сценарий может использоваться как промежуточный этап, в котором имеющиеся базовые станции eNB и ядро EPC сети LTE возможно обновить до ng-eNB и 5GC. В качестве основных базовых станций выступают ng-eNB, которые подключаются к 5GC.  LTE построена более полно и, следовательно, более надежна для обработки сигналов. Базовые станции gNB подключаются через ng-eNB или непосредственно напрямую к 5GC.
\end{itemize}



% Этот вариант особенно привлекателен для областей, где унаследованные eNB LTE и EPC подготовлены/квалифицированы для обновления до ng-eNB и 5GC. Это NSA с точки зрения gNB, которому требуется ng-eNB в качестве якоря для подключения CP к 5GC. Здесь ng-eNB подключается к 5GC, а соединение NR UP с 5GC проходит через ng-eNB (вариант 7) или напрямую (вариант 7A).

% Option 4/4a – представляет собой целевую финальную архитектуру комбинированной сети 5G/LTE. Используется технология двойного подключения. Требует внедрение ядра NGCN и модернизации базовых станций сети LTE до ng-eNb. Базируется на технологии двойного подключения. В качестве интерфейса, связывающего сети радиодоступа E-UTRA/NR и NGCN, и переносящего пользовательский (User Plane) и сигнальный (Control Plane) трафик используется NG. Якорной точкой для терминации NG-C являются базовые станции сети радиодоступа NR (gNb).

% Option 7/7a – может использоваться как промежуточный этап на пути к целевой архитектуре 5G/LTE Option 4/4a, в отличии от которой якорной точкой для терминации NG-C являются базовые станции сети радиодоступа E-UTRAN (eNb).

% \section{Выбор протокола беспроводной широкополосной сети для решения задачи синтеза топологий}
% FIXME: change subsection

% \fixme{Задача размещения узлов БШС для ячеистых сенсорных сетей не актуально, так как ее узлами являются измерительные приборы. Выбор мест их размещения осуществляется на основе конкретного технологического процесса, который осуществляется на объекте. В силу малых дальностей связей, также неактуально говорить о выборе мест размещений точек доступа (шлюзов) сенсорных сетей. На практике, точки доступа размещаются по месту в блок-боксе местной автоматики. Целесообразно решать задачу для сетей дальнего радиуса связи. В силу глобальной цифровизации месторождения, растет спрос на внедрения новых сетей, осуществляющих надежную и эффективную передачу высокоскоростного трафика. К их числу относятся не только технологические данные с объектов, но и данные с мобильных переносных устройств, камер видеонаблюдения и т.д. На сегодняшний день хоть WiMAX широко распространен в нефтегазовой промышленности, к сожалению сам стандарт утратил свои лидерские позиции. С ростом популярности сотовых сетей LTE и 5G, WiMAX потерял все свои преимущества. Еще одна причина падения популярности -- развитие Wi-Fi. Ведущие вендеры не стремятся развивать технологию, а значит и сам стандарт не будет эволюционировать в ближайшее время. }

% \fixme{Для решения синтеза топологии решено использовать семейство протоколов IEEE 802.11. Исходя из вышеизложенного анализа, стандарт может поддерживать работу сетей внутри помещений (Indoor), на открытой местности (Outdoor). Позволяет организовать связь между движущимися объектами, и объединять узлы в самоорганизующие Mesh сети. В совокупности все эти аспекты позволяет сделать вывод о целесообразности выбора Wi-Fi для организации телекоммуникационной связи на месторождении. }

% \fixme{Наиболее массовыми технологиями в области промышленной автомати¬зации являются Wi-Fi, Bluetooth и ZigBee [11-16], ключевые параметры которых представлены в табл. 1.1. Оборудование данных технологий рабо¬тает в нелицензируемом радиочастотном диапазоне 2,4 ГГц и не требует регистрации в государственных органах при соблюдении условий применения [17].}



% \fixme{
% IWN унаследовали многие функции WSN, особенно протоколы связи и промышленные приложения [91]. Однако из-за особенностей промышленной среды и того, что промышленные беспроводные сети Индустрии 4.0 отличаются от традиционных WSN [17, 37], существуют некоторые новые ограничения и требования к IWN. Основные различия между IWN и WSN перечислены ниже:}

% \fixme{
% IWN с задержкой используются в отраслевых системах для определения важных параметров, таких как мониторинг состояния машины и рабочей среды или предоставление команд управления и информации в реальном времени. Поэтому этим приложениям требуется низкая задержка. В большинстве случаев это происходит за счет энергопотребления и высоких затрат на достижение производительности в реальном времени. Напротив, WSN ограничены энергией, и узлы могут быть развернуты в недоступных доменах, что затрудняет замену или обновление батарей. Как следствие, необходимо максимально увеличить время автономной работы в WSN, что приведет к увеличению задержки. }

% \fixme{
% Мобильность Как обсуждалось выше, для повышения гибкости и мобильности для Индустрии 4.0 IWN могут содержать больше движущихся узлов, таких как мобильные продукты, мобильные роботы, автомобили с автоматическим наведением, беспилотные летательные аппараты, рабочие и другие мобильные устройства. Это контрастирует с узлами WSN, которые обычно считаются стационарными или с несколькими движущимися узлами, такими как движущийся приемник, ретрансляционные узлы и т. Д.
% }

% \fixme{Среды Еще одно существенное различие между IWN и WSN - это операционная среда. Во-первых, в промышленной сфере IWN работают в сложных условиях из-за пыли, вибрации, тепла, различных препятствий, а также более высокой температуры и влажности. Во-вторых, существуют более серьезные помехи сигналам от двигателей и других беспроводных сетей, чем для традиционных WSN. Кроме того, промышленная среда может легко повлиять на радиоканал, который отличается от WSN. Следовательно, IWN нуждаются в дополнительных стратегиях, чтобы гарантировать надежность и эффективную связь. Напротив, узлы WSN развернуты в относительно стабильной и дружественной среде.}


% \fixme{Производительность Для Индустрии 4.0 все оборудование, устройства, рабочие, терминалы и другие узлы могут выполнять сложные задачи индивидуально и взаимодействовать с другим оборудованием. Специально для IWN узлы не только обмениваются данными со своими соседями для выполнения механических задач, таких как перемещение, закрепление и транспортировка, но также должны справляться с такими проблемами, как помехи сигнала, пути перемещения и обработка данных. Как следствие, с точки зрения пропускной способности узлов, узлам IWN требуются более высокие мощности для обработки данных, энергии и хранения, и они более умны, чем традиционные узлы WSN.}






\section{Определение параметров БШС, необходимых для решения задач размещения БС} \label{section:part_1_link_distance}

% \fixme{Добавить инфу зачем все это нужно}

% \fixme{удалить это}
% \begin{itemize}
%   \item добавить модель Окамура-Хата
%   \item добавить еще модели, характерные для сотовых сетей.
% \end{itemize}

Этап выбора топологической структуры беспроводной сети состоит из решения задач оптимального размещения БС. В дальнейшем для решения данных задач необходимо будет подготовить параметры БС: радиус связи -- максимальная теоретическая дальность связи базовой станции с соседней станцией, удовлетворяющей требуемому качеству передачи сигнала; и радиус покрытия -- максимальный теоретический радиус зоны покрытия БС для связи с устройствами. Данные параметры рассчитываются исходя из конфигурации оборудования БС. Далее будет представлен метод расчета. Все технические характеристики для расчета берутся из технических паспортов БС.

% \subsection{Расчет дальности связи}


% \fixme{Перед тем как приступить к задаче ЦЛП необходимо рассчитать характеристики станции: радиус связи $R_{jq}$ и радиус покрытия $r_j$.}

% \fixme{При развертывания сети необходимо обеспечить максимальное покрытие данного участка связь между шлюзами через систему размещенных базовых станций беспроводной широкополосной сети}.
В БШС в большинстве случаев используются радиоволны сантиметрового диапазона. Отличительной чертой распространения данных радиоволн  является почти полное отсутствие явления дифракции и прямолинейность распространения. Волны практически не огибают преград при распространении, поэтому существенное влияние на потерю сигнала оказывают рельеф местности, преграды и погодные условия. 

Для расчета дальности действия связи используют модели распространения радиосигналов \cite{ElChall2019, Zhang2021, Caso2015, Kang2020}. Существуют различные модели, которые можно объединить в три основные категории \cite{Oni2017}:
  \begin{itemize}
    \item теоретические модели, обычно основаны на физическом предположении об идеальных условиях;
    \item эмпирические модели. Это наборы уравнений, разработанные на основе различных данных полевых измерений. Одним из основных недостатков таких моделей является то, что они не могут использоваться для различных ситуации без изменений, поскольку они точны только для случая с теми же характеристиками и условиями, в которых проводились измерения;
    \item детерминированные модели. Модели очень сложны, поскольку они требуют детального знания местоположения, размеров и физических параметров всех препятствий в данной области. Такое детальное исследование может приводить к чрезмерным накладным расходам, которые в большинстве случаев могут быть лишними.
  \end{itemize}

\subsection{Энергетический потенциал канала связи}
Для оценки производительности канала связи используется уравнение энергетического потенциала, который учитывает все усиления и потери уровня сигнала при его распространении от передатчика к приемнику через беспроводную  среду передачи, кабели, разъемы, различные препятствия (Рисунок \cref{fig:link_power}) \cite{Proletarsky}.

В определении энергетического потенциала беспроводной линии связи участвуют следующие параметры:
\begin{itemize}
  \item эффективная изотропно-излучаемая мощность передатчика (Equivalent Isotropically Radiated Power, EIRP), являющаяся суммой выходной мощности передатчика и коэффициента усиления антенны за вычетом потерь в антенном кабеле разъемах передающего тракта;
  \item потери при распространении радиосигнала в свободном пространстве;
  \item чувствительность приемника, потери в антенном кабеле и коэффициент усиления антенны приемника.
\end{itemize}
Полное уравнение можно записать следующим образом \cite{Proletarsky}:

% It is essential during deployment to provide maximum coverage of a given area and ensure communication between the placed base stations in the wireless broadband network. 

% Link Budget is a way of estimation of communication link's performance while accounting for the system's power, gains, and losses for both the transmitter and receiver. The complete equation can be written as follows:

\begin{equation}
  \label{eq:part3_link_budget}
  P_{tr} - L_{tr} + G_{tr} - L_{fs} + G_{recv} - L_{recv} = SOM + P_{recv},
\end{equation}
где:

\begin{itemize}

  \item $P_{tr}$ мощность передатчика, дБм;

  \item $L_{tr}$ потери сигнала на антенном кабеле и разъемах передающего тракта, дБ;

  \item $G_{tr}$ усиление антенны передатчика, дБ;

  \item $L_{fs}$ потери в свободном пространстве, дБ;

  \item $G_{recv}$ усиление антенны приемника, дБ;

  \item $L_{recv}$  потери сигнала на антенном кабеле и разъемах приемного тракта, дБ;

  \item $P_{recv}$ чувствительность приемника, дБм;
  
  \item $SOM$ запас на замирание сигнала, дБ.

\end{itemize}
Энергетический потенциал указывает на качество канала передачи радиосигналов.

\begin{figure}[h!]
  \centering
   \includegraphics[width=1\textwidth]{link_power.png}
\caption{Энергетический потенциал линии связи.}
\label{fig:link_power}
\end{figure}

На стороне передатчика выходной мощностью является величина, равная мощности, подводимой к антенне. Данная величина из паспортной документации устройства имеет различные значения в зависимости от каждого поддерживаемого оборудованием стандарта и конкретных скоростей. В реальной условиях значения мощностей, как правило могут незначительно отклоняться от паспортных значений. 

% \fixme{Предельная мощность передатчика определяется государственными органами. Для примера, для БШС семейства протоколов IEEE 802.11 не превышает 100 мВт или, выражая в децибеллах, не боллее 20 дБм \cite{GKRCh18_13}.}


Затухание сигнала могут происходить в кабелях антенны, зависящие от типа кабеля и рабочей частоты. При подключении антенны желательно обходиться минимальной длиной кабеля. Потери сигнала в антенном кабеле принимают $0, 1...2$ дБ/м. В технической документации в потерях кабеля также учтена величина затухания в кабельных разъемах. 

Усиление антенны описывает фокусирование переданного или полученного сигнала. Значения даны относительно полуволнового диполя или теоретического изотропного излучателя \cite{Gost62657}.


К потерям при распространении относятся все виды затухания сигнала, которые имеют место при его распространении от антенны передатчика к антенне приемника. Самая простая оценка потерь в свободном пространстве получается, если предположить, что сигналы передаются во всех направлениях, то есть мощность излучается одинаково во всех направлениях, и в зоне передачи или вокруг нее нет препятствий, которые могли бы повлиять на распространение электромагнитных сигналов \cite{Krouk2010}. 

% Передающий сигнал рассеивается по мере увеличиня расстояния между приемником и передатчиком. Данный тип затухания называется потерями в свободном пространстве (Free Space Path Loss, $FSPL$).

Существуют большое количество моделей распространения радиосигнала. Каждая имеет свои плюсы и минусы. В зависимости от конкретных задач при проектировании возможно использовать наиболее подходящую. Ниже будут представлены модели распространения, широко используемые для сетей БШС на открытой местности.  

\subsection{Модель потерь в свободном пространстве}
При распространении сигнала от передатчика к приемнику часть сигнала рассеивается, по этой причине мощность на приемной стороне будет уменьшаться с увеличением  расстоянии от передающей антенны. Затухание сигнала называют потерями в свободном пространстве (Free Space Path Loss, $FSPL$).

Потери при распространении между двумя неизотропными антеннами в свободном пространстве (в воздухе) можно выразить из уравнения Фрииса \cref{eq:part1_Friis}. Мощность принимаемой антенны рассчитывается:

\begin{equation}
  \label{eq:part1_Friis}
  \frac{P_{recv}}{P_{tr}} = G_{tr}G_{recv}\left(\frac{c}{4\pi R f} \right)^2,
\end{equation}
где
$c$ --  скорость света,
$f$ -- частота, 
$R$ расстояние между приемной и передающей антенной.

Для неизотропных антенн выражение потерь в децибелах для потерь в свободном пространстве принимает следующий вид:

\begin{equation}
  \label{eq:part3_L_fs}
  L_{fs} = 20 \lg{F} + 20\lg{R} - G_{tr} - G_{recv} + K,
  \end{equation}
где $F$ -- центральная частота, на котором работает канал связи, $R$ -- расстояние между приемной и передающей антенной и $K$ -- константа.

Константа $K$ зависит от размерностей частоты и расстояния:

\begin{itemize}
  \item для частоты, выраженной в ГГц, и расстояния, выраженная в км, константа $K$ равна 92,45;
  \item для частоты, выраженной в МГц, и расстояния, выраженная в км, константа $K$ равна 32,4;
  \item для частоты, выраженной в МГц, и расстояния, выраженная в м, константа $K$ равна -27,55.
\end{itemize} 

Потери $L_{fs}$ выражаются из уравнения энергетического потенциала канала связи \cref{eq:part3_link_budget} как:

\begin{equation}
  \label{eq:part3_L_fs_from_link_budget}
  L_{fs} = P_{tr} - L_{tr} + G_{tr} + G_{recv} - L_{recv} - P_{recv} - SOM.
\end{equation}


Запас на замирания сигнала  (System Operating Margin, $SOM$)  учитывает все возможные факторы отрицательно влияющие на дальность связи. К таким факторам относятся:

\begin{itemize}
  \item температурный дрейф чувствительности приемника и выходной мощности передатчика;
  \item влияние погодных условий на передачу сигнала: туман, снег, дождь;
  \item  потери в антенно-фидерном тракте, возникающие из-за рассогласования фидера и антенны.
\end{itemize}
Приемник испытывает совокупное воздействие всех этих физических факторов, которые различаются в зависимости от положения приемника и передатчика в среде распространения. 

Минимальное значение величины запаса на замирания $SOM$ должна быть не меньше  10 дБ. Считается, что 10-ти децибельный запас по усилению достаточен для инженерного расчета, но на практике зачастую используют значение $20 ... 30$ дБ \cite{Proletarsky}.

% \fixme{Энергетический потенциал указывает на качество канала передачи радиосигналов}.



% The Free Space Path Loss ($ FSPL $) equation defines the propagation signal loss between two antennas through free space (air):



Максимально возможную дальность связи между приемником и передатчиком выводится из уравнений \cref{eq:part3_L_fs} и \cref{eq:part3_L_fs_from_link_budget}:

\begin{equation}
  \label{eq:part1_fspl_model_r}
  R = 10^\frac{L_{fs} - 20\lg{F} + G_{tr} + G_{recv} - K}{20}.
\end{equation}





\subsection{Модель распространения SUI}

Модель распространения SUI (Stanford University Intern) предложена рабочей группой, занимающейся исследованием беспроводной широкополосной сети IEEE 802.16 \cite{Mollel2014}. Модель включена в стандарты IEEE и широко используется в WiMax, а также в LTE \cite{Zreikat2017}. Подходит для использования в сельской местности с различным типом рельефа, а также в небольших населенных пунктах. Модель испытана на равнинах, пересеченной, холмистой местности и лесных массивах. SUI модель используется для диапазона частот 1900 МГц -- 11 Ггц \cite{Cabuk2020}. Высоты антенн БС в диапазоне от 10 до 80 м, высота антенны мобильного устройства –- от 2 до 10 м, расстояние между БС и устройством от 0,1 до 8 км.

Грубая оценка потери сигнал описывается с помощью модели SUI как

\begin{equation}
  \label{eq:part1_sui_l0}
  L_0 = A + 10\gamma\lg{(R/R_0)},
\end{equation}
$$
A = 20\lg{(\frac{4\pi R_0 }{\lambda})}
$$
и
$$
\gamma = a - b h_t + \frac{c}{h_t}, 
$$
где $R$ -- дальность связи, $R_0$ -- минимальная разрешенная дальность (100 м), $\lambda = c / f$ -- длина волны, $f$ -- частота в МГц, $h_t$ -- высота антенны БС. Параметры $a, b$ и $c$, определяющие следующие типы местности (Таблица \cref{tab:part1_abc_sui_model}):
\begin{itemize}
  \item тип A -- холмистая местность или густые лесные массивы;
  \item тип B -- пересеченная местность или полугустые лесные массивы;
  \item тип C -- открытые поля.
\end{itemize}


\begin{longtable}[c]{| c | c | c | c |}
  \caption{Численные значения параметров модели SUI.}\label{tab:part1_abc_sui_model}\\

  \hline
  \textbf{Параметры модели} & \textbf{Местность A} & \textbf{Местность B} &  \textbf{Местность C}\\ \hline
  a & 4,6 & 4 & 3.6 \\
  b & 0,0075 & 0,0065 & 0.005 \\
  c & 12,6 & 17,1 &20 \\
  \hline
  \hline
\end{longtable}




% \fixme{ДОДЕЛАТЬ SUI И ДОБАВИТЬ ДВУХЛУЧЕВУЮ}

Формула \cref{eq:part1_sui_l0} была получена эмпирически для несущей частоты 2 ГГц и высоты приемника 2 м. Для использования модели с другими частотами и высотами необходимо добавить поправочные коэффициенты 

\begin{equation}
  \label{eq:part1_sui_lfs}
  L_{fs} = L_0 + \Delta L_f + \Delta L_h + s,
\end{equation}
где $\Delta L_f$ -- корректирующий коэффициент для частот свыше 2 ГГц $\Delta L_h$ --  корректирующий фактор высоты антенны устройства (м) $s$ -- корректирующий фактор теневого эффекта, имеющий значения в диапазоне $8,2 < S < 10,6$ дБ. Параметр $\Delta L_f$ рассчитывается 
$$
\Delta L_f  = 6 \lg{(f / 2000)},
$$
параметр $\Delta L_h$ выбирается исходя из выбора типа местности

$$
\Delta L_h =  
 \begin{cases}
  -10,8 \lg{(h_r/2)} &\text{для типа A и B,}\\
  -20 \lg{(h_r/2)} &\text{для типа C,}
 \end{cases}
$$
где $h_r$ -- высоты антенны устройства.


Из уравнений \cref{eq:part3_L_fs, eq:part1_sui_lfs} можно вывести дальность действия связи:

\begin{equation}
  \label{eq:part1_sui_model_r}
  R = 10^{(\frac{L_{fs} - L_0 - \Delta L_f - \Delta L_h - s - A}{10\gamma} + \lg{R_0})}
\end{equation}

\subsection{Модель двухлучевого распространения}

Двухлучевая модель описывает мощность принятого сигнала как интерференцию двух копий переданного сигнала: первая -- луч прямой видимости, вторая -- отраженная от поверхности \cite{Gaitan2020}. 
Два луча электромагнитных волн от передатчика приходят в приемник с определенной разностью фаз и амплитуд. Разность фаз происходит из-за дополнительного времени распространения волны, отраженного от земли \cite{Rademacher2016, Bacco2014, Zochmann2017, Kurt2017}. 

% \fixme{Проверить}

Мощность принимаемого сигнала, в соответствии с двухлучевой моделью равна

\begin{equation}
  \label{eq:part1_two-ray_model_prcev}
  P_{recv} = \frac{P_{tr} \cdot G_{tr} \cdot G_{recv} \cdot h^2_{tr} \cdot h^2_{recv}}{R^4},
\end{equation}
где $P_{recv}$ -- чувствительность приемника, $P_{tr}$ -- мощность передатчика,$G_{tr}$ -- усиление антенны передатчика, $G_{recv}$ -- усиление антенны приемника, $h_{tr}$ -- высота передатчика, $h_{recv}$ -- высота приемника, $R$ -- расстояние между приемником и передатчиком.

Потери в свободном пространстве из формулы \cref{eq:part1_two-ray_model_prcev} вычисляются как:

\begin{equation}
  \label{eq:part1_two-ray_model_lfs}
  L_{fs} = 40\lg{R} - 10\lg{G_{tr}} - 10\lg{G_{recv}} - 20\lg{h_{tr}} - 20\lg{h_{recv}}.
\end{equation}
Тогда из формул \cref{eq:part3_L_fs, eq:part1_two-ray_model_lfs} дальность рассчитывается как

\begin{equation}
  \label{eq:part1_two-ray_model_r}
  R = 10^\frac{L_{fs} + 10\lg{G_{tr}} + 10\lg{G_{recv}} + 20\lg{h_{tr}h_{recv}}}{40}.
\end{equation}

% \fixme{ПЕРЕДЕЛАТЬ ФОРМУЛУ ФРИИСА, ЧТОБЫ УЧИТЫВАТЬ НЕИЗОТРОПНЫЕ АНТЕННЫ}

% \fixme{ВЕЗДЕ ПРОВЕРИТЬ ЛОГАРИФМ ПО ОСНОВАНИЮ 10}

\subsection{Модель Окамура-Хата}

Модель распространения Окамура-Хата \cite{Hata1980} используется для частотного диапазона 150 -- 1500 МГц, расстояние между БС и абонентским устройством 1 -- 100 метров, высота антенн БС 30--200 метров, высоты антенн абонентских устройств 1 -- 10 метров, дальности телекоммуникационной связи 1--20 км.

Модель Окамура-Хата учитывает особенности территории и плотность застройки: открытая сельская местность, пригородная местность и городская местность. Для каждого случая выражается свой расчет потерь.

Городская  (Urban) местность.
\begin{equation}
  \label{eq:part1_okam_hata_model_l_u}
  \begin{gathered}
  L_u = 69.55 + 26.16\lg{f_c} - 13.82\lg{h_b} - a(h_m) + \\ + (44.9 - 6.55 \lg{h_b}) \lg{R},
  \end{gathered}
\end{equation}

\begin{equation}
  \label{eq:part1_okam_hata_model_l_ufs}
  L_{fs} = L_u,
\end{equation}
где $f_c$ -- несущая частота, $h_b$ --  высота антенн БС, $h_m$ -- высота антенны абонентского устройства, $a(h_m)$ -- поправочный коэффициент. 

Поправочный коэффициент $a(h_m)$ выражается для малых и средних городов

\begin{equation}
  \label{eq:part1_okam_hata_model_a(h_m)_1}
  a(h_m) = (1.1 \lg{f_c} - 0.7) h_m - (1.56 \lg{f_c} - 0.8)
\end{equation}
и для больших городов

\begin{equation}
  \label{eq:part1_okam_hata_model_a(h_m)_2}
  a(h_m) = 
  \begin{cases}
    8.29(\lg{(1.54h_m)})^2 - 1.1, & 150 \leq f_c \leq 200,\\
    3.2(\lg{(11.75h_m)})^2 - 4.97,  & 400 \leq f_c \leq 1500.
  \end{cases}
\end{equation}

Пригородная местность.
\begin{equation}
  \label{eq:part1_okam_hata_model_l_sfs}
  L_{fs} = L_u - 2(\lg(f_c/28))^2 -5.4.
\end{equation}

Сельская (открытая) местность.
\begin{equation}
  \label{eq:part1_okam_hata_model_l_ofs}
  L_{fs} = L_u - 4.78(\lg(f_c))^2 + 18.33\lg(f_c) - 40.94.
\end{equation}


Дальность телекоммуникационной связи вычисляется как:

\begin{equation}
  \label{eq:part1_okam_hata_model_R}
  R = 
  \begin{cases}
    10^\frac{L_{fs} - 69.55 - 26.16\lg{f_c} + 13.82\lg{h_b} + a(h_m)}{44.9 - 6.55 \lg{h_b}}, & \text{городская местность};\\

    10^\frac{L_{fs} - 69.55 - 26.16\lg{f_c} + 13.82\lg{h_b} + a(h_m) + 2(\lg(f_c/28))^2 + 5.4 }{44.9 - 6.55 \lg{h_b}}, & \text{пригородная местность};\\

    10^\frac{L_{fs} - 69.55 - 26.16\lg{f_c} + 13.82\lg{h_b} + a(h_m) + 4.78(\lg(f_c))^2 - 18.33\lg(f_c) + 40.94 }{44.9 - 6.55 \lg{h_b}}, & \text{сельская местность}.\\
  \end{cases}
\end{equation}

\subsection{Расчет параметров БС, необходимых для формулировки и решения задач оптимизации}

Используя любую из представленных моделей распространения (уравнения \cref{eq:part1_fspl_model_r}, \cref{eq:part1_sui_model_r}, \cref{eq:part1_two-ray_model_r}  или \cref{eq:part1_okam_hata_model_R}) можно рассчитать входные параметры задачи синтеза топологии: теоретическую максимальную дальность телекоммуникационной связи $ R_{jq}$ между станциями и радиус телекоммуникационного покрытия $ r_j $ с предположением об отсутствии препятствий, отражений, влияния контуров местности и т. д. Это допущение приемлемо для случая размещения БС БШС вдоль протяженной транспортной магистрали с открытой местностью, рассматриваемого в диссертационной работе.

\begin{figure}[h!]
  \centering
   \includegraphics[width=0.8\textwidth]{link_distance.pdf}
\caption{Соединение между базовыми станциями.}
\label{fig:part3_link_distance}
\end{figure}

Для расчета дальности связи $R_{jq}$ (Рисунок \cref{fig:part3_link_distance}), станции $s_j$ и $s_q$ будут рассматриваться как станции \textit{передатчик} и \textit{приемник}, соответственно. Будем считать, что станции оборудованы направленными антеннами с усилениями $G_{tr}^{R}$ и $G_{recv}^{R}$.

\begin{figure}[h!]
  \centering
   \includegraphics[width=0.8\textwidth]{coverage.pdf}
\caption{Телекоммуникационное покрытие базовой станции}
\label{fig:part3_coverage}
\end{figure}

Каждая БС оснащена всенаправленной антенной с чувствительностью антенны $G_ {recv}^{r}$. Данная антенна необходимо для покрытия заданной области.

% Each base station is equipped with an omnidirectional antenna with given gain antenna $G_{tr}^{r}$. A station uses this antenna to cover a given area.

При вычислении радиуса покрытия $r_j$ (Рисунок  \cref{fig:part3_coverage}) базовую станцию будем считать \textit{приемником}, а пользовательское устройство  \textit{передатчиком}.

% \fixme{В приложения А и Б представлены примеры использования модели распространения в свободном пространстве (Free space propagation model) для решения задачи оптимизации.}
% \FloatBarrier
\section{Оценка характеристик производительности сети с помощью стохастических моделей массового обслуживания}

В диссертационной работе рассматривается задача максимизации области телекоммуникационного покрытия БС при проектировании БШС.  С увеличением дальности телекоммуникационного покрытия сети ухудшается качества передачи : уменьшается мощность передаваемого сигнала, в свою очередь падает пропускная способность, увеличивается время передачи пакета по сети. В связи с этим одной из важнейших задач при проектировании является оценка характеристик производительности будущей сети. 

% Одной из такой характеристик является величина средней межконцевой задержки. Данная характеристика представляет собой время необходимое для передачи пакета через сеть от источника до места назначения.

В качестве оценки характеристики производительности сети в диссертационной работе исследуется величина межконцевой задержки. Данная характеристика представляет собой время необходимое для передачи пакета через сеть от источника до места назначения.  В параграфе будет описана общая структура кадров и процедура передачи пакетов в канале для БШС, описанных в этой главе ранее. Для оценки времени передачи пакетов будет рассмотрены стохастические модели сетей массового обслуживания.

% \fixme{Добавит абзац, зачем описываем структуру кадров и время передачи пакета}
% \fixme{удалить это}
% \begin{itemize}
%   \item формат кадра WiMax
%   \item формат кадра LTE
%   \item формат кадра 5G;
% \end{itemize}


\subsection{Структура кадра Wi-Fi}
 Основой стандарта 802.11 является распределенная функция координации DCF (Distributed Coordination Function). Согласно данному механизму, станция, желающая передать пакет данных, слушает шумы в передающей среде в течение заранее определённого периода времени. Если передающая среда свободна, станция может передавать пакеты. В противном случае, станция ждет определенное количество времени, прежде чем опять предпринять попытку отправки пакета.

 Время передачи пакета в беспроводной среде, главным образом зависит от коллизии. Протокол 802.11 борется с этой проблемой с помощью метода множественного доступа к среде с прослушиванием среды и избежанием коллизий (Carrier-Sense Multiple Access with Collision Avoidance, CSMA/CA), когда станции выполняют процедуру отсрочки (backoff procedure) передачи для всех узлов, которые ожидают передачи. Для решений проблемы <<скрытой станции>> стандарт поддерживает механизм RTC/CTS (Request To Send / Clear To Send), представленный на рисунке \cref{fig:part1_interframe_time}.

\begin{figure}[h!]
  \centering
   \includegraphics[width=1\textwidth]{interframe_time.png}
\caption{Процедура Backoff}
\label{fig:part1_interframe_time}
\end{figure}

На MAC-подуровне  для определения состояния среды передачи существуют межкадровые интервалы (Inter Frame Space, IFS) -- периоды ожидания между передачей кадров. В стандарте определено несколько типов межкадровых интервалов:

\begin{itemize}
  \item уменьшенный межкадровый интервал (Reduced IFS, RIFS), используется в некоторых случаях, когда необходимо уменьшить накладные расходы при передаче;
  \item короткий межкадровый интервал (Short IFS, SIFS), необходим беспроводному устройству для обработки принятого кадра и ответом об его получении;
  \item межкадровый интервал функции PCF (IFS, PIFS);
  \item межкадровый интервал функции DCF (IFS, DIFS);
  \item арбитражный межкадровый интервал (Arbitration IFS, AIFS), используется устройствами для получения доступа к среде с помощью метода EDCA, поддерживающими QoS (Quality Of Service);
  \item расширенный межкадровый интервал (Extended IFS, EIFS), является самым длинным интервалом, используется устройством в случае, если кадр пришел с ошибкой.
\end{itemize}

В течение времени равному интервалу DIFS базовая станция, желающая передать кадр должна определить состояние среды. По истечению времени DIFS, если среда свободна, начинается процедура отсрочки. Станция начинает период молчания случайной длины, равномерно распределенного в интервале $[0, CW]$, где $CW$ -- это окно конкуренции (Contetion Window) равное целому числу из интервала $CW_{min} < CW < CW_{max}$ \cite{Proletarsky}. Далее в соответствии c алгоритмом RTS/CTS каждый узел сети, перед тем как послать данные в «эфир», сначала отправляет специальное короткое сообщение, которое называется RTS (Ready To Send). Это говорит о готовности узла к отправке данных. Такое RTS-сообщение, содержащее информацию о продолжительности предстоящей передачи и об адресате, доступно всем узлам в сети, если только они не скрыты от отправителя. Это позволяет другим узлам задержать передачу на время, равное объявленной длительности сообщения NAV (Network Allocation Vector). Приемная базовая станция, получив сигнал RTS, отвечает посылкой сигнала CTS, свидетельствующий о готовности станции к приему информации через интервал SIFS. После успешного получения кадра CTS передающая станция выжидает время SIFS и посылает пакет данных. По завершению приемная станция должна послать передатчику кадр ACK, подтверждающий безошибочный прием. Если подтверждение нет, передатчик делает вывод об ошибке \cite{Tanenbaum2022}. В таком случае отправитель удваивает период молчания и повторяет передачу кадра, продолжая экспоненциально увеличивать длину паузы. Увеличение $CW$ происходит, пока кадр успешно не будет передан или не будет достигнуто максимальное число повторов.


В 802.11e добавили функцию гибридной координации HCF, основанная на DCF и PCF. В стандарте появился расширенный распределенный доступ к каналу (Enhanced Distributed Channel Access, EDCA).  Метод обеспечивает доступ к среде, использую 8 уровней приоритетов пользователей (User Priority, IP) и 4 категории доступа (Access Categories, AC): background (AK\_BK), best effort (AK\_BE), video (AK\_VI), voice (AK\_VO). На беспроводном устройстве, реализующем QoS, поддерживается четыре независимые очереди (Рисунок \cref{fig:part1_edca_prior}). Вместо одного таймера DIFS назначаются различные таймеры AIFS и продолжительность $CW$ для каждой категории AC. Когда станция в сети желает начать передачу, она ждет, пока NAV уменьшится до 0. После, если канал свободен, станция ждет время AIFS и сразу отсчитывает CW до завершения.

\begin{figure}[h!]
  \centering
   \includegraphics[width=.85\textwidth]{edca_prior.png}
\caption{Очереди с поддержкой QoS}
\label{fig:part1_edca_prior}
\end{figure}



Функция HCF обеспечивает возможность передавать станции сразу последовательность кадров. Когда станция получает доступ к среде передачи, ей выделяется определенный период времени TXOP (Transmisson Opportunity). Интервал TXOP  -- это время, в течение которого станция может отправлять кадры после того как получит доступ к среде передачи. TXOP необходим для того, чтобы увеличить пропускную способность высокоприоритетных данных, таких как видео и голос.

\subsection{Структура кадра LTE}

% \fixme{Дать описание кадра на канальном уровне}

При формировании информационного ресурса в стандарте LTE используются два метода. Первый метод -- мультиплексирование с
ортогональным частотным разнесением (Orthogonal Frequency Division Multiplexing, OFDM), используется для нисходящего канала DL от базовой станции к абонентскому устройству. Второй метод -- мультиплексирование с частотным разнесением с передачей на одной несущей (Single-Carrier Frequency Division Multiple Access, SC-FDMA), используется для восходящего канала UL от абонентского устройства к базовой станции. 
 

Стандарт LTE на физическом уровне использует технологию  Orthogonal Frequency Division Multiplexing (OFDM), решающая проблему межсимвольной интерференции при высокоскоростной передаче с многолучевым распространением сигнала. Технология OFDM предполагает передачу широкополосного сигнала посредством независимой модуляции узкополосных поднесущих. Один OFDM символ содержит набор модулированных поднесущих. Во временной области OFDM-символ включает поле данных и циклический префикс, повторно передаваемый в конце предыдущего символа. Данный префикс необходим для борьбы с межсимвольной интерференции в приемнике, вследствие многолучевого распространения сигнала. Отраженный сигнал, приходящий с задержкой, попадает в зону префикса и не накладывается на полезный сигнал. В LTE принят шаг между поднесущими $\Delta f = 15$  кГц, что соответствует длительности OFDM-символа 66,7 мкс \cite{Dahlman2013}.

Для формирования группового сигнала восходящих каналов в сетях LTE используется схема мультиплексирования с частотным разнесением с передачей на одной несущей SC-FDMA. Эта схема может работать в системах, функционирующих как в режиме с временным дуплексированием, так и в режиме с частотным дуплексированием. 

Существующий канальный ресурс состоит из ресурсных блоков, каждый из которых включает 12 последовательных поднесущих в частотной области, занимающих полосу 180 кГц, и одного интервала 0,5 мс во временной области. Таким образом, каждый блок ресурсов состоит из $7 \cdot 12 = 84$ ресурсных элементов в случае обычного циклического префикса и $6 \cdot 12 = 72$ ресурсных элементов в случае расширенного циклического префикса \cite{Dahlman2013}.

Стандарт поддерживает две структуры кадров: частотный дуплекс (Frequency Division Duplex, FDD) и временной дуплекс (Time Division Duplex, TDD). Практически вся обработка на физическом уровне идентична для FDD и TDD, разница между ними в основном заключается в структуре кадров, как показано на рисунке \cref{fig:part1_lte_phy_frame}. Передача по радиоканалу осуществляется кадрами длиной $T_{\text{кадр}} = 10$ мс. Все временные параметры в спецификации LTE привязаны к минимальному временному кванту $T_s = 1 / (2048 \cdot \Delta f)$, где $\Delta f$ -- шаг между поднесущими.

\begin{figure}[h!]
  \centering
   \includegraphics[width=1\textwidth]{lte_phy_frame.png}
\caption{Структура кадра LTE}
\label{fig:part1_lte_phy_frame}
\end{figure}

В случае работы FDD имеются две несущие частоты: для передачи по восходящей линии связи ($f_{UL}$) и для передачи по нисходящей линии связи ($f_{DL}$). Каждый кадр состоит из 10 субкадров восходящей линии связи и 10 субкадров нисходящей линии связи \cite{David2009}. Передача по восходящей и нисходящей линиям связи может происходить одновременно в пределах одной соты. Каждый субкадр состоит из двух смежных слотов длительностью 0,5 мс. При полномдуплексном режиме радиоканалы в восходящем и нисходящем каналах передаются параллельно, но с оговоренным в стандарте сдвигом \cite{Vishnevsky2009}. 

В случае работы TDD имеется только одна несущая частота. Передачи по восходящей и нисходящей линиям связи всегда разделены во времени \cite{David2009}. Для обеспечения ассимметрии трафика восходящего и нисходящего каналов, в TDD поддерживаются 7 различных конфигураций, соответствующих разным соотношениям восходящего и нисходящего каналов (таблица \cref{tab:part1_lte_tdd_config}). Буква <<D>> соответствует субкадру нисходящей линии связи, <<U>> соответствует субкадру восходящей линии связи, а <<S>> соответствует специальному субкадру. Как видно из таблицы, в субкадрах 0 и 5 всегда осуществляется передача <<вниз>>, а в субкадре, следующим за специальным, всегда осуществляется переда <<вверх>>.

\begin{table}[h!]\centering
  \begin{tabular}{|c|c|c|c|c|c|c|c|c|c|c|c|}\hline
      \multirow{2}{*}{\thead{Конфигурация \\ вверх-вниз}} & \multirow{2}{*}{\thead{Периодичность \\ вверх-вниз}}	&\multicolumn{10}{|c|}{Номер субкадра} \\
      \cline{3-12}
      & &0&1&2&3&4&5&6&7&8&9 \\
      \hline
      0& 5 мс& D& S& U& U& U& D& S& U& U& U \\
      1& 5 мс &D &S &U &U &D &D &S &U &U &D \\
      2& 5 мс &D &S &U &D &D &D &S &U &D &D \\
      3& 10 мс &D &S &U &U &U &D &D &D &D &D \\
      4& 10 мс &D &S &U &U &D &D &D &D &D &D \\
      5& 10 мс &D &S &U &D &D &D &D &D &D &D \\
      6& 5 мс &D &S &U &U &U &D &S &U &U &D\\
      \hline

\end{tabular}\caption{ Конфигурации кадра в LTE при временном дуплексе (TDD)}\label{tab:part1_lte_tdd_config}
\end{table}

Важной частью любой системы TDD является обеспечение достаточно больших защитных периодов, в течение которых оборудование может переключаться между передачей и приемом без перекрытия передаваемых и принимаемых сигналов. В LTE защитные периоды создаются путем разделения одного или двух субкадров, называемых специальными субкадрами. Информация в специальных субкадрах разделена на три поля: часть нисходящего канала (Downlnk Pilot Time Slot, DwPTS), защитный период (Guard Period, GP) и часть восходящего канала (Uplink Pilot Time Slot, UpPTS). 

% \fixme{Проверить терминологию субкадр и подкадр}

\subsection{Структура кадра 5G NR}
В сетях 5G NR передача данных в восходящем UL и нисходящем DL направлениях организуется на основе кадров (frame) длительностью $T_f=10$ мс. Каждый кадр делится на 10 субкадров (subframe) длительностью $T_{sf}= 1$ мс. Каждый субкадр содержит $2^\mu$ слотов. \cite{Lien2017, Omri2019}. 

Одной из ключевых особенностей 5G, открывающая многие преимущества, является тот факт, что расстояние между поднесущими может принимать различные значения, в отличие от LTE, где значение фиксировано и равно 15 кГц. В зависимости от $\mu$ (таблица \cref{tab:part1_5g_slot}) в сетях 5G-NR используются OFDM поднесущие с различной шириной спектра: 15кГц, 30кГц, 60кГц, 120кГц и 240кГц. Количество слотов определяется шириной спектра поднесущей и составляет 1, 2, 4, 8 или 16 слотов, в отличие от сетей LTE, где используется единая структура с двумя слотами на каждый субкадр.

\begin{table}[h!]\centering
  \begin{tabular}{|c|c|c|c|}\hline

      $\mu$ & \multirow{3}{0.2\textwidth}{Интервалы \\ поднесущих, \\ $\Delta f$, кГц }& \multirow{3}{*}{Циклический префикс} & \multirow{3}{*}{Количество слотов} \\
      &&& \\
      &&& \\
      \hline
      0& 15 & Обычный& 1  \\
      1& 30 & Обычный& 2  \\
      2& 60 & Обычный, Расширенный& 4  \\
      3& 120 & Обычный& 8  \\
      4& 240 & Обычный& 16  \\

      \hline

\end{tabular}\caption{ Интервалы поднесущих}\label{tab:part1_5g_slot}
\end{table}

Использование различных частот открывает широкие возможности для гибкой настройки сети при предоставлении тех или иных услуг. Так для приложений критичных к уровню задержек, целесообразно использовать поднесущие с широким спектром при меньшей длительности символа, и наоборот, при передаче трафика широкополосного доступа в интернет и низкоскоростного трафика интернета вещей -- использовать узкий спектр поднесущих. Уплотнение развертывания засчет уменьшения размерами сот и использование высоких частот с большей пропускной способностью может привести к уменьшению задержки радиоинтерфейса.


\begin{figure}[h!]
  \centering
   \includegraphics[width=.9\textwidth]{5g_frame_structure.png}
\caption{Структура кадра 5G NR}
\label{fig:part1_5g_frame_structure}
\end{figure}

\FloatBarrier
\subsection{Расчет межконцевой задержки}\label{part4_e2e_delay_section}

% \fixme{Добавить статью с машинным обучением}

Как уже было отмечено ранее, любая сеть обязана гарантировать качество обслуживания QoS. При проектировании БШС необходимо оценивать характеристики производительности будущей сети. Одной из таких характеристик сети является межконцевая (сквозная) задержка \cite{Vishnevsky2016_Methods_of_performance, Wang2017, Liu2016, Chen2019, Hosni2017, Capone2019, Abbas2017, Seliem2019, Malandra2018, Kalor2018, Larionov2019, Gao2016}, характеризующее максимальное время передачи пакетов в сети от источника до места назначения. Для расчета сквозной задержки сети часто используют стохастические модели массового обслуживания \cite{Vishnevsky2016_Methods_of_performance, Wang2017, Liu2016, Malandra2018, Larionov2019, Gao2016}. 

% При проектировании необходимо оценивать характеристики производительности БШС. Одной из таких характеристик производительности сети является межконцевая задержка.

\subsubsection{Расчет оценок с помощью  моделей массового обслуживания}

В работе исследуются различные стохастические модели многофазной сети массового обслуживания (СеМО) с кросс-трафиком. Точность моделей СеМО зависит от выбора моделей распределений входящего потока и времени обслуживания на фазах. В классическом случае время между поступлениями заявок в СеМО задается экспоненциальной функцией распределения \cite{VishnevskyBook, Kleinrock1975}

\begin{displaymath}
  F(t) = 1 - e^{-\lambda t}, f(t) = \lambda e^{-\lambda t},
\end{displaymath}
где $\lambda$ -- интенсивность между поступлениями заявок (1/с).

Для аппроксимации случайной величины времени обслуживания используют экспоненциальное распределение
\begin{displaymath}
  F(t) = 1 - e^{-\mu t}, f(t) = \mu e^{-\mu t},
\end{displaymath}
где $\mu$ -- интенсивность времени обслуживания заявок в обслуживающем приборе (1/с).


В работах \cite{Vishnevsky2017, Vishnevsky2016_Methods_of_performance, Larionov2019, Kawanishi2019, Dogan2021} для оценки времени обслуживания  используют распределение фазового типа ($PH$). $PH$ - распределение является обобщением экспоненциального распределения, случайная величина которой выражается как время, проведенное в цепи Маркова до попадания в поглощающее состояние. Цепь Маркова состоит из $V+1$ состояний, нулевое является поглощающим. Попадая в него, заявка считается обслуженной. $PH(\overline{\tau}, S)$-распределение задается вектором начального распределения $\overline{\tau} \in \mathbb{R}^V$ и  матрицей перехода $S \in \mathbb{R}^{V \times V}$, для которых должно выполняться:


\begin{displaymath}
	\label{eq:part1_ph_dist}
	\begin{aligned}
		&\tilde{S} = \begin{bmatrix}
			S  & -S\mathbf{1} \\
			\mathbf{0} &  0
		\end{bmatrix} \mbox{"--- инфинитезимальная матрица,}\\
		&\forall i = \overline{1,V}: \: 0 \leq \tau_i \leq 1, \qquad
			\sum\limits_{i=1}^{V} \tau_i = 1.
	\end{aligned}
\end{displaymath}

Существуют более сложные модели очередей для оценок характеристик с более сложными видами распределения входящего трафика и времени обслуживания. Адекватные оценки дают модели с коррелированными входным потоком \cite{VishnevskyDudin2017, Vishnevsky2017, Vishnevsky2016_Methods_of_performance, Larionov2019, Samouylov2022} и с фазовым распределением времени обслуживания \cite{Vishnevsky2017, Vishnevsky2016_Methods_of_performance, Larionov2019, Kawanishi2019, Dogan2021}. $PH$-распределения дают точные оценки времени обслуживания, однако модели СеМО с фазовыми распределениями труднорешаемы и использование их в задачах оптимизации может привести к большим временным затратам на получение оценок.

В работе были исследованы различные многофазные модели СеМО. Рассматривался простейший случай многофазной сети с узлами $M/M/1/ \infty$ ($M/M/1$), в котором поступает пуассоновский поток и время обслуживания в узлах задается экспоненциальным законом, размер буфера неограничен. Для такой модели, используя закон Литтла, можно получить аналитическое решение для расчета средних задержек. Рассматривались также модель СеМО с конечными размером буфера, т.е. с узлами $M/M/1/N$, и модель c распределением времени обслуживания фазового типа -- СеМО с узлами $M/PH/1/N$. Такие сети очень тяжело исследовать аналитически, поэтому для вычисления их характеристик использовались имитационные модели (метод Монте-Карло). Валидация исследуемых моделей проводилась с помощью имитационного моделирования в среде NS-3 для БШС протокола IEEE 802.11n. Подробно эксперимент описан в работе \cite{Larionov2021}.  По умолчанию в NS-3 размер буфера MAC-уровня составляет $N=500$ пакетов \cite{ns-3}. 



% Для таких моделей расчет времени задержки проводится методом Монте-Карло, так как при большом числе фаз трудно получить аналитическое решение.

Сравнение средних временных задержек, рассчитанных с помощью моделей СеМО представлено на рисунке \cref{fig:comparison_queue_models}. Как видно из графиков СеМО с узлами $M/M/1$ показывает достаточное приближение к данным NS-3 в сравнении с более сложными моделями СеМО с узлами $M/M/1/N$ и СеМО с узлами $M/PH/1/N$.

% \fixme{В работе исследуются различные стохастические модели многофазной сети массового обслуживания (СеМО) с кросс-трафиком, экспоненциальным распределением интервалов между поступлениями пакетов и различными распределениями времени обслуживания. Рассмотрены модели многофазной СеМО: с узлами $M/M/1/N$ -- с экспоненциальным распределением времени обслуживания и конечным размером буфера, с узлами $M/PH/1/N$ -- с фазовым распределением времени обслуживания и конечным размером буфера. Такие модели труднорешаемы, для которых тяжело получить аналитическое решение. Оценка времени задержки для сетей с большим числом фаз возможно только с помощью метода Монте-Карло. Также исследовалась простейшая модель СеМО с узлами $M/M/1$, для которой существует аналитическое решение. Модели сравнивались с задержками, полученные с помощью имитационного моделирования в среде NS-3 для БШС протокола IEEE 802.11n. Сравнение моделей представлено на рисунке \cref{fig:comparison_queue_models}. Как видно из графиков СеМО с узлами $M/M/1$ показывает достаточное приближение.} 



% В классическом случае обслуживания в узлах сети и время между поступлениями заявок задаются экспоненциальной функцией распределения \cite{VishnevskyBook,Kleinrock1975}. 


% Существуют более сложные модели очередей для оценок характеристик с более сложными видами распределения входящего трафика и времени обслуживания. Адекватные оценки дают модели с коррелированными входным потоком \cite{VishnevskyDudin2017, Vishnevsky2017, Vishnevsky2016_Methods_of_performance, Larionov2019, Samouylov2022}. Для аппроксимации времени обслуживания в беспроводных каналах используют фазовые распределения \cite{Vishnevsky2017, Vishnevsky2016_Methods_of_performance, Larionov2019, Kawanishi2019, Dogan2021}. К сожалению, такие модели труднорешаемы и использование их в задачах оптимизации нецелесообразно в связи с большими временными затратами на расчет. 
% Исследования таких СеМО представлены в работе \cite{Larionov2021}, в которой был предложен метод калибровки моделей массового обслуживания с помощью имитационного моделирования в среде NS-3 БШС протокола IEEE 802.11n. На вход поступали пакеты, сгенерированные по экспоненциальному закону. С помощью NS-3 была получена выборка, для которой было восстановлено PH-распределения по трем моментам для случая с узлами $M/PH/1/N$ и экспоненциальное распределение по среднему значению для случая с узлами $M/M/1/N$. По умолчанию в NS-3 размер буфера MAC-уровня составляет $N=500$ \cite{ns-3}.
%  Сравнение моделей представлено на рисунке \cref{fig:comparison_queue_models}. На рисунке также представлена модель $M/M/1$ с бесконечным буфером. Как видно из графиков СеМО с узлами $M/M/1$ показывает достаточное приближение. Выбор модели с бесконечным буфером обусловлен тем, что производители не указывают в документации размер выделенной памяти для хранения пакетов в своем оборудовании и определение его размера является отдельной большой задачей. Из результатов сравнения видно, что $M/M/1/N$ и $M/M/1$ отличаются несущественно. Потери пакетов, часто происходят не из-за переполнения буфера памяти, а коллизий и потерь в беспроводном канале. Существенным преимуществом СеМО с узлами $M/M/1$ перед другими более сложными моделями является наличие доказанного быстрого аналитического решения.

\begin{figure}[h!]
  \centering
   \includegraphics[width=1\textwidth]{comparison_queue_models.png}
\caption{Сравнение моделей массового обслуживания с данными NS-3}
\label{fig:comparison_queue_models}
\end{figure}


% \subsubsection{Оценка межконцевой задержки с помощью методов МО}
\subsubsection{Расчет оценок с помощью методов машинного обучения}

В теории массового обслуживания при построении СеМО основным допущением является независимость обслуживания на каждых фазах. Каждый раз при поступлении новой заявки время обслуживания на фазах разыгрывается случайным образом. В действительности в реальных сетях это не так, при малых загрузках в БШС, обслуживание пакетов на всех узлах коррелируют друг с другом и в большой степени зависит от размера пакетов. В работах \cite{LazarevaLarionovMukhtarovITTMM2020_RSCI,VishnevskyLarionovMukhtarovICAM2020_RSCI} автором был представлен случай, в котором размер пакетов фиксируется при их первом появлении в имитационной модели и не изменяется в течении всего времени обслуживания на фазах СеМО. При первом появлении пакета время обслуживания характеризуется длиной пакета и битовой скоростью, приписанной фазе СеМО.  На каждой последующей фазе время обслуживания не разыгрывается случайно, а зависит от длины уже сгенерированного входящего пакета. В таком случае нарушается главное допущение о независимости, обеспечивающее аналитический вывод среднего времени пребывания заявки в системе для СеМО с фазами $M/M/1$. 

Для быстрого расчета времени задержки для СеМО с фиксацией размеров пакетов при их появлении было предложено использовать прогнозные модели, построенные на данных имитационного моделирования с зависимым обслуживанием. Аппроксимационные модели строились с помощью методов машинного обучения (МО): деревья решений, градиентный бустинг, искусственные нейронные сети. На рисунке \cref{fig:part1_e2e_delay_ml} представлены графики оценок межконцевых задержек на тестовой выборке от коэффициента загрузки для различных длин тандема, полученных с помощью искусственной нейронной сети с одним скрытым слоем. Зеленым цветом представлен данные аналитической модели с независимым обслуживанием, фиолетовым цветом представлены данные имитационной модели с фиксированными размерами пакетов, т.е. с зависимым обслуживанием, и пунктирными линиями представлены прогнозные значения регрессионной модели для тандемов длиной 7, 12, 15 и 18 БС.

\begin{figure}[h!]
  \centering
   \includegraphics[width=1\textwidth]{e2e_delay_ml.png}
\caption{Прогнозная модель величины межконцевой задержки}
\label{fig:part1_e2e_delay_ml}
\end{figure}

Анализ графиков и рассчитанные метрики по предложенным регрессионным моделям величины межконцевой задержки позволяют сделать вывод о целесообразности использования прогнозирующих моделей, построенных с помощью методов МО, для оценки характеристик сетей в практических и теоретических приложениях. Такую сеть интересно рассматривать в случае БШС с топологии <<точка-точка>>, построении беспроводного моста, радиорелейной связи и т.д. В таких сетях при малых загрузках потери в каналах играют несущественную роль, и время передачи в канале пропорционально зависит от длины пакета. 

% В таком случае между узлами сети появляется связь, которая нарушает допущение о независимости обслуживания на всех фазах. Данное нарушение не позволяет использовать аналитические методы. 

% Расчёт характеристик производительности БШС в классическом случае проводится с помощью стохастических моделей массового обслуживания. БШС рассматривают как многофазную сеть массового обслуживания с узлами $M/M/1$, т.е. с экспоненциальным распределением времени между поступлениями входящих пакетов и времени обслуживания на всех фазах. Время обслуживания пакетов на каждой фазе разыгрывается случайным образом и независимо от обслуживания на других фазах. 
% В работах \cite{LazarevaLarionovMukhtarovITTMM2020_RSCI,VishnevskyLarionovMukhtarovICAM2020_RSCI} был представлен случай, в котором фиксируется размер пакетов при их первом появлении и не изменяющийся в течении всего времени обслуживания в сети. На каждой фазе время обслуживания пакета характеризуется длиной пакета и битовой скоростью, приписанная данной фазе.  На каждой последующей фазе время обслуживания не разыгрывается случайно, а зависит от длины уже сгенерированного входящего пакета. Такую сеть интересно рассматривать в случае БШС с топологии <<точка-точка>>. В таких сетях потери в каналаъ играют несущественную роль, и время передачи в канале при невысоких коэффициентах загрузки пропорционально зависит от длины пакета. В таком случае между узлами сети появляется связь, которая нарушает допущение о независимости обслуживания на всех фазах. Данное нарушение не позволяет использовать аналитические методы. 

% Было предложено использовать аппроксимационные модели на данных имитационного моделирования с зависимым обслуживанием в СеМО. Аппроксимационные модели строились с помощью методов машинного обучения: деревья решений, градиентный бустинг, искусственные нейронные сети. На рисунке \cref{fig:part1_e2e_delay_ml} представлены графики оценок межконцевых задержек на тестовой выборке от коэффициента загрузки для различных длин тандема, полученных с помощью искусственной нейронной сети с одним скрытым слоем. Зеленым цветом представлен данные аналитической модели с независимым обслуживанием, фиолетовым цветом представлены данные имитационной модели с фиксированными размерами пакетов, т.е. с зависимым обслуживанием, и пунктирными линиями представлены прогнозные значения регрессионной модели для тандемов длиной 7, 12, 15 и 18 БС.


\FloatBarrier
\subsubsection{Многофазная сеть массового обслуживания с узлами \textit{M}/\textit{M}/1}


С учетом быстрого аналитического решения и достаточного приближения для оценки величины межконцевой задержки в ходе поиска оптимального размещения БС БШС было решено в работе использовать СеМО с узлами $M/M/1$ и кросс-трафиком. Выбор модели с бесконечным буфером обусловлен тем, что часто производители оборудования не указывают в паспортной документации размер выделенной памяти для хранения пакетов, и определение его размера является отдельной большой задачей. Из результатов сравнения видно, что $M/M/1/N$ и $M/M/1$ отличаются несущественно. Потери пакетов, часто происходят не из-за переполнения буфера памяти, а коллизий и потерь в беспроводном канале. Существенным преимуществом СеМО с узлами $M/M/1$ перед другими более сложными моделями является наличие доказанного быстрого аналитического решения.


% В диссертационной работе было решено использовать аналитическую модель для оценки задержки для СеМО с узлами $M/M/1$ для задачи оптимального размещения БС БШС. 
Согласно предложенной концепции проектирования, в дальнейшем полученную БШС, после этапа синтеза топологии, можно будет проверить на более сложных моделях. На следующем шаге -- на этапе математического моделирования более основательно проводится оценка характеристик производительности как время задержек, длины очередей, пропускная способность, вероятность потери пакетов и т.д. Такой подход проектирования позволяет провести комплексную проверку соответствия QoS для полученного размещения БС. На этапе синтеза топологической структуры будет проводиться расчет с помощью быстрой модели для отсева вариантов размещения БС с большими задержками.

Рассмотрим специальный случай БШС. Все БС связаны последовательно между собой в сеть, образуя линейную топологию. Для расчета межконцевой задержки представим БШС как многофазную сеть массового обслуживания (СеМО) с кросс-трафиком и узлами $M/M/1$ (Рисунок \cref{fig:tandem_queue}). Кросс-трафик подразумевает поступление пакетов с некоторой заданной интенсивностью $\lambda$ на каждую фазу СеМО. Узлами сети являются БС. Согласно символике Дж. Кендала, обозначение $M$ указывает на показательное распределение случайной величины \cite{VishnevskyBook, Kleinrock1975}. Каждая такая БC характеризуется случайными величинами входящего потоком пакетов и временем их обслуживания, принадлежащие экспоненциальному закону распределения. Каждый узел имеет один обслуживающий прибор. Для такой СеМО принято допущение о бесконечном размере буфера, в котором пакеты ожидают своего обслуживания. Данное допущение позволяет получить аналитическое решение, которое возможно использовать для произвольного размера СеМО данной топологии.

\begin{figure}[h!]
  \centering
   \includegraphics[width=1\textwidth]{tandem_queue.png}
\caption{СеМО с кросс-трафиком и узлами $M/M/1$}
\label{fig:tandem_queue}
\end{figure}
 
На вход каждой станции поступает пуассоновский поток. Пуассоновский процесс представляет собой случайный процесс, характеризующийся  экспоненциально распределенным временем между появлением событий. Это один из наиболее важных случайных процессов в теории вероятностей, который широко используется для моделирования поведения трафика во многих коммуникационных сетях и системах \cite{Kalor2018, Gao2016, Malandra2018, Seliem2019}. 

В пуассоновском процессе события происходят непрерывно и независимо друг от друга. Функция распределения имеет вид  \cite{VishnevskyBook, Kleinrock1975}:

\begin{displaymath}
P(X<x) = 
  \begin{cases}
    1 - e^{- \lambda x}, \quad x \geqslant 0; \\
    0, \quad x < 0.\\
  \end{cases}
\end{displaymath}

Для входящего потока интервалы между поступлениями заданы случайной величиной с экспоненциальным распределением и интенсивностью $\lambda$. Время обслуживания на узле задана также экспоненциальным распределением и интенсивностью $\mu$. 


По теореме Бурке \cite{Burke1956}, поток на выходе узла $M/M/1$, а значит на входе каждой последующей фазы тоже пуассоновский. Интенсивность на выходе каждой фазы равна суммарной интенсивности всех входящих потоков с интенсивностями $\lambda$.

Пропускная способность на практике часто составляет половину от заданной в спецификации оборудования \cite{Proletarsky, Vladimirov2019}. Интенсивность времени обслуживания рассчитывается по формуле: 

\begin{displaymath}
    \mu_j = 0.5 \cdot p_j / w,
\end{displaymath}
где: $p_j$ - пропускная способность $j$-ой станции, Мбит/с; $w$ - средний размер пакета, Мбит.

Для каждой станции коэффициент загрузки равен:


\begin{displaymath}
\rho_j= \frac{\sum{\lambda}}{\mu_j} = \frac{q \cdot \lambda}{\mu_j} <1,
\end{displaymath}
где $q$ -- число входящих потоков. Условие $\rho_j<1$ является необходимым и достаточным условием существования стационарного режима функционирования СеМО.

Далее по формуле Литтла \cite{Little1961} можно рассчитать время задержки на каждой станции:

\begin{displaymath}
    \overline{T_j} = \frac{\rho_j}{1 - \rho_j} \cdot \frac{1}{q \cdot \lambda}.
\end{displaymath}

Тогда межконцевая задержки в сети равна

\begin{equation}
    \label{eq:end_to_end_delay}
    \overline{T}= \sum{\overline{T_j}}.
\end{equation}

Представленный расчет межконцевой задержки будет использован в главе 2 при описании разработанного комбинаторного алгоритма.

\section{Выводы по главе 1}
 
В главе представлены следующие результаты исследования:

\begin{enumerate}
  \item Обоснована актуальность БШС для организации связи вдоль протяженных транспортных магистралей: автомобильные и железные дороги, линии метрополитена, магистральные и промысловые нефте- и газопроводы.
  \item Представлены этапы комплексного проектирования БШС и место в нем задачи синтеза топологий. 
  \item Проведен анализ современных беспроводных широкополосных технологий, широко использующихся при проектировании БШС вдоль протяженных участков автомобильных и железных дорог, линий метрополитена и магистральных трубопроводов.
  \item Проведен анализ методики расчета дальности телекоммуникационной связи при развертывании БШС на открытой местности.
  \item Проведен анализ расчета  средних задержек в каналах БШС с помощью стохастических моделей массового обслуживания. 

\end{enumerate}

Результаты исследования, представленные в этой главе, были опубликованы в работах \cite{Larionov2021, LazarevaLarionovMukhtarovITTMM2020_RSCI, VishnevskyLarionovMukhtarovICAM2020_RSCI, }.






% \begin{figure}[h!]
%   \centering
%    \includegraphics[width=1\textwidth]{industry4.png}
% \caption{Задача синтеза топологии при проектировании БШС в рамках цифровой трансформации "Индустрия 4.0".}
% \label{fig:industry4}
% \end{figure}





\FloatBarrier
