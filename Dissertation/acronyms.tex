\printnomenclature[3.5cm] % Значение ширины столбца с обозначениями стоит подбирать вручную
\nomenclature{БПЛА}{беспилотный летательный аппарат}
\nomenclature{БШС}{беспроводная широкополосная сеть}
\nomenclature{МВиГ}{метод ветвей и границ}
\nomenclature{МПП}{метод полного перебора}
\nomenclature{ЦЛП}{целочисленное линейное программирование}
\nomenclature{БС}{базовая станция}
% 5G сеть

\nomenclature{URLLC}{Ultra-Reliable and Low Latency Communications, cверхнадежная передача данных с малой задержкой}
\nomenclature{URLLC}{Ultra-Reliable and Low Latency Communications, cверхнадежная передача данных с малой задержкой}
\nomenclature{UL}{Up Link, восходящая линия связи}
\nomenclature{DL}{Down Link, нисходящая линия связи}

% TODO:  чекнуть все сокращения
\nomenclature{ТМО}{теория массового обслуживания}
\nomenclature{МО}{машинное обучение}
\nomenclature{EIRP}{Equivalent Isotropically Radiated Power, эффективная изотропно-излучаемая мощность передатчика}
\nomenclature{SOM}{System Operating Margin, запас на замирание сигнала}
\nomenclature{FSPL}{Free Space Path Loss, уравнение потерь в свободном пространстве}
\nomenclature{QoS}{Quality Of Service, качество обслуживания}

\nomenclature{СеМО}{сеть массового обслуживания}
\nomenclature{ИТС}{интеллектуальная транспортная система}
\nomenclature{VANET}{Vehicular Ad Hoc network, автомобильная самоорганизующаяся сеть}
\nomenclature{RSU}{Roadside Unit, придорожные стационарные объекты телекоммуникационной связи}

\nomenclature{V2V}{Vehicle-To-Vehicle, беспроводная связь для обмена данными между движущимися транспортными средствами}


\nomenclature{V2X}{Vehicle-To-Infrastructure, беспроводная связь для обмена данными между транспортным средством и инфраструктурой}

\nomenclature{LIFO}{Last Input, First Out, последним пришёл — первым ушёл}
\nomenclature{ЛП}{линейное программирование}
\nomenclature{ЧЦЛП}{частично целочисленное линейное программирование}