\chapter*{Заключение}                       % Заголовок
\addcontentsline{toc}{chapter}{Заключение} 
Создание и развитие современной структуры передачи данных являются неотъемлемой частью произвоздства. В рамках комплексного проектирования беспроводных средств связи в данной работе исследуются проблемы синтез топологии беспроводных сетей связи.

Основные результаты данного диссертационного исследования следующие:
\begin{enumerate}
    \item В ходе исследования был проведен анализ современного состояния и тенденций развития БШС на месторожденях.
    \item Был проведен анализ методики проектирования современных БШС. В рамках такой методики были исследованы проблемы синтеза топологии БШС.
    \item Был проведен анализ проблем проектирования БШС вдоль протяженных участков: трубопроводные магистрали, протяженные автомобильные дороги. Для телекоммуникационного покрытия таких объектов были предложены модели максимального покрытия.
    \item Была предложена математическая модель ЦЛП для размещения базовых станций с линейной топологией.
    \item Была представлена математическая модель экстремальной задачи в комбинаторной форме. 
    % Данная модель учитывает специфику задачи для нахождения оптимального решения.
    \item Для комбинаторной модели был разработан специальный алгоритм типа МВиГ. Представлены результаты сравнения поиска оптимального решения с помощью МВиГ с алгоритмами решения задачи в общем виде.
    \item В рамках комплексного проектирования была представлена итерационная процедура нахождения последовательности лучших решения задачи оптимального размещения для случая, когда оптимальное решение не удовлетворяет ограничения поставленным на этапе моделирования БШС.
    \item Предложены формулировка задачи ЛП проверки допустимого потока при заданных мест размещения станций.
    \item Предложена математическая модель ЧЦЛП оптимального размещения станция из имеющегося множества типов станций на избыточном множестве типов станций.
\end{enumerate}


Поставленные в начале данной диссертационной работы цели были достигнуты, а соответствующие им задачи решены. В качестве перспектив дальнейшего исследования можно выделить следующие:
\begin{itemize}
    \item внедрение программных реализаций в состав итерационной процедуры комплексного проектирования беспроводных сетей связи;
    \item  применение предложенного МВиГ в решении более общих   задач оптимального размещения базовых станций, в которых кроме времени межконцевой задержки необходимо учитывать не только  других характеристик производительности БШС;
    \item исследование влияний реальной среды на физический уровень канала передачи данных для расчета оценок дальности телекоммуникационной связи.
    
\end{itemize}


% Добавляем его в оглавление

%% Согласно ГОСТ Р 7.0.11-2011:
%% 5.3.3 В заключении диссертации излагают итоги выполненного исследования, рекомендации, перспективы дальнейшей разработки темы.
%% 9.2.3 В заключении автореферата диссертации излагают итоги данного исследования, рекомендации и перспективы дальнейшей разработки темы.
%% Поэтому имеет смысл сделать эту часть общей и загрузить из одного файла в автореферат и в диссертацию:

% Основные результаты работы заключаются в следующем.
% %% Согласно ГОСТ Р 7.0.11-2011:
%% 5.3.3 В заключении диссертации излагают итоги выполненного исследования, рекомендации, перспективы дальнейшей разработки темы.
%% 9.2.3 В заключении автореферата диссертации излагают итоги данного исследования, рекомендации и перспективы дальнейшей разработки темы.

\begin{enumerate}
    \item Проведен анализ методики проектирования современных беспроводных широкополосных сетей. В рамках такой методики были исследованы проблемы синтеза топологии беспроводных сетй вдоль протяженных транспортных магистралей: автомобильные дороги, трубопроводные магистрали, лини метрополитена, железные дороги. 
    \item Предложена новая математическая модель в виде задачи целочисленного линейного программирования оптимального размещения базовых станций с линейной топологией.
    \item Представлена новая математическая модель задачи оптимального размещения БС в виде комбинаторной модели в экстремальной форме. 
    % Данная модель учитывает специфику задачи для нахождения оптимального решения.
    \item Для комбинаторной модели разработан новый специальный алгоритм типа ветвей и границ, учитывающий специфике решения задачи размещения базовых станций широкополосной сети вдоль протежянных транспортных магистралей.  
    
    % Представлены результаты сравнения поиска оптимального решения с помощью МВиГ с алгоритмами решения задачи в общем виде.
    \item В рамках комплексного проектирования БШС представлена новая итерационная процедура нахождения последовательности лучших решений задачи оптимального размещения базовых станций для случая, когда найденное оптимальное решение построения топологии сети не удовлетворяет  критериями функционирования БШС, проверяемых на следующих этапах проектирования.
    \item Предложена новая математическая модель виде задачи частично целочисленного линейного программирования оптимального размещения базовых станций для покрытия множества рассредоточенных объектов. 
    \item Разработан программный комплекс для расчета задачи оптимального размещения базовых станций с помощью нового алгоритма типа ветвей и границ.
    \item Представлены результаты численных экспериментов, доказывающие эффективность предложенных моделей и методов для решения задачи синтеза топологии при проектировании БШС.
\end{enumerate}

% \begin{enumerate}
%     \item Был проведен анализ методики проектирования современных БШС. В рамках такой методики были исследованы проблемы синтеза топологии БШС вдоль протяженных участков: трубопроводные магистрали, протяженные автомобильные дороги. 
%     \item Была предложена математическая модель в виде задачи ЦЛП размещения БС с линейной топологией.
%     \item Была представлена математическая модель экстремальной задачи оптимального размещения БС в комбинаторной форме. 
%     % Данная модель учитывает специфику задачи для нахождения оптимального решения.
%     \item Для комбинаторной модели был разработан специальный алгоритм типа ветвей и границ. Представлены результаты сравнения поиска оптимального решения с помощью МВиГ с алгоритмами решения задачи в общем виде.
%     \item В рамках комплексного проектирования была представлена итерационная процедура нахождения последовательности лучших решений задачи оптимального размещения для случая, когда найденное оптимальное решение построение топологии сети не удовлетворяет некоторым критериями функционирования БШС, проверяемых на этапе моделирования процесса передачи данных.
%     \item Предложена математическая модель оптимального размещения БС для покрытия множества рассредоточенных объектов.
%     \item Разработан программный комплекс для расчета задачи оптимального размещения БС  с помощью предложенного алгоритма типа ветвей и границ.
% \end{enumerate}




% \begin{enumerate}
%   \item построены математические модели в виде экстремальной комбинаторной задачи и задачи ЦЛП для оптимального размещения базовых станций при проектировании БШС с линейной топологией;
%   \item представлен алгоритм метода ветвей и границ для задачи размещения базовых станций с линейной топологией; 
%   \item разработана итерационная процедура нахождения последовательности лучших решений для задачи размещения базовых станций в рамках комплексного проектирования БШС с линейной топологией;
%   \item разработаны математические модели для задач проектирования БШС для покрытия множества рассредоточенныз объектов;
%   \item разработаны модели прогнозирования оценок характеристик производительности сети с помощью методов машинного обучения.
% \end{enumerate}

% И какая-нибудь заключающая фраза.

% Последний параграф может включать благодарности.  В заключение автор
% выражает благодарность и большую признательность научному руководителю
% Иванову~И.\,И. за поддержку, помощь, обсуждение результатов и~научное
% руководство. Также автор благодарит Сидорова~А.\,А. и~Петрова~Б.\,Б.
% за помощь в~работе с~образцами, Рабиновича~В.\,В. за предоставленные
% образцы и~обсуждение результатов, Занудятину~Г.\,Г. и авторов шаблона
% *Russian-Phd-LaTeX-Dissertation-Template* за~помощь в оформлении
% диссертации. Автор также благодарит много разных людей
% и~всех, кто сделал настоящую работу автора возможной.
