\chapter{Размещение базовых станций беспроводной широкополосной сети для обслуживания множества рассредоточенных объектов}\label{ch:ch3}
 
В данной главе будут представлены модели задачи синтеза топологии при развертывании БШС на плоскости для телекоммуникационного покрытия множества рассредоточенных объектов. 

\section{Актуальность внедрения БШС для обслуживания рассредоточенных объектов}

Построение современной инфраструктуры передачи информации для обслуживания множества объектов промышленного или гражданского назначения, рассредоточенных на некоторой территории, является актуальной задачей при создании единой систем контроля и управления указанными объектами.  Создание такой инфраструктуры позволяет обеспечить оперативный контроль и управление объектами путем передачи необходимой информации с сенсоров и датчиков объектов в соответствующий внешнее приемное устройство. Для создания подобной инфраструктуры эффективно используются сети широкополосной беспроводной связи, необходимым этапом проектирования которых является решение задачи определения мест размещения базовых станций \cite{VishnevskyBook}.

В работах \cite{Cicek2019, Medvedeva2020, Khayrov2020} представлены задачи максимального телекоммуникационного покрытия рассредоточенных объектов с использованием БПЛА.

% \fixme{В работе \cite{Mehmood2016} предложен новый протокол сенсорной сети на базе IEEE 802.11 для мониторинга случаев загрязнений углеводородами. В работе \cite{Abbas2021} исследуются различные протоколы сенсорных сетей для мониторинга над газораспределительной сети. Вся сеть разделена на более мелкие, управляемые сегменты, каждый из которых имеет свою базовую станцию для отправки пакетов в центральный пункт управления. В \cite{Bukhari2021} решают задачу размещения мощностей с помощью генетического алгоритма. Авторы занимаются развертыванием устройств распределенных вычислений, серверов, вблизи устройств конечных пользователей. Связующим звеном между конечным пользователем и сервером являются базовые станции.}

В настоящей работе строятся и исследуются две математические модели задач размещения БС, которые применимы на этапе синтеза топологии сети в процессе комплексного проектирования мультимедийных сетей. Предлагается модель для проверки существования допустимого решения при условии выполнении ограничений для предложенной на предыдущих этапах схемы расстановки БС и модель для оптимизационной задачи. Оптимизационная задача состоит в выборе множества БС из заданного набора типов БС с различными характеристиками и их расстановки на избыточном множестве возможных мест размещения. В поставленной задаче рассматривается задача обслуживания объектов, расположение которых задано их координатами на плоскости. Особенностью такой задачи в широком классе задач оптимального размещения мощностей является наличие условия на наличие информационной связи между БС и внешним приемным устройством (шлюзом), выполнение которого гарантирует поступление всей информации с контролируемых объектов в центр управления. 

Предложена задача оптимального размещения БС, принадлежащая к широкому классу задач размещения мощностей (Location Allocation Problem). В рамках широкого класса задач размещения мощностей в данных задачах размещения присутствуют специфика на связь между всеми узлами сети. 


% Эффективная политика кэширования контента на границе сети мобильной сотовой связи может улучшить качество услуг для мобильных пользователей и уменьшить перегрузку сети на транзитном рейсе. С другой стороны, виртуализация беспроводной сети становится передовым методом решения проблемы ограниченной пропускной способности сети из-за экспоненциального роста трафика мобильных данных. Кроме того, виртуализация беспроводной сети может принести огромные преимущества, такие как сокращение капитальных затрат (CAPEX) и эксплуатационных расходов (OPEX), а также повышение пропускной способности сети. В этом отношении одним из ключевых требований для признания преимуществ вышеупомянутых двух технологий является наличие применимой структуры распределения ресурсов, которая позволяет развертывать схемы кэширования контента в виртуализированной беспроводной сети. В этом исследовании мы исследуем новую проблему совместного распределения радиоресурсов и кэширования контента, чтобы эффективно использовать блоки радиоресурсов, мощность передачи и доступную кэш-память на базовых станциях (BS). Цель сформулированной проблемы, представленной в этой статье, направлена ​​на минимизацию задержек, с которыми сталкиваются конечные мобильные пользователи операторов мобильных виртуальных сетей (MVNO). Мы показываем, что сформулированная задача является невыпуклой смешанной целочисленной нелинейной задачей (MINLP), которая NP-трудна и просто неразрешима. Поэтому мы применяем алгоритм блочной минимизации верхней границы (BSUM) для решения сформулированной задачи. Численные результаты показывают, что наш метод превосходит существующие базовые схемы распределения ресурсов с приростом производительности до 19% с точки зрения сетевой задержки.}




\section{Математическая модель задачи проверки допустимого решения  при заданных местах размещения станций.}
Модели задачи оптимизации, которые исследуются в диссертации, предлагается использовать при проектировании БШС на этапе синтеза топологии. После ввода в эксплуатацию сети часто требуется модернизировать, так как любое производство непрерывно развивается. Со временем, телекоммуникационную сеть требует усовершенствование своей инфраструктуры: масштабирование с целью увеличения покрытия сети, демонтаж оборудования, смена протоколов и т.д. Любое изменение приводит к тому, что необходимо провести качество обслуживания сети QoS, надежность и в целом проверить возможно ли обеспечить телекоммуникационное покрытие будущей сети. В данном параграфе будет представлена задача оптимизации при уже заданных размещения БС. В такой постановке возможность сбора такой информации с множества рассредоточенных объектов и поиска кратчайшего пути передачи пакетов от множества объектов к шлюзу через множества размещенных БС.


\subsection{Постановка задачи}

Задано множество узлов БШС рассредоточенных на плоскости. Все множество можно разбить на две категории:
\begin{itemize}
    \item объекты, с которых необходимо собирать информацию, являются оконечными узлами сети;
    \item станции для сбора и передачи на шлюз данных с объектов, являются промежуточными узлами сети; 
\end{itemize}
Под объектом понимается любое устройство с антенной для передачи пакетов в канале. К ним можно отнести измерительные устройства, шлюзы сенсорных сетей и т.д. В частности, объектами могут быть любые стационарные абонентские устройства сети 802.11n.

Задано множество вершин $A= \left\{ a_i \right\}, i=\overline{0,n}$ на некоторой территории. Каждая вершина $a_i$ имеет координаты $\left\{ x_i, y_i \right\}$.

Множество $A$ состоит из двух подмножеств:
\begin{itemize}
    \item $A_1$ –- множество вершин, соответствующее объектам; с которых необходимо собирать информацию. 
    \item $A_2$ -- множество мест, где размещены БС. В дальнейшем вершину из $A_2$ будем идентифицировать  не только как место размещения, но и как соответствующую станцию.
\end{itemize}
С вершин $A_1$ необходимо собирать информацию. Каждой вершине $a_i \in A_1$ приписана величина $v_i$ -- максимальный объем информации в единицу времени, который генерирует расположенный на этой вершине объект.  В дальнейшем будем считать, что каждая вершина из $A_1$ является, непосредственно, объектом. В дальнейшем вершинs  $a_i \in A_2$ будем идентифицировать не только как место размещения, но и как соответствующую станцию.

По определению:
$$
A_1 \cup A_2 = \varnothing;
$$

$$
A_1 \cap A_2 = A.
$$

Все вершины пронумерованы так, что:
$$
A_1 = \left\{a_i \right\}, i= \overline{1,n_1};
$$

$$
A_2 = \left\{ a_i  \right\}, i= \overline{n_1+1,n}.
$$

Каждой БС, размещенной на вершине множества $A_2$ приписаны три параметра $s_i = \left\{ \{r_{ij}\}, \{R_{ij}\},\vartheta_i \right\} $, где:

\begin{itemize}
    \item $\{r_{ij}\}$ -- множество радиусов телекоммуникационного покрытия БС. Параметр $r_{ij}$ характеризует дальность связи между БС размещенной в вершине $a_i, a_i \in A_2$ и объектом в вершине $a_j, a_j \in A_1$;
    \item $\{R_{ij}\}$ -- множество радиусов связи БС. Параметр $R_{ij}$ характеризует дальность связи между станциями $s_i$ и $s_j$, $i= \overline{n_1+1,n}, j = \overline{n_1+1,n}, i \neq j$;
    \item $\vartheta_i$ -- объем информации в единицу времени, который может быть получен от объектов, обслуживаемых БС.
\end{itemize}

Также БС специального вида -- шлюз  $s_0 = \left\{\{R_{0j}\}, \vartheta_0 \right\} $, размещенная на вершине $a_0$ с координатами $\left\{x_0, y_0 \right\}$. Данная БС не имеет телекоммуникационного покрытия и служит для сбора всей информации в сети. По условию задачи величина $\vartheta_0$ больше суммы величин $\vartheta_i$ всех вершин множества $A_1$.

Задано условие, со шлюзом и между собой могут быть связаны только вершины множества $A_2$, то есть только станции.

Требуется проверить, что при заданных наборе и размещении БС на множества $A_2$ вся имеющаяся информация с объектов множества $A_1$ может быть собрана и передана системой БС  до шлюза $s_0$.

\subsection{Модель линейного программирования}
Перед тем как приступить к задаче оптимизации, необходимо подготовить правила составления графа сети, в соответствии с постановкой задачи.

\subsubsection{Построение матрицы смежности}

Составим граф $ H = \left\{A,E \right\} $ для возможного потока информации между вершинами множества $ A = A_1 \cup A_2 $. По определению, каждой вершине $a_i \in A_2 $ соответствует БС $s_i$ со своим набором параметров $s_i = \left\{ \{r_{ij}\}, \{R_{ij}\},\vartheta_i \right\} $.

Матрица смежности $E = \left\{ e_{ij} \right\}$ графа $H$ строится по следующим правилам:

\begin{itemize}
    \item $e_{ij} = 1$, если расстояние между $i$-ым объектом вершины $a_i \in A_1$ и $j$-ой БС, размещенной на вершине $a_j \in A_2$ не более радиуса покрытия для БС соответствующего этой вершине типа; 
    \item $e_{ij} = 1$, если расстояние между $i$-ой БС на вершине $a_i \in A_2$ и $j$-ой БС на вершине $a_j \in A_2$, не более минимального из радиусов связей этих станций;
    \item $e_{i0} = 1$, если расстояние от вершины $a_i \in A_2$ до шлюза не более минимального из радиусов связей БС и шлюза;
    \item $e_{ij} = 0$, во всех остальных случаях.
\end{itemize}

\subsubsection{Формулировка в виде задачи линейного программирования}
С помощью полученной матрицы смежности $E$, необходимо подготовить условия ограничения для величины потока в каналах.

Введем переменные $x_{ij} \geqslant 0$, определяющее количество информации, передаваемой в единицу времени по дуге $e_{ij}$ графа $H$.

Каждый объект множества $A_1$ генерирует пакеты объемом $\vartheta_i$ в единицу времени. Для канала $e_{ij}, i = \overline{1,n_1}, j = \overline{n_1+1,n}$ величина потока равна весу $\vartheta_i$:

\begin{equation}\label{eq:part2_1.1}
    \sum_{a_j \in \Gamma^+(a_i)} x_{ij} = \vartheta_i, \forall a_i, i=\overline{1, n_1},
\end{equation}
где $\Gamma^+(a_i)$ – множество вершин на графе $H$, в которые входят дуги, исходящие из вершины $a_i$. 

Для каждой вершины $a_i,  a_i \in A_2$ необходимо обеспечить выполнения условия баланса между потоком входящем в эту вершину от объектов множества $A_1$, а также других БС множества $A_2$ и выходящего потока из данной вершины. 

Сумма входящих и выходящих потоков для любой вершины $a_i$  множества $A_2$ должна быть равна нулю:

\begin{equation}\label{eq:part2_1.2}
    \sum_{a_j \in \Gamma_1^-(a_i)} x_{ji} + \sum_{a_j \in \Gamma_2^-(a_i)} x_{ji} -  \sum_{a_j \in \Gamma_2^+(a_i)} x_{ij} =0 ,\forall a_i \in A_2. 
\end{equation}

Здесь множество $\Gamma_1^-(a_i)$ -– вершины множества $A_1$, из которых выходят дуги, входящие в вершину $a_i$; $\Gamma_2^-(a_i)$ –- вершины множества $A_2$, из которых выходят дуги, входящие в  вершину $a_i$; $\Gamma_2^+(a_i)$ –- вершины множества $A_2$, в которые входят дуги, исходящие из вершины $a_i$.


Необходимо чтобы на выходе сети собирался весь трафик. Через систему БС на вершинах $a_j, a_j \in A_2$, вся информация от объектов на вершинах $a_i, a_i \in A_1$ поступала  на шлюз $s_0$:

\begin{equation}\label{eq:part2_1.3}
    \sum_{a_j \in \Gamma_2^-(a_0)} x_{j0} =  \sum_{a_i \in A_1} \vartheta_i;
\end{equation}


Поток объема информации в каналах ограничен сверху. В случае каналов передачи от объектов на вершинах $A_1$ до БС на вершинах $A_2$ поток ограничен объемом сгенерированного трафика на объекте $\vartheta_i$:

% \begin{equation}\label{eq:part2_1.4_1}
%     \sum_{a_j \in \Gamma_2^+(a_i)} x_{ij} \leqslant \vartheta_i, \forall a_i \in A_1.
% \end{equation}

\begin{equation}\label{eq:part2_1.4_1}
    x_{ij} \leqslant \vartheta_i, \forall a_i \in A_1, a_j \in A_2.
\end{equation}

% \fixme{Объем информации выходящий из станции на вершине $a_j, a_j \in A_2$ ограничен пропускной способностью $\vartheta_j$ станции : }

% \begin{equation}\label{eq:part2_1.4_2}
%     \sum_{a_i \in \Gamma_2^+(a_j)} x_{ji} \leqslant \vartheta_j, \forall a_j \in A_2.
% \end{equation}

Объем информации входящий на БС на вершине $a_j, a_j \in A_2$ ограничен пропускной способностью $\vartheta_j$ станции : 


\begin{equation}\label{eq:part2_1.4_2}
    \sum_{a_i \in \Gamma^-(a_j)} x_{ij} \leqslant \vartheta_j, \forall a_j \in A_2.
\end{equation}


Если к системе уравнений ограничений \cref{eq:part2_1.1, eq:part2_1.2, eq:part2_1.3, eq:part2_1.4_1, eq:part2_1.4_2} добавить целевую функцию

\begin{equation}
    \label{eq:part2_1.5}
    \sum_{(a_i, a_j) \in A} c_{ij} x_{ij} \rightarrow \min ,
\end{equation}
где $c_{ij}$ -- стоимость потока в ребре, тогда данная модель является задачей о потоке минимальной стоимости. Задача о потоке минимальной стоимости играет одну из основных ролей в области оптимизации сетей \cite{Kovacs2015}. Она используется для нахождения минимальной стоимости потока с множества узлов поставок до множества узлов потребителей в направленном графе с ограничениями на пропускную способность и целевой функцией стоимости, зависящей от пути потока в графе. Задача имеет широкой спектр приложений в различных областях: задачах транспортировки, расписания, ресурсного планирования, телекоммуникации, проектировании сетей и маршрутизации \cite{Kovacs2015, Kiraly2012, Jiang2020}.  

% Представленная модель является задачей ЛП, которую можно решить с помощью симплекс-метода \cite{Dantzig1963}. 

% Для решения классическим способом задачи ЛП необходимо задавать ориентированный граф. В случае модели \cref{eq:part2_1.1, eq:part2_1.2, eq:part2_1.3, eq:part2_1.4_1, eq:part2_1.4_2}, ребра графа между станциями $s_j$ в точках $A_2$ двунаправленные. Предполагается, что передача информации может идти в обоих направлениях либо от $s_i$ к $s_j$ через ребро $w_{ij}$, либо от $s_j $ к $s_i$ через ребро $w_{ji}$, соответственно, $i= \overline{n_1+1,n}, j= \overline{n_1+1,n}, i \neq j$ (Рисунок \cref{fig:part3_edges_between_stations}).  Данное особенность задачи портит решение, полученное с помощью классического симплекс метода в линейной форме. Допустимым решением в таком случае будет являться сеть, содержащая циклы между узлами $s_i$ и $s_j$. Для получения объективного решения воспользуемся сетевым симплекс методом, чтобы учесть специфику задачи. 

% \begin{figure}[h!]
%     \centering
%      \includegraphics[width=.7\textwidth]{edges_between_stations.png}
%   \caption{Направления потоков между станциями.}
%   \label{fig:part3_edges_between_stations}
%   \end{figure}



С момента публикации Данцигом симплекс-метода \cite{Dantzig1963}, изначально разработанного для задач транспортировки, были получены много новых усовершенствованные моделей, большой обзор метод представлен автором в \cite{Kovacs2015}. Одним из популярных методов решения является сетевой симплекс-метод, которой представляет собой версию хорошо известного симплекс метода ЛП, использующий графовое представление задачи о потоке минимальной стоимости. Метод симплекс-типа применяется для решения задач потока минимальной стоимости. Сетевой симплекс алгоритм с наилучшей стоимостью был разработан Орлином \cite{Orlin1997} в сочетании с древовидной структурой данных Тарьяна \cite{Tarjan1997}. Алгоритм симплекс-метода основана на концепции нахождения минимального остовного дерева. Более подробно алгоритм нахождения решения в виде остовного дерева представлен в работах \cite{Kiraly2012, Kovacs2015, Holzhauser2017, Jiang2020}.

Для нахождения допустимого решения задачи \cref{eq:part2_1.1, eq:part2_1.2, eq:part2_1.3, eq:part2_1.4_1, eq:part2_1.4_2,  eq:part2_1.5} (или доказательства, что допустимого решения не существует) можно найти возможный граф передачи потока информации от объектов до шлюза, если ввести единичные стоимости $c_{ij}$ передачи потока $w_{ij}$ по ребру $e_{ij}$ задача \cref{eq:part2_1.1, eq:part2_1.2, eq:part2_1.3, eq:part2_1.4_1, eq:part2_1.4_2,  eq:part2_1.5} является задачей поиска кратчайшего пути от передачи информации к шлюзу. 
% \fixme{проверить эту задачу}

\subsubsection{Пример решения задачи ЛП}


\begin{figure}[h!]
    \centering
     \includegraphics[width=.8\textwidth]{lp_input.png}
  \caption{Заданное размещение.}
  \label{fig:part3_lp_input}
  \end{figure}

На рисунке \cref{fig:part3_lp_input} представлен пример заданного размещения БС. Задано множество рассредоточенных объектов $A_1, |A_1| = 5$. Задано множество БС и точки их размещения $A_2, |A_2| = 3$. Координаты множества $A, A = A_1 \cup A_2 $ представлены в таблице \cref{tab:part3_lp_input_coordinates} и мощности узлов сети представлены в таблице \cref{tab:part3_lp_intensity}. Необходимо проверить, возможно ли при данном наборе БС собрать всю информации с объектов  и передать ее на шлюз $s_0$, размещенной в точке $a_0$. 

\begin{table}[h!]\centering
    \begin{tabular}{| c| c|c |}\hline
        $a_0$& (6, 15)& Координаты шлюза \\
        \hline
        \hline
        $a_1$&(1, 2) & Координаты объектов \\
        $a_2$&(14, 13) &  \\
        $a_3$&(1, 8) & \\
        $a_4$&(12, 6) &  \\
        $a_5$&(4, 12) &  \\
        \hline
        \hline
        $a_6$&(4, 2) & Координаты размещения станций \\
        $a_7$&(15, 10) & \\
        $a_8$&(6, 8) & \\
        \hline
    \end{tabular}\caption{Координаты вершин}\label{tab:part3_lp_input_coordinates}
\end{table}

\begin{table}[h!]\centering
\begin{tabular}{| c| c c c c c |c c c|}\hline
    $\vartheta_0$& $\vartheta_1$& $\vartheta_2$& $\vartheta_3$& $\vartheta_4$& $\vartheta_5$& $\vartheta_6$& $\vartheta_7$ & $\vartheta_8$\\
    \hline
    \hline
    $\infty$ & 11&	12&	13&	14&	15&	110& 120 & 130\\
    \hline
\end{tabular}\caption{Мощности узлов графа}\label{tab:part3_lp_intensity}
\end{table}
   

По паспортным характеристиками оборудования и уравнениями представленными в разделе \cref{section:part_1_link_distance} были получены параметры БС: радиус телекоммуникационного покрытия $r_{ij}$ и радиус связи между станциями $R_{ij}$. С помощью этих параметров была получена матрица смежности $E$ граф потока $H$ (таблица \cref{tab:part3_lp_adj_mat}).




\begin{table}[h!]\centering
\begin{tabular}{|c|| c| c c c c c| c c c|}\hline
    
    & $a_0$& $a_1$& $a_2$& $a_3$& $a_4$& $a_5$& $a_6$& $a_7$ & $a_8$\\
    \hline
    \hline
    $a_0$ & 0&	0&	0&	0&	0&	0&	0 &	0&	0\\
    \hline
    $a_1$ & 0&	0&	0&	0&	0&	0&	1 &	0&	0\\
    $a_2$ & 0&	0&	0&	0&	0&	0&	0 &	1&	1\\
    $a_3$ & 0&	0&	0&	0&	0&	0&	1 &	0&	1\\
    $a_4$ & 0&	0&	0&	0&	0&	0&	0 &	1&	1\\
    $a_5$ & 0&	0&	0&	0&	0&	0&	0 &	0&	1\\
    \hline
    $a_6$ & 0&	0&	0&	0&	0&	0&	0 &	0&	1\\
    $a_7$ & 1&	0&	0&	0&	0&	0&	0 &	0&	1\\
    $a_8$ & 1&	0&	0&	0&	0&	0&	1 &	1&	0\\
    \hline
\end{tabular}\caption{Матрица смежности графа потока}\label{tab:part3_lp_adj_mat}
\end{table}



\begin{figure}[h!]
    \centering
        \includegraphics[width=.8\textwidth]{lp_solution.png}
    \caption{Допустимое решение.}
    \label{fig:part3_lp_solution}
\end{figure}

Теперь можно решить задачу ЛП \cref{eq:part2_1.1, eq:part2_1.2, eq:part2_1.3, eq:part2_1.4_1, eq:part2_1.4_2, eq:part2_1.5}. На рисунке \cref{fig:part3_lp_solution} представлен полученный граф допустимого решения. Жирными линиями представлена телекоммуникационная связь между объектами и БС. Стрелками указан полученный граф потока информации от объектов до шлюза. Математическая модель была подготовлена на языке Python, задачи ЛП была рассчитана с использованием библиотеки с открытым исходным кодом NetworkX \cite{networkx}. 

% \fixme{проверить текст}

% может быть применена стандартная процедура нахождения допустимого решения задачи линейного программирования с вводом искусственных переменных в уравнения \cref{eq:part2_1.1, eq:part2_1.2, eq:part2_1.3, eq:part2_1.4_1, eq:part2_1.4_2} и минимизации состоящей из этих переменных линейной формы. Если значение целевой функции в результате решения задачи окажется больше нуля, то допустимого решения для данного размещения станций не существует, в противном случае полученное решение дает допустимое распределение потоков по каналам связи.

% Далее мы рассмотрим понятие базисных структур в контексте BCMCFPR. В отличие от сетевого симплексного алгоритма для традиционной задачи потока с минимальными затратами (см. [2, стр. 405–407]), нам нужно отказаться от предположения, что подграф, индуцированный базовыми ребрами, не имеет циклов. Вместо этого в основе лежит цикл с ненулевой платой за использование, как это будет подробно показано ниже. 



% \fixme{================================================================}

% \fixme{После получения, матрицы смежности $H$ необходимо, убедиться, что данный граф является связным. Хотя может это и проверяем в ЛП. Пока так оставить}

% \fixme{ПРОВЕРИТЬ Задача о потоке минимальной стоимости}
% \fixme{Примем допущение, что интенсивность везде одинакова и равна $\lambda$ . Потом умножить на средий размер пакетов.}
% \fixme{Проверить идею кратчайшего остовного дерева. Проверить как считает стоимости}
% \fixme{================================================================}


\FloatBarrier
\section{Математическая модель оптимальной задачи выбора набора размещаемых БС и определения мест их размещения}

В данном параграфе будет представлена оптимизационная задача размещения типов БС БШС для обеспечения телекоммуникационного покрытия рассредоточенных объектов. Задача имеет ту же постановку как для модели ЛП, теперь  только множество вершин $A_2$ задаются свободными. Необходимо разместить БС из заданного множества типов БС для развертывания БШС на плоскости.

\subsection{Постановка задачи.}

Задано множество вершин $A = \{a_i\}$, $i=\overline{0,n}$ на некоторой территории. Каждая вершина $a_i$ имеет координаты $\left\{ x_i, y_i \right\}$.
Множество $A$ состоит из двух подмножеств: 
\begin{itemize}
    \item $A_1$ -- множество вершин, с которых необходимо собирать информацию;
    \item $A_2$ -- множество возможных мест размещения БС. 
\end{itemize}
Каждой вершине $a_i$ приписана   величина $v_i$ -- максимальный объем информации, снимаемой с объекта, расположенного на этой вершине.

По определению

$$
A_1 \cup A_2 = A;
$$

$$
A_1 \cap A_2 = \varnothing.
$$

Все вершины пронумерованы так, что:

$$
A_1 = \left\{a_i \right\}, i= \overline{1,n_1};
$$

$$
A_2 = \left\{ a_i  \right\}, i= \overline{n_1+1,n}.
$$


Задано множество типов БС $S = \{s_j$\}, $j=\overline{1,m}$, которые необходимо разместить на множестве точек $A_2$.

Каждому типу БС приписаны четыре параметра $s_j = \left\{\{r_{ji}\}, \{R_{ji}\}, \vartheta_j, c_j \right\}$, где: 
\begin{itemize}
    \item $\{r_{ji}\}$ -- множество радиусов покрытия. Параметр $r_{ji}$ характеризует телекоммуникационную связь для обеспечения соединения между $j$-ой станцией и объектом, размещенный в координате $a_i$, $j= \overline{n_1+1,n}$, $i= \overline{1,n_1}$;
    \item $\{R_{ji}\}$ -- радиус связи между $j$-ой и $i$-ой станциями. Параметр характеризует максимальную дальность связи $j$-ой станции, обеспечивающее заданное качество соединения с $i$-ой станцией, $j= \overline{n_1+1,n}$, $i= \overline{n_1+1,n}$, $j \neq i$;
    \item $\vartheta_j$ -- пропускная способность;
    \item $c_j$ -- стоимость.
\end{itemize}

% максимальный объем информации в единицу времени, который может быть получен от объектов, обслуживаемых данной станцией

Задана БС специального вида (шлюз) $s_0 = \left\{ \{R_{0j}\}, \vartheta_0 \right\}$ с координатами $\left\{x_0, y_0 \right\}$. Шлюз уже имеет свое расположение, стоимость размещения $c_0 = 0$. Параметр шлюза $\{R_{0j}\}, j = \overline{n_1+1,n}$ радиус связи необходим для соединения с размещаемыми БС. Полагается, что шлюз не имеет соединения напрямую с объектами. По шлюз $s_0$ позволяет собрать данные со всех объектов, размещенных в точках $a_i$, $i= \overline{1,n_1}$, в данной постановке задачи пропускная способность шлюза равна $\vartheta_0 = \infty$.


Множества вершин $A_1$ будем идентифицировать как размещенные на них объекты. Множества вершин $A_2$, на которых будут размещены БС, будем рассматривать, непосредственно, как сами БС. 

Требуется разместить БС таким образом, чтобы вся информация с объектов на вершинах множества $A_1$ могла быть собрана и передана системой БС, размещенных на выбранных в результате решения задачи в вершинах множества  $A_2$, до шлюза $s_0$ и итоговая стоимость размещения была бы минимальной.

% \fixme{Вершин и станции будем, соответственно, идентифицировать как объекты или станции на них размещенные.}

Задано условие, что информация с вершин множества $A_1$ может передаваться непосредственно только на вершины множества $A_2$, а со шлюзом и между собой могут быть связаны только вершины множества $A_2$.

\subsection{Модель частично целочисленного линейного программирования}

На этапе обследования местности проектировании БШС были отобраны точки, куда возможно расставить БС. Необходимо отметить, что в данной постановке на этапе синтеза топологии, рассматривается более общий случай, когда размещаются не множества имеющихся БС, а выбираются их типы. Так результатом данного этапа будут набор типов БС и их места размещения.

\subsubsection{Построение матрицы смежности}

На каждой вершине $a_i$, $i= \overline{n_1+1,n}$ может разместиться одна из $m$-типов БС. Вместо каждой такой вершины $a_i$ введем $m$ вершин с координатами вершины $\{x_i, y_i \}$, и различными параметрами, соответствующими различным типамБС. Обозначим такую группу вершин, записанных с одинаковыми координатами вместо вершины $a_i$,как $D_i$. Каждой вершине из $D_i$ поставим в соответствие набор параметров только одного типа БС из $S$, т.е. на данной вершине может стоять либо станция приписанного типа либо никакая. Обозначим расширенное множество вершин $A_2$ через $A_2D = \{a_i\}, i = \overline{n_1 + 1,\ n \cdot m}$.



Составим граф $H=\left\{AD,E\right\}$, описывающий сеть для передачи потока информации между вершинами расширенного множества $AD=A_1 \cup A_2D$ и шлюзом $s_0$ в вершине $a_0$.
Матрица смежности $E = \{e_{ij} \}$ графа $H$, где каждое ребро $e_{ij}$ определяет возможность передачи информации между вершинами, строится по следующим правилам. 

\begin{itemize}
    \item $e_{ij} = 1$, если расстояние между $i$-ой вершиной ($a_i \in A_1$) и $j$-ой вершиной ($a_j \in A_2D$) не более радиуса покрытия $r_{ji}$, приписанного этой вершине станции;
    \item $e_{ij} = 1$, если вершины $a_i$ и $a_j$ принадлежат разным множествам $D_i$ и $D_j$ и расстояние между ними не больше минимального из радиусов связи $\min\{R_{ij}, R_{ji}\}$, приписанных данным вершинам станциям;
    \item $e_{i0} = 1$ ($a_i \in A_2D$), если расстояние от вершины до шлюза не больше минимального радиуса связей $\min\{R_{i0}, R_{0i}\}$;
    \item $e_{ij} = 0$, во всех остальных случаях.
\end{itemize}

\subsubsection{Формулировка в виде ЧЦЛП}

С помощью полученного графа потока, опишем ограничения для задачи частично целочисленного линейного программирования (ЧЦЛП).

Введем булевы переменные $z_{ij} = \{0, 1\}$, $, i = \overline{1,n_1}, j = \overline{n_1+1, \ n \cdot m}$, определяющее наличие соединения между объектом в точке $a_i, a_i \in A_1$  и БС, размещенной в точке $a_j, a_j \in A_2D$.


Все объекты, размещенные на вершинах $A_1$, оснащены антеннами для передачи сигнала в беспроводной среде. Каждая объект одновременно может поддерживать соединение только с одной БС. Данной условие можно записать в виде ограничения равенства \cref{eq:part3_only_1_link_from_device}

% Распишем условия для нашей задачи.
% Величина суммарного потока, который выходит с вершины $a_i$ равен весу $\vartheta_i$ \cref{eq:part2_1.5}

\begin{equation}\label{eq:part3_only_1_link_from_device}
    \sum_{a_j \in \Gamma_2^+(a_i)} z_{ij} = 1, \forall a_i, i =\overline{1, n_1},
\end{equation} 
где $\Gamma^+(a_i)$ -- множество вершин на графе $H$, в которые входят дуги, исходящие из вершины $a_i$.

Введем потоковые переменные $x_{ij} \in \mathbb{R}^+$, определяющее количество информации, передаваемой в единицу времени по дуге $e_{ij}$ графа $H$.

Потоки информации объектов с вершин $A_1$ должны поступать на станции. Также на станции может поступать потоки с других БС. Необходимо, чтобы сумма входящих и выходящих потоков для любой $j$-ой вершины множества $A_2D$ был равен нулю \cref{eq:part3_sta_io_flows} 
% \fixme{проверить индексы}
\begin{equation}\label{eq:part3_sta_io_flows} 
    \sum_{a_i \in \Gamma_1^-(a_j)} z_{ij} \cdot \vartheta_i + \sum_{a_i \in \Gamma_2^-(a_j)} x_{ij} -  \sum_{a_i \in \Gamma_2^+(a_j)} x_{ji} =0 ,\forall a_j \in A_2. 
\end{equation} 

% todo: delete this comment
% \begin{equation}\label{eq:part3_sta_io_flows} 
%     \sum_{a_j \in \Gamma_1^-(a_i)} z_{ij} \cdot \vartheta_i + \sum_{a_j \in \Gamma_2^-(a_i)} x_{ji} -  \sum_{a_j \in \Gamma_2^+(a_i)} x_{ij} =0 ,\forall a_i \in A_2. 
% \end{equation} 

Здесь множество $\Gamma_1^-(a_i)$ -- вершины множества $A_1$, из которых выходят дуги, входящие в вершину $a_i$; $\Gamma_2^-(a_i)$ -- вершины множества $A_2D$, из которых выходят дуги, входящие в  вершину $a_i$; $\Gamma_2^+(a_i)$ -- вершины множества $A_2D$, в которые входят дуги, исходящие из вершины  $a_i$.

Через систему БС вся информация от объектов  должна поступить на шлюз $s_0$ \cref{eq:part3_device2gateway_flow} 
\begin{equation}\label{eq:part3_device2gateway_flow}
    \sum_{a_j \in \Gamma_2^-(a_0)} x_{j0} = \sum_{a_i \in A_1} \vartheta_i,
\end{equation}
здесь $\Gamma_2^-(a_0)$ –- подмножество вершин множества $A_2D$, дуги которых входят в шлюз $a_0$.

Введем булевы переменные $y_{ij} = \{0,1\}$ для потока $x_{ij}$, исходящего из вершины $a_i$, $a_i \in A_2D$ в вершину $a_j$, $a_j \in A_2D$. Данная переменная характеризует наличие соединения между вершинами.
% \begin{itemize}
%     \item $y_i = 1$, если станция стоит на месте $a_i$;
%     \item $y_i = 0$, в противном случае.
% \end{itemize}

% Объем информации, поступающей от вершин множества $A_1$ на вершину $a_i \in A_2D$, ограничен мощностью станции $\vartheta_i$ \cref{eq:part2_1.8}
Поток информации $w_{ij}$ между вершинами множества $A_2D$ может передаваться только при наличии соединения $y_{ij}$. Также данный поток ограничен пропускной способностью $\vartheta_i$ БС  \cref{eq:part3_flow_link_sta}
\begin{equation}\label{eq:part3_flow_link_sta}
    \sum_{a_j \in \Gamma_2^-(a_i)} x_{ij} \leqslant y_{ij} \cdot \vartheta_i, \forall a_i \in A_2D.
\end{equation}

Каждая БС может иметь только одно соединение для передачи потока информации в единицу времени. Необходимо обеспечить условие, что в каждом множестве $D_i$ может быть размещено не более одной станции. Оба этих требования можно записать в виде ограничения неравенства \cref{eq:part3_only_1_link_yij}

\begin{equation}\label{eq:part3_only_1_link_yij}
    \sum_{a_j \in \Gamma_2^-(a_i)} y_{ij} \leqslant 1, \forall D_i.
\end{equation}

Целевая функция задачи минимизации стоимости размещения \cref{eq:part3_of_min}

\begin{equation}\label{eq:part3_of_min}
    \sum_{a_i \in A_2D} \sum_{a_j \in \Gamma_2^-(a_i)}c_i \cdot y_{ij} \to min.
\end{equation}

Задача \cref{eq:part3_only_1_link_from_device, eq:part3_sta_io_flows, eq:part3_device2gateway_flow, eq:part3_flow_link_sta, eq:part3_only_1_link_yij, eq:part3_of_min} представляет собой частично целочисленную задачу линейного программирования с $m \cdot |A_2|$ булевыми переменными. 

\subsubsection{Пример решения задачи ЧЦЛП}

Рассмотрим пример для оптимизационной задачи выбора набора размещаемых БС и определения мест их размещения.

Задано множество рассредоточенных объектов $A_1$, $|A_1| = 8$, параметры которых представлены в таблице \cref{tab:part3_mip_devices}, и множество точек возможного размещения БС \cref{tab:part3_mip_station_coordinates} Рисунок \cref{fig:part3_mip_input_data}). 

Задано множество типов БС $S, |S| = 2$ (таблица \cref{tab:part3_mip_station_types}). 

Задана станция специального вида -- шлюз с параметрами, представленными в таблице  \cref{tab:part3_mip_gateway}. Необходимо разместить БС таким образом, чтобы вся информация с объектов могла быть собрана и  передана на шлюз и с учетом бюджетного ограничения. 

\begin{figure}[h!]
    \centering
     \includegraphics[width=.8\textwidth]{mip_input_data.png}
  \caption{Множество $A$.}
  \label{fig:part3_mip_input_data}
\end{figure}



Вместо каждой такой вершины $a_i$ из множества возможных точек размещения БС $A_2$ необходимо ввести $|S| =2 $ вершин с координатами точки $a_i$ и различными параметрами, соответствующими различным типам БС. Полученное расширенное множество $A_2D$ представлено в таблице \cref{tab:part3_mip_station_point}.


\begin{table}[h!]\centering
    \begin{tabular}{|c||c|c|c|c|c|}\hline
        
        Точки размещения & \multirow{2}{*}{Координаты} & \multirow{2}{*}{$\vartheta$}	&\multirow{2}{*}{$P_{tr}$}&	\multirow{2}{*}{$G_{tr}$}& \multirow{2}{*}{$L$}\\

        станций, $A_2$& & & & & \\
        \hline
        \textnumero & м & 1/с & дБм& дБ& дБ\\
        \hline
        $a_1$& (22, 15)& 11 & 10&	1&	0 \\
        $a_2$& (1, 20)& 11 & 10&	1&	0 \\
        $a_3$& (19, 27)& 11 & 10&	1&	0 \\
        $a_4$& (14, 6)& 12 & 10&	1&	0 \\
        $a_5$& (25, 27)& 13 & 10&	1&	0 \\
        $a_6$& (15.2, 10.1)& 15 & 10&	1&	0 \\
        $a_7$& (1, 15)& 17 & 10&	1&	0 \\
        $a_8$& (0, 25)& 19 & 10&	1&	0 \\
    
        \hline
  
  \end{tabular}\caption{Рассредоточенные объекты}\label{tab:part3_mip_devices}
\end{table}


\begin{table}[h!]\centering
    \begin{tabular}{|c||c|}\hline
        
        $A_2$& Координаты\\
        \hline
        $a_9$& (18, 14) \\
        $a_{10}$& (4, 3.5) \\
        $a_{11}$& (5, 20) \\
        $a_{12}$& (10, 14) \\
        $a_{13}$& (25, 20) \\
       
        \hline
  
  \end{tabular}\caption{Точки размещения станций}\label{tab:part3_mip_station_coordinates}
\end{table}



\begin{table}[h!]\centering
    \begin{tabular}{|c||c|c|c|c|c|c|}\hline
        
        S& $\vartheta$	&$P_{tr}$&	$G_{tr}$&	$P_{recv}$ &$L$ & $c$\\
        \hline
        \textnumero & 1/c & дБм&	дБ&	дБм&	дБ & у.е.\\
        \hline
        $s_1$& 1300 & 20&	4&	-79& 1& 70\\
        $s_2$& 1400& 18&	3&	-81& 1 & 80\\

        \hline
  
  \end{tabular}\caption{Типы станций}\label{tab:part3_mip_station_types}
\end{table}

\begin{table}[h]\centering
    \begin{tabular}{|c||c|c|c|c|c|c|}\hline
        
        Шлюз& Координаты & $\vartheta$	&$P_{tr}$&	$G_{tr}$&	$P_{recv}$ &$L$\\
        \hline
        \textnumero & м & 1/c & дБм&	дБ&	дБм& дБ\\
        \hline
        $s_0$&  (8, 32)& $\infty$ & 20&	4&	-79& 1\\
        \hline
  
  \end{tabular}\caption{Параметры шлюза}\label{tab:part3_mip_gateway}
\end{table}


% \fontsize{10pt}{10pt}\selectfont
\begin{table}[h]
    \begin{tabular}{|  c||  c|  c|  c|  c|  c|  c|  c|  c|  c|  c|}
    

    \hline
    \tiny
    Множество $A_2$&\multicolumn{2}{c|}{$a_9$}&\multicolumn{2}{c|}{$a_{10}$}&\multicolumn{2}{c|}{$a_{11}$}& \multicolumn{2}{c|}{$a_{12}$}&\multicolumn{2}{c|}{$a_{13}$} \\
    \hline
    Множество типов станций $S$&$s_1$& $s_2$&$s_1$& $s_2$& $s_1$& $s_2$& $s_1$& $s_2$& $s_1$& $s_2$  \\
    \hline
    Расширенное множество $A_2D$&$a_9$& $a_{10}$&$a_{11}$&$a_{12}$& $a_{13}$& $a_{14}$& $a_{15}$& $a_{16}$& $a_{17}$& $a_{18}$  \\
  
    \hline
    \end{tabular}
    \caption{Расширенное множество $A_2D$}\label{tab:part3_mip_station_point}
\end{table}
\normalsize

Используя уравнения потерь в свободном пространстве и уравнение энергетического баланса необходимо рассчитать радиус телекоммуникационного покрытия и радиус связи станций, чтобы получить граф потока $H$. Запас на замирание сигнала $SOM = 30$ и несущая частота $f=5537$. В таблице \cref{tab:part3_mip_adj_mat} представлена матрица смежности $E$ полученного графа $H$.


\begin{table}[hbt!]
    \tiny
    \begin{tabular}{| c|| c| c c c c c c c c| c c c c c c c c c c|}

    \hline

    &$a_0$ &$a_1$ &$a_2$ &$a_3$ &$a_4$ &$a_5$ &$a_6$ &$a_7$ &$a_8$ &$a_9$& $a_{10}$&$a_{11}$&$a_{12}$& $a_{13}$& $a_{14}$& $a_{15}$& $a_{16}$& $a_{17}$& $a_{18}$  \\
    \hline \hline
    $a_0$ & 0 &0 &0 &0 &0 &0 &0 &0 &0 &0 &0 &0 &0 &0 &0 &0 &0 &0 &0  \\
    \hline
    $a_1$ & 0 &0 &0 &0 &0 &0 &0 &0 &0 &1 &1 &0 &0 &0 &0 &0 &0 &1 &1  \\
    $a_2$ & 0 &0 &0 &0 &0 &0 &0 &0 &0 &0 &0 &0 &0 &1 &1 &0 &0 &0 &0  \\
    $a_3$ & 0 &0 &0 &0 &0 &0 &0 &0 &0 &0 &0 &0 &0 &0 &0 &0 &0 &1 &1  \\
    $a_4$ & 0 &0 &0 &0 &0 &0 &0 &0 &0 &1 &1 &1 &1 &0 &0 &1 &1 &0 &0  \\
    $a_5$ & 0 &0 &0 &0 &0 &0 &0 &0 &0 &0 &0 &0 &0 &0 &0 &0 &0 &1 &1  \\
    $a_6$ & 0 &0 &0 &0 &0 &0 &0 &0 &0 &1 &1 &0 &0 &0 &0 &1 &1 &0 &0  \\
    $a_7$ & 0 &0 &0 &0 &0 &0 &0 &0 &0 &0 &0 &1 &1 &1 &1 &1 &1 &0 &0  \\
    $a_8$ & 0 &0 &0 &0 &0 &0 &0 &0 &0 &0 &0 &0 &0 &1 &1 &0 &0 &0 &0  \\
    \hline
    $a_9$ & 1 &0 &0 &0 &0 &0 &0 &0 &0 &0 &1 &1 &1 &1 &1 &1 &1 &1 &1  \\
    $a_{10}$ & 1 &0 &0 &0 &0 &0 &0 &0 &0 &1 &0 &1 &1 &1 &1 &1 &1 &1 &1  \\
    $a_{11}$ & 1 &0 &0 &0 &0 &0 &0 &0 &0 &1 &1 &0 &1 &1 &1 &1 &1 &1 &1  \\
    $a_{12}$ & 1 &0 &0 &0 &0 &0 &0 &0 &0 &1 &1 &1 &0 &1 &1 &1 &1 &1 &1  \\
    $a_{13}$ & 1 &0 &0 &0 &0 &0 &0 &0 &0 &1 &1 &1 &1 &0 &1 &1 &1 &1 &1  \\
    $a_{14}$ & 1 &0 &0 &0 &0 &0 &0 &0 &0 &1 &1 &1 &1 &1 &0 &1 &1 &1 &1  \\
    $a_{15}$ & 1 &0 &0 &0 &0 &0 &0 &0 &0 &1 &1 &1 &1 &1 &1 &0 &1 &1 &1  \\
    $a_{16}$ & 1 &0 &0 &0 &0 &0 &0 &0 &0 &1 &1 &1 &1 &1 &1 &1 &0 &1 &1  \\
    $a_{17}$ & 1 &0 &0 &0 &0 &0 &0 &0 &0 &1 &1 &1 &1 &1 &1 &1 &1 &0 &1  \\
    $a_{18}$ & 1 &0 &0 &0 &0 &0 &0 &0 &0 &1 &1 &1 &1 &1 &1 &1 &1 &1 &0  \\
    \hline
    \end{tabular}\caption{Матрица смежности}\label{tab:part3_mip_adj_mat}
\end{table}

Решением задачи ЧЦЛП \cref{eq:part3_only_1_link_from_device, eq:part3_sta_io_flows, eq:part3_device2gateway_flow, eq:part3_flow_link_sta, eq:part3_only_1_link_yij, eq:part3_of_min} является размещение, представленное на рисунке \cref{fig:part3_mip_solution}. Были размещены базовые станции типы $s_3$ в точках $a_11$, $a_12$ и $a_13$. Жирным цветом представлены возможные соединения базовых станций, стрелками граф потока информации от объектов до шлюза. Математическая модель была подготовлена на языке Python, задача ЧЦЛП решалась в коммерческом продукте Gurobi Optimization.  Код программы представлен в \url{https://github.com/m0000Amir/BSP-on-plane}.


% \FloatBarrier
\section{Выводы по главе 3}

В работе рассмотрены задачи размещения базовых станций при проектировании беспроводных широкополосных сетей связи для покрытия множества рассредоточенных объектов. 
\begin{itemize}
    \item Предложена формулировка задачи в виде математической модели линейного программирования при заданных мест размещения станция для проверки условия допустимой передачи потока от множества объектов до точки корневого узла сети;
    \item Предложена математическая модель экстремальной задачи в виде частично целочисленного линейного программирования оптимального размещения станций из имеющегося набора типов станций на избыточном множестве возможных мест размещения;
    \item Предложены алгоритмы построения графа информационных потоков, позволяющий формализовать задачи в виде соответствующих моделей математического программирования. 
\end{itemize}

Результаты исследования по данной главе были опубликованы в работах \cite{MukhtarovPershinGUBKIN2018_RSCI, 
MukhtarovPershinGUBKIN2019_RSCI,MukhtarovPershinVSPU2019_RSCI, MukhtarovPershinMLSD2019works_RSCI, MukhtarovPershinMLSD2019materials_RSCI,}. 

\begin{figure}[hbt!]
    \centering
     \includegraphics[width=.8\textwidth]{mip_solution.png}
  \caption{Решение задачи ЧЦЛП.}
  \label{fig:part3_mip_solution}
\end{figure}


% В Приложении \cref{app:milp_place_solution} приведены результаты вычислительного эксперимента. 



\FloatBarrier
