
{\actuality} 
В настоящее время тенденция бурного развития информационных технологий во всех сферах деятельности человека оказывает весомое на развитие нефтегазовый сектора страны. Современные компании, представляющие собой сложную многоуровневую производственно-технологической систему в силу своего устойчивого развития требуют постоянного движения в направлении развития технологий.  Нефтегазовая отрасль России является ключевым сектором топливно-энергетическим комплекса страны. Особенностью данной отрасли является масштабы объектов управления, наличие больших объемов информации, высокие требования к безопасности и надежности. Сегодня наблюдается этапом бурного развития «цифровизации». Лидеры крупнейших международных нефтегазовых компаний имеют подразделения, задачами которых является разработка и реализация в дальнейшем принципов интеллектуального месторождения: «Умные месторождения» («Smart Fields») в компании Shell, «Месторождение будущего» («Field of the Future») в компании BP и «iFields» в компании Chevron и др. Данное развитие нефтегазового комплекса предусматривает переход к малолюдным системам управления добычи, транспортировки и переработки сырья. Основными информационными технологиями являются: большие данные (англ. Big Data), искусственные нейронные сети (англ. Artificial Neural Network – ANN), системы распределенного реестра (англ. Blockchain), промышленные интернет вещей (англ. Industrial internet of things – IIoT), технологии виртуальной и дополненной реальности (англ. Virtual Reality – VR), мониторинг распределенных объектов беспилотными летательными аппаратами БПЛА ( англ. Unmanned Aerial Vehicle – UAV). Большой объем передачи информации  привел к еще одной из наиболее интересных тенденций цифрового развития – внедрения беспроводных технологий. Современные месторождения сегодня, помимо данных первичного сбора и обработки информации технологических параметров основных производственных объектов содержат также колоссальный объем  информации мультимедийного трафика. Сюда входят данные БПЛА по обнаружению утечек и  разрушения трубопроводов; камер видеонаблюдений; а также большой поток данных цифровых двойников, аналитики и т.д. В совокупности со всеми вышеизложенными перспективными направлениями беспроводные технологии являются неотъемлемой частью «цифровизации» месторождения.

Отсюда возникает научно - техническая проблема организации распределенной сети связи, соответствующая реальным требованием современного производства.




% Обзор, введение в тему, обозначение места данной работы в
% мировых исследованиях и~т.\:п., можно использовать ссылки на~другие
% работы~\autocite{Gosele1999161,Lermontov}
% (если их~нет, то~в~автореферате
% автоматически пропадёт раздел <<Список литературы>>). Внимание! Ссылки
% на~другие работы в~разделе общей характеристики работы можно
% использовать только при использовании \verb!biblatex! (из-за технических
% ограничений \verb!bibtex8!. Это связано с тем, что одна
% и~та~же~характеристика используются и~в~тексте диссертации, и в
% автореферате. В~последнем, согласно ГОСТ, должен присутствовать список
% работ автора по~теме диссертации, а~\verb!bibtex8! не~умеет выводить в~одном
% файле два списка литературы).
% При использовании \verb!biblatex! возможно использование исключительно
% в~автореферате подстрочных ссылок
% для других работ командой \verb!\autocite!, а~также цитирование
% собственных работ командой \verb!\cite!. Для этого в~файле
% \verb!common/setup.tex! необходимо присвоить положительное значение
% счётчику \verb!\setcounter{usefootcite}{1}!.

% Для генерации содержимого титульного листа автореферата, диссертации
% и~презентации используются данные из файла \verb!common/data.tex!. Если,
% например, вы меняете название диссертации, то оно автоматически
% появится в~итоговых файлах после очередного запуска \LaTeX. Согласно
% ГОСТ 7.0.11-2011 <<5.1.1 Титульный лист является первой страницей
% диссертации, служит источником информации, необходимой для обработки и
% поиска документа>>. Наличие логотипа организации на~титульном листе
% упрощает обработку и~поиск, для этого разметите логотип вашей
% организации в папке images в~формате PDF (лучше найти его в векторном
% варианте, чтобы он хорошо смотрелся при печати) под именем
% \verb!logo.pdf!. Настроить размер изображения с логотипом можно
% в~соответствующих местах файлов \verb!title.tex!  отдельно для
% диссертации и автореферата. Если вам логотип не~нужен, то просто
% удалите файл с~логотипом.

% \ifsynopsis
% Этот абзац появляется только в~автореферате.
% Для формирования блоков, которые будут обрабатываться только в~автореферате,
% заведена проверка условия \verb!\!\verb!ifsynopsis!.
% Значение условия задаётся в~основном файле документа (\verb!synopsis.tex! для
% автореферата).
% \else
% Этот абзац появляется только в~диссертации.
% Через проверку условия \verb!\!\verb!ifsynopsis!, задаваемого в~основном файле
% документа (\verb!dissertation.tex! для диссертации), можно сделать новую
% команду, обеспечивающую появление цитаты в~диссертации, но~не~в~автореферате.
% \fi

% {\progress}
% Этот раздел должен быть отдельным структурным элементом по
% ГОСТ, но он, как правило, включается в описание актуальности
% темы. Нужен он отдельным структурынм элемементом или нет ---
% смотрите другие диссертации вашего совета, скорее всего не нужен.


{\progress}\fixme{Степень разработанности.}

{\objectresearch} в данной работе являются беспроводные широкополосные сети.

{\subjectresearch} является синтез топологической структуры беспроводной широкополосной сети.


{\aim} состоит в разработке моделей и методов задачи оптимального размещения базовых станций беспроводной широкополосной сети.

Для~достижения поставленной цели необходимо было решить следующие {\tasks}:
\begin{enumerate}[beginpenalty=10000] % https://tex.stackexchange.com/a/476052/104425
  \item анализ состояния, основных проблем и перспектив развития современных инфраструктур систем коммуникаций по беспроводным каналам на месторождениях;
  \item разработка моделей задач размещения базовых станций в рамках комплексного проектирования сетей коммуникаций мониторинга объектов нефтегазовых месторождений;
  \item разработка моделей оценки характеристик производительности сетей связи;
\end{enumerate}


{\novelty} результатов исследования заключается в следующем:
\begin{enumerate}[beginpenalty=10000] % https://tex.stackexchange.com/a/476052/104425
  \item разработаны модели задачи размещения базовых станций на плоскости и для частного случая с линейной топологией;
  \item разработаны модели имитационного моделирования для оценки характеристик производительности сети;
  \item разработаны модели прогнозирования оценок характеристик производительности с помощью методов машинного обучения.
\end{enumerate}

{\influence} \ldots

{\methods} \ldots

{\defpositions} \fixme{
\begin{enumerate}[beginpenalty=10000] % https://tex.stackexchange.com/a/476052/104425
  \item Первое положение
  \item Второе положение
  \item Третье положение
  \item Четвертое положение
\end{enumerate}
}
В папке Documents можно ознакомиться с решением совета из Томского~ГУ (в~файле \verb+Def_positions.pdf+), где обоснованно даются рекомендации по~формулировкам защищаемых положений.

{\reliability} \fixme{полученных результатов обеспечивается \ldots \ Результаты находятся в соответствии с результатами, полученными другими авторами.}


{\probation}
Основные положения и результаты исследования представлены и обсуждены на научных конференциях «Губкинский университет в решении вопросов нефтегазовой отрасли России» (Москва, 17-21 сентября 2018); «13-е Всероссийское совещание по проблемам управления» (Москва, 17-20 июня 2019); «International Conference on Distributed Computer and Communication Networks: Control, Computation, Communications» (Москва, 22-27 сентября 2019), «Губкинский университет в решении вопросов нефтегазовой отрасли России» (Москва, 24-26 сентября 2019); «Управление развитием крупномасштабных систем» (Москва, 1-3 октября 2019); «Information and Telecommunication Technologies and Mathematical Modeling of High-Tech Systems» (Москва, 13-17 апреля 2020); «Computer-aided technologies in applied mathematics» (Томск, сентябрь 2020); «International Conference on Distributed Computer and Communication Networks: Control, Computation, Communications» (Москва, 14-18 сентября 2020); «Information and Telecommunication Technologies and Mathematical Modeling of High-Tech Systems» (Москва, 19-23 апреля 2021);


{\contribution} Все основные научные положения диссертационного исследования разработаны автором лично.

\ifnumequal{\value{bibliosel}}{0}
{%%% Встроенная реализация с загрузкой файла через движок bibtex8. (При желании, внутри можно использовать обычные ссылки, наподобие `\cite{vakbib1,vakbib2}`).
    {\publications} Основные результаты по теме диссертации изложены
    в~XX~печатных изданиях,
    X из которых изданы в журналах, рекомендованных ВАК,
    X "--- в сборниках трудов конференций.
}%
{%%% Реализация пакетом biblatex через движок biber
    \begin{refsection}[bl-author, bl-registered]
        % Это refsection=1.
        % Процитированные здесь работы:
        %  * подсчитываются, для автоматического составления фразы "Основные результаты ..."
        %  * попадают в авторскую библиографию, при usefootcite==0 и стиле `\insertbiblioauthor` или `\insertbiblioauthorgrouped`
        %  * нумеруются там в зависимости от порядка команд `\printbibliography` в этом разделе.
        %  * при использовании `\insertbiblioauthorgrouped`, порядок команд `\printbibliography` в нём должен быть тем же (см. biblio/biblatex.tex)
        %
        % Невидимый библиографический список для подсчёта количества публикаций:
        \printbibliography[heading=nobibheading, section=1, env=countauthorvak,          keyword=biblioauthorvak]%
        \printbibliography[heading=nobibheading, section=1, env=countauthorwos,          keyword=biblioauthorwos]%
        \printbibliography[heading=nobibheading, section=1, env=countauthorscopus,       keyword=biblioauthorscopus]%
        \printbibliography[heading=nobibheading, section=1, env=countauthorconf,         keyword=biblioauthorconf]%
        \printbibliography[heading=nobibheading, section=1, env=countauthorother,        keyword=biblioauthorother]%
        \printbibliography[heading=nobibheading, section=1, env=countregistered,         keyword=biblioregistered]%
        \printbibliography[heading=nobibheading, section=1, env=countauthorpatent,       keyword=biblioauthorpatent]%
        \printbibliography[heading=nobibheading, section=1, env=countauthorprogram,      keyword=biblioauthorprogram]%
        \printbibliography[heading=nobibheading, section=1, env=countauthor,             keyword=biblioauthor]%
        \printbibliography[heading=nobibheading, section=1, env=countauthorvakscopuswos, filter=vakscopuswos]%
        \printbibliography[heading=nobibheading, section=1, env=countauthorscopuswos,    filter=scopuswos]%
        %
        \nocite{*}%
        %
        {\publications} Основные результаты по теме диссертации изложены в~\arabic{citeauthor}~печатных изданиях,
        \arabic{citeauthorvak} из которых изданы в журналах, рекомендованных ВАК\sloppy%
        \ifnum \value{citeauthorscopuswos}>0%
            , \arabic{citeauthorscopuswos} "--- в~периодических научных журналах, индексируемых Web of~Science и Scopus\sloppy%
        \fi%
        \ifnum \value{citeauthorconf}>0%
            , \arabic{citeauthorconf} "--- в~сборниках трудов конференции.
        \else%
            .
        \fi%
        \ifnum \value{citeregistered}=1%
            \ifnum \value{citeauthorpatent}=1%
                Зарегистрирован \arabic{citeauthorpatent} патент.
            \fi%
            \ifnum \value{citeauthorprogram}=1%
                Зарегистрирована \arabic{citeauthorprogram} программа для ЭВМ.
            \fi%
        \fi%
        \ifnum \value{citeregistered}>1%
            Зарегистрированы\ %
            \ifnum \value{citeauthorpatent}>0%
            \formbytotal{citeauthorpatent}{патент}{}{а}{}\sloppy%
            \ifnum \value{citeauthorprogram}=0 . \else \ и~\fi%
            \fi%
            \ifnum \value{citeauthorprogram}>0%
            \formbytotal{citeauthorprogram}{программ}{а}{ы}{} для ЭВМ.
            \fi%
        \fi%
        % К публикациям, в которых излагаются основные научные результаты диссертации на соискание учёной
        % степени, в рецензируемых изданиях приравниваются патенты на изобретения, патенты (свидетельства) на
        % полезную модель, патенты на промышленный образец, патенты на селекционные достижения, свидетельства
        % на программу для электронных вычислительных машин, базу данных, топологию интегральных микросхем,
        % зарегистрированные в установленном порядке.(в ред. Постановления Правительства РФ от 21.04.2016 N 335)
    \end{refsection}%
    \begin{refsection}[bl-author, bl-registered]
        % Это refsection=2.
        % Процитированные здесь работы:
        %  * попадают в авторскую библиографию, при usefootcite==0 и стиле `\insertbiblioauthorimportant`.
        %  * ни на что не влияют в противном случае
        \nocite{vakbib2}%vak
        \nocite{patbib1}%patent
        \nocite{progbib1}%program
        \nocite{bib1}%other
        \nocite{confbib1}%conf
    \end{refsection}%
        %
        % Всё, что вне этих двух refsection, это refsection=0,
        %  * для диссертации - это нормальные ссылки, попадающие в обычную библиографию
        %  * для автореферата:
        %     * при usefootcite==0, ссылка корректно сработает только для источника из `external.bib`. Для своих работ --- напечатает "[0]" (и даже Warning не вылезет).
        %     * при usefootcite==1, ссылка сработает нормально. В авторской библиографии будут только процитированные в refsection=0 работы.
}

При использовании пакета \verb!biblatex! будут подсчитаны все работы, добавленные
в файл \verb!biblio/author.bib!. Для правильного подсчёта работ в~различных
системах цитирования требуется использовать поля:
\begin{itemize}
        \item \texttt{authorvak} если публикация индексирована ВАК,
        \item \texttt{authorscopus} если публикация индексирована Scopus,
        \item \texttt{authorwos} если публикация индексирована Web of Science,
        \item \texttt{authorconf} для докладов конференций,
        \item \texttt{authorpatent} для патентов,
        \item \texttt{authorprogram} для зарегистрированных программ для ЭВМ,
        \item \texttt{authorother} для других публикаций.
\end{itemize}
Для подсчёта используются счётчики:
\begin{itemize}
        \item \texttt{citeauthorvak} для работ, индексируемых ВАК,
        \item \texttt{citeauthorscopus} для работ, индексируемых Scopus,
        \item \texttt{citeauthorwos} для работ, индексируемых Web of Science,
        \item \texttt{citeauthorvakscopuswos} для работ, индексируемых одной из трёх баз,
        \item \texttt{citeauthorscopuswos} для работ, индексируемых Scopus или Web of~Science,
        \item \texttt{citeauthorconf} для докладов на конференциях,
        \item \texttt{citeauthorother} для остальных работ,
        \item \texttt{citeauthorpatent} для патентов,
        \item \texttt{citeauthorprogram} для зарегистрированных программ для ЭВМ,
        \item \texttt{citeauthor} для суммарного количества работ.
\end{itemize}
% Счётчик \texttt{citeexternal} используется для подсчёта процитированных публикаций;
% \texttt{citeregistered} "--- для подсчёта суммарного количества патентов и программ для ЭВМ.

Для добавления в список публикаций автора работ, которые не были процитированы в
автореферате, требуется их~перечислить с использованием команды \verb!\nocite! в
\verb!Synopsis/content.tex!.
