{\actuality} 
% \fixme{Крупные международные нефтегазовые компании имеют подразделения, задачами которых является разработка и реализация в дальнейшем принципов интеллектуального месторождения: «Умные месторождения» («Smart Fields») в компании Shell, «Месторождение будущего» («Field of the Future») в компании BP и «iFields» в компании Chevron и др \cite{Tcharo2018}.   информации привела к развитию беспроводных технологий. Беспроводные сенсорные сети широко используются на газовых месторождениях \cite{Krasnov2016}. 
% Сюда входят данные по обнаружению утечек и разрушения трубопроводов; информация с камер видеонаблюдения, аналитики и т.д.}

В настоящее время тенденция бурного развития информационных технологий во всех сферах деятельности человека оказывает весомое влияние на нефтегазовый сектор страны. Современные компании, представляющие собой сложные многоуровневые производственные системы, для своего устойчивого развития требуют постоянного развития и совершенствования передовых технологий.  Сегодня наблюдается  бурное развитие процесса «цифровизации» нефтегазовой отрасли. Крупные международные нефтегазовые компании имеют подразделения, задачами которых является разработка и реализация принципов интеллектуального месторождения \cite{Tcharo2018} на промысле, организация безопасности на технологических объектах, развитие концепции  перехода к малолюдным системам управления добычей, транспортировкой и переработкой сырья. Уже сейчас основными современными информационными технологиями, встречающиеся в отрасли являются: большие данные (англ. Big Data), искусственные нейронные сети (англ. Artificial Neural Network – ANN), системы распределенного реестра (англ. Blockchain), промышленные интернет вещей (англ. Industrial internet of things – IIoT), технологии виртуальной и дополненной реальности (англ. Virtual Reality – VR), мониторинг распределенных объектов беспилотными летательными аппаратами БПЛА (англ. Unmanned Aerial Vehicle – UAV). 

В совокупности данные технологии создают необходимость эффективной передачи больших объемов высокоскоростного трафика. Информационные  системы современных месторождений сегодня помимо данных первичного сбора и обработки информации технологических параметров основных производственных объектов содержат также колоссальный объем информации высокоскоростного мультимедийного трафика. Одним из путей решения является внедрение беспроводных технологий \cite{Eremin2020, Dmitrievskiy2020}.

В настоящее время беспроводные технологии являются неотъемлемой частью процесса «цифровизации» месторождения. Активное использование беспроводных сетей основывается на ряде их преимуществ по сравнению с кабельными сетями:
\begin{itemize}
    \item возможность получения информации с любой точки контролируемой территории;
    \item быстрый ввод в эксплуатацию по системе подключение типа Plug-\&-Play;
    \item сокращение капитальных затрат на создание сети; 
    \item уменьшение затрат на эксплуатацию;
    \item высокая гибкость, мобильность, масштабируемость;
    \item упрощенные требования к обслуживанию оборудования.
\end{itemize}


В рамках этого процесса возникает актуальная научно - техническая проблема повышения качества проектирования беспроводной сети связи, осуществляющей сбор и передачу информации в центр  управления с множества контролируемых объектов на некоторой территории и контроль самой территории.   

% В совокупности со всеми вышеизложенными перспективными направлениями беспроводные технологии являются неотъемлемой частью «цифровизации» месторождения. Отсюда возникает научно - техническая проблема организации распределенной беспроводной сети связи, соответствующая реальным требованием современного производства.

% Большой объем передачи информации  привел к еще одной из наиболее интересных тенденций цифрового развития – внедрения беспроводных технологий. Современные месторождения сегодня, помимо данных первичного сбора и обработки информации технологических параметров основных производственных объектов содержат также колоссальный объем  информации мультимедийного трафика. Сюда входят данные БПЛА по обнаружению утечек и  разрушения трубопроводов; камер видеонаблюдений; а также большой поток данных цифровых двойников, аналитики и т.д. В совокупности со всеми вышеизложенными перспективными направлениями беспроводные технологии являются неотъемлемой частью «цифровизации» месторождения.



Процесс проектирования современной  беспроводных сетей связи состоит из последовательного решения взаимосвязанных задач:

\begin{itemize}
    \item выбор типов технических средств и протоколов;
    \item выбор топологической структуры сети;
    \item анализ и оптимизация пропускной способности каналов связи, маршрутизация информационных потоков и др.
\end{itemize}

Данная работа посвящена проблеме оптимизации топологической структуры беспроводной широкополосной сети (БШС).
% Задача синтеза топологии при комплексном проектировании БШС является основной проблемой исследования в данной работе.


{\progress} Создание и постоянное развитие современной инфраструктуры передачи данных является одной из основных задач современного производства. Бурное развитие беспроводных сетей во всех областях деятельности человека обосновывает целесообразность их использования на нефтегазовых месторождениях. В настоящее время в России и за рубежом исследованию беспроводных сетей связи посвящен ряд работ, рассматривающие сети для контроля гражданских  и промышленных объектов. Примерами таких объектов является жилые районы города, протяженные автомагистрали, железные дороги, трубопроводы и др. В частности, при исследовании проблемы синтеза топологии сети автор опирался на труды таких отечественных ученых как: В.М. Вишневский, О.Ю. Першин, А.А. Ларионов, и другие.
Наряду с отчественными работами диссертант обращался к трудам зарубежных авторов: Е.С. Кавальканте, Х. Лиу, А.Б. Рейз, Д.Ли, Д.П. Хейман, С. Шен, Д. Бендель, У. М. Амин, Б. Брахим, Х.Э. Кызылёз и другие. 

В работах указанных авторов рассматриваются задачи оптимального синтеза топологии сети и исследуются вопросы анализа сетей, в том числе рассматриваются оценки характеристик сетей с помощью стохастических моделей сетей массового обслуживания. 
% Таким образом актуальность задачи синтеза топологии сети в составе комплексного проектирования беспроводных сетей предопределили и положили начало целям и задачам данного диссертационного исследования. 

Исследования доведены до разработки алгоритмов и программ, применимых для решения практических задач. Проведены численные эксперименты, позволяющие оценить характеристики вычислительных методов.


{\objectresearch} являются БШС специальных типов, широко представленных на практике: БШС для контроля линейных траекторий и БШС с ячеистой топологией (mesh) для контроля объектов, рассредоточенных на некоторой территории.

{\subjectresearch} является синтез топологической структуры беспроводной широкополосной сети.

{\aim} состоит в разработке моделей и методов оптимального размещения базовых станций для БШС указанных типов, определяющего топологию таких сетей.

Для достижения поставленной цели были решены следующие {\tasks}:
\begin{enumerate}[beginpenalty=10000] % https://tex.stackexchange.com/a/476052/104425
  \item сделан анализ современного состояния и перспектив развития БШС для  обоснования  актуальности исследований в области оптимизации их топологии; 
  \item проанализирована методика проектирования современных БШС с целью определения технологических требований к решению задачи синтеза оптимальной топологии сета, а также расчета технологических параметров БШС, необходимых для решения задач размещения базовых станций;
  \item построены математические модели для задачи оптимального размещения базовых станций БШС с линейной топологией, разработан алгоритм типа метода ветвей и границ (МВиГ) для решения указанной задачи, предложена итерационная процедура нахождения последовательности лучших решений в размещении базовых станций в рамках комплексного проектирования сети;
  \item разработаны математические модели для проектирования и анализа БШС с ячеистой топологией.
  %   \item разработаны методы оценки характеристик производительности сети с помощью данных имитационного моделирования и методов машинного обучения. 

\end{enumerate}



{\novelty} результатов исследования заключается в следующем:
\begin{enumerate}[beginpenalty=10000] % https://tex.stackexchange.com/a/476052/104425
  \item построены математические модели в виде экстремальной комбинаторной задачи и задачи целлочисленного линейного программирование (ЦЛП) для оптимального размещения базовых станций при проектировании БШС с линейной топологией;  
  \item разработан специальный алгоритм МВиГ для решения сформулированной экстремальной комбинаторной задачи.;
  \item разработана итерационная процедура нахождения последовательности лучших решений для задачи размещения базовых станций в рамках комплексного проектирования БШС с линейной топологией;
  \item разработаны математические модели для задач проектирования БШС с ячеистой топологией.
%   \item  \fixme{разработаны алгоритмы для анализа выполнения технологических требований и оптимального размещения базовых станций для  БШС с ячеистой топологией};
%   \item работаны имитационные модели многофазной сети массового обслуживания с зависимым временем обслуживания;
%   \item разработаны модели прогнозирования оценок характеристик производительности сети с помощью методов машинного обучения. 
\end{enumerate}

{\influence}. Разработанные модели и методы позволяют повысить качество и эффективность проектирования БШС для распространенных типов таких сетей.

% {\elaboration}. Исследования доведены до разработки алгоритмов и программ, применимых для решения практических задач. Проведены численные эксперименты, позволяющие оценить характеристики вычислительных методов.

{\methods} В работе использованы теория и методы оптимизации на конечных множествах и теории массового обслуживания

{\defpositions}
% \begin{enumerate}[beginpenalty=10000] % https://tex.stackexchange.com/a/476052/104425
%   \item математические модели линейной задачи и алгоритм решения линейной комбинаторной задачи методом ветвей и границ;
%   \item итерационная процедура построения последовательности топологий; 
%   \item математическая модель задачи покрытия множества рассредоточенных объектов; 
%   \item имитационная модель сети массового обслуживания с зависимым распределением времени обслуживания;
%   \item регрессионные модели характеристики производительности сети, полученные с помощью методов машинного обучения.
% \end{enumerate}

\begin{enumerate}[beginpenalty=10000] % https://tex.stackexchange.com/a/476052/104425
    \item математические модели в виде экстремальной комбинаторной задачи и
    задачи ЦЛП для оптимального размещения базовых станций при
    проектировании БШС с линейной топологией;
    \item специальный алгоритм МВ и Г для решения сформулированной
    экстремальной комбинаторной задачи;
    \item итерационная процедура нахождения последовательности лучших
    решений для задачи размещения базовых станций в рамках комплексного
    проектирования БШС с линейной топологией;
    \item математические модели для задач проектирования БШС с ячеистой
    топологией;
    % \item модели прогнозирования оценок характеристик производительности сети с
    % помощью методов машинного обучения для многофазной сети массового обслуживания.
  \end{enumerate}

% В папке Documents можно ознакомиться с решением совета из Томского~ГУ (в~файле \verb+Def_positions.pdf+), где обоснованно даются рекомендации по~формулировкам защищаемых положений.

% {\reliability} \fixme{полученных результатов обеспечивается \ldots \ Результаты находятся в соответствии с результатами, полученными другими авторами.}


{\probation}
Основные положения и результаты исследования представлены и обсуждены на научных конференциях «Губкинский университет в решении вопросов нефтегазовой отрасли России» (Москва, 17-21 сентября 2018); «13-е Всероссийское совещание по проблемам управления» (Москва, 17-20 июня 2019); «International Conference on Distributed Computer and Communication Networks: Control, Computation, Communications» (Москва, 22-27 сентября 2019), «Губкинский университет в решении вопросов нефтегазовой отрасли России» (Москва, 24-26 сентября 2019); «Conference Management of Large-Scale System Development» (Москва, 1-3 октября 2019); «Information and Telecommunication Technologies and Mathematical Modeling of High-Tech Systems» (Москва, 13-17 апреля 2020); «Computer-aided technologies in applied mathematics» (Томск, сентябрь 2020); «International Conference on Distributed Computer and Communication Networks: Control, Computation, Communications» (Москва, 14-18 сентября 2020); «Information and Telecommunication Technologies and Mathematical Modeling of High-Tech Systems» (Москва, 19-23 апреля 2021);


{\contribution} Основные результаты диссертации, выносимые на защиту получены автором самостоятельно.

\ifnumequal{\value{bibliosel}}{1}
{%%% Встроенная реализация с загрузкой файла через движок bibtex8. (При желании, внутри можно использовать обычные ссылки, наподобие `\cite{vakbib1,vakbib2}`).
    {\publications} Основные результаты по теме диссертации изложены в 13 печатных изданиях, 1 из которых издана в журнале, рекомендованных ВАК, 2 — в периодических научных журналах, индексируемых Web of Science и Scopus, 10 — в сборниках трудов конференции. 
}%
{%%% Реализация пакетом biblatex через движок biber
    \begin{refsection}[bl-author, bl-registered]
        % Это refsection=1.
        % Процитированные здесь работы:
        %  * подсчитываются, для автоматического составления фразы "Основные результаты ..."
        %  * попадают в авторскую библиографию, при usefootcite==0 и стиле `\insertbiblioauthor` или `\insertbiblioauthorgrouped`
        %  * нумеруются там в зависимости от порядка команд `\printbibliography` в этом разделе.
        %  * при использовании `\insertbiblioauthorgrouped`, порядок команд `\printbibliography` в нём должен быть тем же (см. biblio/biblatex.tex)
        %
        % Невидимый библиографический список для подсчёта количества публикаций:
        \printbibliography[heading=nobibheading, section=1, env=countauthorvak,          keyword=biblioauthorvak]%
        \printbibliography[heading=nobibheading, section=1, env=countauthorwos,          keyword=biblioauthorwos]%
        \printbibliography[heading=nobibheading, section=1, env=countauthorscopus,       keyword=biblioauthorscopus]%
        \printbibliography[heading=nobibheading, section=1, env=countauthorconf,         keyword=biblioauthorconf]%
        \printbibliography[heading=nobibheading, section=1, env=countauthorother,        keyword=biblioauthorother]%
        \printbibliography[heading=nobibheading, section=1, env=countregistered,         keyword=biblioregistered]%
        \printbibliography[heading=nobibheading, section=1, env=countauthorpatent,       keyword=biblioauthorpatent]%
        \printbibliography[heading=nobibheading, section=1, env=countauthorprogram,      keyword=biblioauthorprogram]%
        \printbibliography[heading=nobibheading, section=1, env=countauthor,             keyword=biblioauthor]%
        \printbibliography[heading=nobibheading, section=1, env=countauthorvakscopuswos, filter=vakscopuswos]%
        \printbibliography[heading=nobibheading, section=1, env=countauthorscopuswos,    filter=scopuswos]%
        %
        \nocite{*}%
        %
        {\publications} Основные результаты по теме диссертации изложены в~\arabic{citeauthor}~печатных изданиях,
        \arabic{citeauthorvak} из которых изданы в журналах, рекомендованных ВАК\sloppy%
        \ifnum \value{citeauthorscopuswos}>0%
            , \arabic{citeauthorscopuswos} "--- в~периодических научных журналах, индексируемых Web of~Science и Scopus\sloppy%
        \fi%
        \ifnum \value{citeauthorconf}>0%
            , \arabic{citeauthorconf} "--- в~сборниках трудов конференции.
        \else%
            .
        \fi%
        \ifnum \value{citeregistered}=1%
            \ifnum \value{citeauthorpatent}=1%
                Зарегистрирован \arabic{citeauthorpatent} патент.
            \fi%
            \ifnum \value{citeauthorprogram}=1%
                Зарегистрирована \arabic{citeauthorprogram} программа для ЭВМ.
            \fi%
        \fi%
        \ifnum \value{citeregistered}>1%
            Зарегистрированы\ %
            \ifnum \value{citeauthorpatent}>0%
            \formbytotal{citeauthorpatent}{патент}{}{а}{}\sloppy%
            \ifnum \value{citeauthorprogram}=0 . \else \ и~\fi%
            \fi%
            \ifnum \value{citeauthorprogram}>0%
            \formbytotal{citeauthorprogram}{программ}{а}{ы}{} для ЭВМ.
            \fi%
        \fi%
        % К публикациям, в которых излагаются основные научные результаты диссертации на соискание учёной
        % степени, в рецензируемых изданиях приравниваются патенты на изобретения, патенты (свидетельства) на
        % полезную модель, патенты на промышленный образец, патенты на селекционные достижения, свидетельства
        % на программу для электронных вычислительных машин, базу данных, топологию интегральных микросхем,
        % зарегистрированные в установленном порядке.(в ред. Постановления Правительства РФ от 21.04.2016 N 335)
    \end{refsection}%
    \begin{refsection}[bl-author, bl-registered]
        % Это refsection=2.
        % Процитированные здесь работы:
        %  * попадают в авторскую библиографию, при usefootcite==0 и стиле `\insertbiblioauthorimportant`.
        %  * ни на что не влияют в противном случае
        \nocite{vakbib2}%vak
        % \nocite{patbib1}%patent
        % \nocite{progbib1}%program
        \nocite{bib1}%other
        \nocite{confbib1}%conf
    \end{refsection}%
        %
        % Всё, что вне этих двух refsection, это refsection=0,
        %  * для диссертации - это нормальные ссылки, попадающие в обычную библиографию
        %  * для автореферата:
        %     * при usefootcite==0, ссылка корректно сработает только для источника из `external.bib`. Для своих работ --- напечатает "[0]" (и даже Warning не вылезет).
        %     * при usefootcite==1, ссылка сработает нормально. В авторской библиографии будут только процитированные в refsection=0 работы.
}


% При использовании пакета \verb!biblatex! будут подсчитаны все работы, добавленные
% в файл \verb!biblio/author.bib!. Для правильного подсчёта работ в~различных
% системах цитирования требуется использовать поля:
% \begin{itemize}
%         \item \texttt{authorvak} если публикация индексирована ВАК,
%         \item \texttt{authorscopus} если публикация индексирована Scopus,
%         \item \texttt{authorwos} если публикация индексирована Web of Science,
%         \item \texttt{authorconf} для докладов конференций,
%         \item \texttt{authorpatent} для патентов,
%         \item \texttt{authorprogram} для зарегистрированных программ для ЭВМ,
%         \item \texttt{authorother} для других публикаций.
% \end{itemize}
% Для подсчёта используются счётчики:
% \begin{itemize}
%         \item \texttt{citeauthorvak} для работ, индексируемых ВАК,
%         \item \texttt{citeauthorscopus} для работ, индексируемых Scopus,
%         \item \texttt{citeauthorwos} для работ, индексируемых Web of Science,
%         \item \texttt{citeauthorvakscopuswos} для работ, индексируемых одной из трёх баз,
%         \item \texttt{citeauthorscopuswos} для работ, индексируемых Scopus или Web of~Science,
%         \item \texttt{citeauthorconf} для докладов на конференциях,
%         \item \texttt{citeauthorother} для остальных работ,
%         \item \texttt{citeauthorpatent} для патентов,
%         \item \texttt{citeauthorprogram} для зарегистрированных программ для ЭВМ,
%         \item \texttt{citeauthor} для суммарного количества работ.
% \end{itemize}
% % Счётчик \texttt{citeexternal} используется для подсчёта процитированных публикаций;
% % \texttt{citeregistered} "--- для подсчёта суммарного количества патентов и программ для ЭВМ.

% Для добавления в список публикаций автора работ, которые не были процитированы в
% автореферате, требуется их~перечислить с использованием команды \verb!\nocite! в
% \verb!Synopsis/content.tex!.