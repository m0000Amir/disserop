{\actuality} 

В настоящее время заметна тенденция цифровой трансформации, что подразумевает под собой бурное развитие информационных технологий во всех сферах деятельности человека. Основными перспективными направлениями цифровизации, на котороые сделан акцент в диссертационной работе, являются цифровая трансформация транспортного комплекса и глобальная цифровизация нефтегазового сектора страны. Этот подход подразумевает не только оснащение современным технологичным оборудованием, но и глобальное изменение в подходах управления, сбора информации и средств коммуникаций. Уже сейчас основными современными информационными технологиями, встречающимися на пути трансформации являются: большие данные (Big Data), предиктивные модели на искусственных нейронных сетях (Artificial Neural Network), системы распределенного реестра (Blockchain), промышленный интернет вещей (Industrial Internet of Things, IIoT), технологии виртуальной и дополненной реальности (Virtual Reality, VR), мониторинг распределенных объектов с помощью беспилотных летательных аппаратов БПЛА (Unmanned Aerial Vehicle, UAV). В совокупности данные технологии создают необходимость в эффективной передаче больших объемов высокоскоростного трафика. Одним из путей решенияй проблемы является интенсивное развитие и внедрение беспроводных технологий.

% \fixme{В настоящее время заметна тенденция бурного развития информационных технологий во всех сферах деятельности человека. На сегодня 

% оказывает весомое влияние на нефтегазовый сектор страны. Современные компании, представляющие собой сложные многоуровневые производственные системы, для своего устойчивого развития требуют постоянного развития и совершенствования передовых технологий.  Сегодня наблюдается  бурное развитие процесса «цифровизации» нефтегазовой отрасли. Крупные международные нефтегазовые компании имеют подразделения, задачами которых является разработка и реализация принципов интеллектуального месторождения \cite{Tcharo2018} на промысле, организация безопасности на технологических объектах, развитие концепции  перехода к малолюдным системам управления добычей, транспортировкой и переработкой сырья. Уже сейчас основными современными информационными технологиями, встречающиеся в отрасли являются: большие данные (Big Data), предикстивные модели на искусственных нейронных сетях (Artificial Neural Network), системы распределенного реестра (Blockchain), промышленный интернет вещей (Industrial internet of things, IIoT), технологии виртуальной и дополненной реальности (Virtual Reality, VR), мониторинг распределенных объектов с помощью беспилотных летательных аппаратов БПЛА (Unmanned Aerial Vehicle, UAV). }
 
 
% \fixme{В совокупности данные технологии создают необходимость эффективной передачи больших объемов высокоскоростного трафика. Информационные  системы современных  сегодня содержат колоссальный объем информации высокоскоростного мультимедийного трафика. Одним из путей решения является внедрение беспроводных технологий.}

В настоящее время беспроводные технологии являются неотъемлемой частью процесса «цифровизации». Активное использование беспроводных сетей основывается на ряде их преимуществ по сравнению с кабельными сетями:
\begin{itemize}
    \item организация связи в труднодоступных регионах;
    \item быстрый ввод в эксплуатацию по системе подключение типа <<Подключил и Работай>> (Plug-\&-Play);
    \item сокращение капитальных затрат на создание сети; 
    \item уменьшение затрат на эксплуатацию;
    \item высокая гибкость, мобильность, масштабируемость;
    \item упрощенные требования к обслуживанию оборудования.
\end{itemize}

В рамках этого процесса возникает актуальная научно-техническая проблема повышения качества проектирования беспроводной сети связи, осуществляющей мониторинг, сбор и передачу информации в центр управления с множества объектов на заданной территории.   

% В совокупности со всеми вышеизложенными перспективными направлениями беспроводные технологии являются неотъемлемой частью «цифровизации» месторождения. Отсюда возникает научно - техническая проблема организации распределенной беспроводной сети связи, соответствующая реальным требованием современного производства.

% Большой объем передачи информации  привел к еще одной из наиболее интересных тенденций цифрового развития – внедрения беспроводных технологий. Современные месторождения сегодня, помимо данных первичного сбора и обработки информации технологических параметров основных производственных объектов содержат также колоссальный объем  информации мультимедийного трафика. Сюда входят данные БПЛА по обнаружению утечек и  разрушения трубопроводов; камер видеонаблюдений; а также большой поток данных цифровых двойников, аналитики и т.д. В совокупности со всеми вышеизложенными перспективными направлениями беспроводные технологии являются неотъемлемой частью «цифровизации» месторождения.



Процесс проектирования современной  беспроводной сети связи состоит из последовательного решения взаимосвязанных задач:

\begin{enumerate}
    \item обследование местности;
    \item выбор типов технических средств и протоколов;
    \item выбор топологической структуры сети;
    \item анализ и оценка будущей беспроводной сети с помощью математического моделирования.
\end{enumerate}

Данная работа посвящена актуальной проблеме оптимизации топологической структуры беспроводной широкополосной сети (БШС). Задача выбора топологической структуры при проектировании является одной из важнейших задач, ошибки при которой могут привести к большим капитальным затратам и ухудшению качества обслуживания. С математической точки зрения задачи синтеза топологии являются сложной задачей, время счета для которой растет экспоненциально с ростом размерности. Таким образом высокий интерес к разработке новых моделей и методов оптимизации топологической структуры БШС определяют актуальность и новизну диссертационной работы.


% Задача синтеза топологии при комплексном проектировании БШС является основной проблемой исследования в данной работе.

% В.М. Вишневский, Ю. В. Гайдамака, А.Е. Кучерявый, А.А. Ларионов, В.М. Малыш,  О.Ю. Першин, К. Е. Самуйлов, Р.Л. Смелянский и другие.

{\progress} Создание и постоянное развитие современной инфраструктуры передачи данных является одной из основных задач цифровизации. Бурное развитие беспроводных сетей во всех областях деятельности человека: интеллектуальные транспортные системы вдоль автомобильных дорог, мониторинг нефтегазовых объектов, организация современного высокоскоростного покрытия в общественных зонах обосновывают целесообразность их использования. В настоящее время в России и за рубежом исследованию беспроводных сетей связи посвящены ряд работ, где рассматриваются сети для мониторинга гражданских  и промышленных объектов. Примерами таких объектов является жилые районы города, протяженные автомагистрали, линии метрополитена и железные дороги, магистральные трубопроводы и др. В частности, при исследовании проблемы синтеза топологии сети автор опирался на труды отечественных ученых, занимающиеся исследованиями в области телекоммуникационных сетей: В.М. Вишневский, Ю. В. Гайдамака, А. Е. Кучерявый, Е. А. Кучерявый, А. А. Ларионов, В. М. Малыш,  О. Ю. Першин, К. Е. Самуйлов, Р. Л. Смелянский и, в частности, решающие задачу синтеза топологии В. М. Вишневский, А. А. Ларионов, В.М. Малыш, О. Ю. Першин.
Наряду с отечественными работами указанные проблемы рассматривались в работах зарубежных авторов: Е.С. Кавальканте, Х. Лиу, А.Б. Рейз, Д.Ли, Д.П. Хейман, С. Шен, Д. Бендель, У. М. Амин, Б. Брахим, Х.Э. Кызылёз и др. В работах указанных авторов рассматриваются задачи оптимального синтеза топологии сети и исследуются вопросы анализа сетей, в том числе рассматриваются оценки характеристик сетей с помощью стохастических моделей сетей массового обслуживания. 

В работе представлены математические модели задачи оптимального размещения базовых станций, предложен новый алгоритм типа ветвей и границ для решения задачи в комбинаторной форме.
Исследования доведены до разработки алгоритмов и программ, применимых для решения практических задач. Приведены результаты численных экспериментов, позволяющие оценить характеристики вычислительных методов.


{\objectresearch} являются БШС вдоль протяженных транспортных магистралей: автомобильные и железные дороги, магистральные и промысловые трубопроводы, линии метрополитена. 

% специальных типов, широко представленных на практике: БШС для контроля линейного участка и БШС с ячеистой топологией (mesh) для контроля объектов, рассредоточенных на заданной территории.

{\subjectresearch} является синтез топологической структуры БШС.

{\aim} состоит в разработке моделей и методов оптимального размещения базовых станций для БШС, определяющих топологию таких сетей.

Для достижения поставленной цели были решены следующие {\tasks}:
\begin{enumerate}[beginpenalty=10000] % https://tex.stackexchange.com/a/476052/104425
  \item Проведен анализ современного состояния и перспектив развития БШС для  обоснования  актуальности исследований в области оптимизации их топологии. 
  \item Проанализирована методика проектирования современных БШС с целью определения требований к решению задачи синтеза оптимальной топологии сети, а также расчета параметров БШС, необходимых для решения задач размещения базовых станций.
  \item Построены математические модели для задачи оптимального размещения базовых станций БШС с линейной топологией, разработан алгоритм типа метода ветвей и границ (МВиГ) для решения указанной задачи, предложена итерационная процедура нахождения последовательности лучших решений в размещении базовых станций в рамках комплексного проектирования сети.
  \item Построена математическая модель в виде задачи частично целочисленного линейного программирования (ЧЦЛП) для решения задач проектирования и анализа БШС с ячеистой топологией.
%   \item Разработан программный комплекс для расчета задачи размещения оптимального размещения базовых станций.
  \item Приведены численные эксперименты, доказывающие значимость представленных математических моделей и разработанного алгоритма.
  %   \item разработаны методы оценки характеристик производительности сети с помощью данных имитационного моделирования и методов машинного обучения. 

\end{enumerate}

 

{\novelty} результатов исследования заключается в следующем:
\begin{enumerate}[beginpenalty=10000] % https://tex.stackexchange.com/a/476052/104425

    \item Построена новая математическая модель целочисленного линейного программирования задачи оптимального размещения базовых станций при проектировании БШС с линейной топологией. 
    \item Построена новая комбинаторная модель в экстремальной форме решения задачи оптимального размещения базовых станций при проектировании БШС с линейной топологией.   
    \item Разработан специальный алгоритм типа ветвей и границ для решения численным методом сформулированной экстремальной комбинаторной задачи, учитывающий специфику размещения базовых станций БШС для телекоммуникационного покрытия протяженных объектов.
    \item  Разработана новая итерационная процедура нахождения последовательности лучших решений для задачи размещения базовых станций в рамках комплексного проектирования БШС для телекоммуникационного покрытия протяженных объектов.
    \item Разработан программный комплекс для расчета комбинаторной задачи с помощью предложенного алгоритма.
    \item Разработана новая математическая модель в виде задачи ЧЦЛП для задачи проектирования БШС с ячеистой топологией.

    % \item Построены новые математические модели в виде экстремальной комбинаторной задачи и задачи целочисленного линейного программирование для оптимального размещения базовых станций при проектировании БШС с линейной топологией.  
    % \item Разработан специальный алгоритм МВиГ для решения сформулированной экстремальной комбинаторной задачи размещения базовых станций вдоль линейной территории.
    % \item Разработана итерационная процедура нахождения последовательности лучших решений для задачи размещения базовых станций в рамках комплексного проектирования БШС с линейной топологией.
    % \item Разработан программный комплекс для ЭВМ расчета комбинаторной задачи.
    % \item Разработаны новые математические модели в виде задач ЦЛП для задач проектирования БШС с ячеистой топологией.
    

%   \item  \fixme{разработаны алгоритмы для анализа выполнения технологических требований и оптимального размещения базовых станций для  БШС с ячеистой топологией};
%   \item работаны имитационные модели многофазной сети массового обслуживания с зависимым временем обслуживания;
%   \item разработаны модели прогнозирования оценок характеристик производительности сети с помощью методов машинного обучения. 
\end{enumerate}

{\fieldresearch}. Диссертационная работа соответствует содержанию специальности 05.13.18 <<Математическое моделирование, численные методы и комплексы программ>>, а именно следующим пунктам специальности:
\begin{enumerate}
    \item Разработка новых математических методов моделирования объектов и явлений.
    \item Развитие качественных и приближенных аналитических методов исследования математических моделей.
    \item Реализация эффективных численных методов и алгоритмов в виде комплексов проблемно-ориентированных программ для проведения вычислительного эксперимента.
    \item Комплексные исследования научных и технических проблем с применением современной технологии математического моделирования и вычислительного эксперимента.
\end{enumerate}

{\influence}. Разработанные модели, методы и программный комплекс позволяют повысить качество проектирования БШС. Представленные численные эксперименты в работе подтверждают эффективность при решении задачи синтеза топологии сети. Результаты исследования, изложенные в диссертации, были получены в рамках выполнения работ при финансовой поддержке Российского фонда фундаментальных исследований по грантам \textnumero  19-07-00919,  \textnumero 19-29-06043, \textnumero 20-37-70059 и Российского научного фонда \textnumero 22-49-02023. 

% {\elaboration}. Исследования доведены до разработки алгоритмов и программ, применимых для решения практических задач. Проведены численные эксперименты, позволяющие оценить характеристики вычислительных методов.

{\methods} В работе использованы теория и методы дискретной оптимизации, математического программирования, оптимизации на конечных множествах, теории графов, методы теории вероятностей и случайных процессов, математической статистики, теории массового обслуживания. Разработка программного комплекса проводилось с использованием парадигмы объектно-ориентированного программирования.

{\defpositions}
% \begin{enumerate}[beginpenalty=10000] % https://tex.stackexchange.com/a/476052/104425
%   \item математические модели линейной задачи и алгоритм решения линейной комбинаторной задачи методом ветвей и границ;
%   \item итерационная процедура построения последовательности топологий; 
%   \item математическая модель задачи покрытия множества рассредоточенных объектов; 
%   \item имитационная модель сети массового обслуживания с зависимым распределением времени обслуживания;
%   \item регрессионные модели характеристики производительности сети, полученные с помощью методов машинного обучения.
% \end{enumerate}

\begin{enumerate}[beginpenalty=10000] % https://tex.stackexchange.com/a/476052/104425
    \item Математичесая модель в виде задачи ЦЛП для оптимального размещения базовых станций при проектировании БШС с линейной топологией.
    \item Математичесая модель в виде экстремальной комбинаторной задачи для оптимального размещения базовых станций при проектировании БШС с линейной топологией.
    \item Специальный алгоритм МВиГ для решения сформулированной
    экстремальной комбинаторной задачи.
    \item Итерационная процедура нахождения последовательности лучших
    решений для задачи размещения базовых станций в рамках комплексного
    проектирования БШС с линейной топологией.
    \item Математические модели для задач проектирования БШС с ячеистой
    топологией.
    % \item модели прогнозирования оценок характеристик производительности сети с
    % помощью методов машинного обучения для многофазной сети массового обслуживания.
  \end{enumerate}

% В папке Documents можно ознакомиться с решением совета из Томского~ГУ (в~файле \verb+Def_positions.pdf+), где обоснованно даются рекомендации по~формулировкам защищаемых положений.

% {\reliability} \fixme{полученных результатов обеспечивается \ldots \ Результаты находятся в соответствии с результатами, полученными другими авторами.}


{\probation}
Основные положения и результаты исследования представлены и обсуждены на научных конференциях «Губкинский университет в решении вопросов нефтегазовой отрасли России» (Москва, 17-21 сентября 2018); «13-е Всероссийское совещание по проблемам управления» (Москва, 17-20 июня 2019); «International Conference on Distributed Computer and Communication Networks: Control, Computation, Communications» (Москва, 22-27 сентября 2019), «Губкинский университет в решении вопросов нефтегазовой отрасли России» (Москва, 24-26 сентября 2019); «Conference Management of Large-Scale System Development» (Москва, 1-3 октября 2019); «Information and Telecommunication Technologies and Mathematical Modeling of High-Tech Systems» (Москва, 13-17 апреля 2020); «Computer-aided technologies in applied mathematics» (Томск, 7-9 сентября 2020); «International Conference on Distributed Computer and Communication Networks: Control, Computation, Communications» (Москва, 14-18 сентября 2020); «Information and Telecommunication Technologies and Mathematical Modeling of High-Tech Systems» (Москва, 19-23 апреля 2021);  «5th International Scientific Conference on Information, Control, and Communication Technologies» (Астрахань, 4-7 октября 2021)


{\contribution} Основные результаты диссертации, выносимые на защиту получены автором самостоятельно.

\ifnumequal{\value{bibliosel}}{1}
{%%% Встроенная реализация с загрузкой файла через движок bibtex8. (При желании, внутри можно использовать обычные ссылки, наподобие `\cite{vakbib1,vakbib2}`).
    {\publications} Основные результаты по теме диссертации изложены в 15 печатных изданиях, 2 из которых изданы в журналах, рекомендованных ВАК, 3 — в периодических научных журналах, индексируемых Web of Science и Scopus, 10 — в сборниках трудов конференции, индексируемых РИНЦ. Зарегистрирована 1 программа для ЭВМ.
}%
{%%% Реализация пакетом biblatex через движок biber
    \begin{refsection}[bl-author, bl-registered]
        % Это refsection=1.
        % Процитированные здесь работы:
        %  * подсчитываются, для автоматического составления фразы "Основные результаты ..."
        %  * попадают в авторскую библиографию, при usefootcite==0 и стиле `\insertbiblioauthor` или `\insertbiblioauthorgrouped`
        %  * нумеруются там в зависимости от порядка команд `\printbibliography` в этом разделе.
        %  * при использовании `\insertbiblioauthorgrouped`, порядок команд `\printbibliography` в нём должен быть тем же (см. biblio/biblatex.tex)
        %
        % Невидимый библиографический список для подсчёта количества публикаций:
        \printbibliography[heading=nobibheading, section=1, env=countauthorvak,          keyword=biblioauthorvak]%
        \printbibliography[heading=nobibheading, section=1, env=countauthorwos,          keyword=biblioauthorwos]%
        \printbibliography[heading=nobibheading, section=1, env=countauthorscopus,       keyword=biblioauthorscopus]%
        \printbibliography[heading=nobibheading, section=1, env=countauthorconf,         keyword=biblioauthorconf]%
        \printbibliography[heading=nobibheading, section=1, env=countauthorother,        keyword=biblioauthorother]%
        \printbibliography[heading=nobibheading, section=1, env=countregistered,         keyword=biblioregistered]%
        \printbibliography[heading=nobibheading, section=1, env=countauthorpatent,       keyword=biblioauthorpatent]%
        \printbibliography[heading=nobibheading, section=1, env=countauthorprogram,      keyword=biblioauthorprogram]%
        \printbibliography[heading=nobibheading, section=1, env=countauthor,             keyword=biblioauthor]%
        \printbibliography[heading=nobibheading, section=1, env=countauthorvakscopuswos, filter=vakscopuswos]%
        \printbibliography[heading=nobibheading, section=1, env=countauthorscopuswos,    filter=scopuswos]%
        %
        \nocite{*}%
        %
        {\publications} Основные результаты по теме диссертации изложены в~\arabic{citeauthor}~печатных изданиях,
        \arabic{citeauthorvak} из которых изданы в журналах, рекомендованных ВАК\sloppy%
        \ifnum \value{citeauthorscopuswos}>0%
            , \arabic{citeauthorscopuswos} "--- в~периодических научных журналах, индексируемых Web of~Science и Scopus\sloppy%
        \fi%
        \ifnum \value{citeauthorconf}>0%
            , \arabic{citeauthorconf} "--- в~сборниках трудов конференции.
        \else%
            .
        \fi%
        \ifnum \value{citeregistered}=1%
            \ifnum \value{citeauthorpatent}=1%
                Зарегистрирован \arabic{citeauthorpatent} патент.
            \fi%
            \ifnum \value{citeauthorprogram}=1%
                Зарегистрирована \arabic{citeauthorprogram} программа для ЭВМ.
            \fi%
        \fi%
        \ifnum \value{citeregistered}>1%
            Зарегистрированы\ %
            \ifnum \value{citeauthorpatent}>0%
            \formbytotal{citeauthorpatent}{патент}{}{а}{}\sloppy%
            \ifnum \value{citeauthorprogram}=0 . \else \ и~\fi%
            \fi%
            \ifnum \value{citeauthorprogram}>0%
            \formbytotal{citeauthorprogram}{программ}{а}{ы}{} для ЭВМ.
            \fi%
        \fi%
        % К публикациям, в которых излагаются основные научные результаты диссертации на соискание учёной
        % степени, в рецензируемых изданиях приравниваются патенты на изобретения, патенты (свидетельства) на
        % полезную модель, патенты на промышленный образец, патенты на селекционные достижения, свидетельства
        % на программу для электронных вычислительных машин, базу данных, топологию интегральных микросхем,
        % зарегистрированные в установленном порядке.(в ред. Постановления Правительства РФ от 21.04.2016 N 335)
    \end{refsection}%
    \begin{refsection}[bl-author, bl-registered]
        % Это refsection=2.
        % Процитированные здесь работы:
        %  * попадают в авторскую библиографию, при usefootcite==0 и стиле `\insertbiblioauthorimportant`.
        %  * ни на что не влияют в противном случае
        \nocite{vakbib2}%vak
        % \nocite{patbib1}%patent
        % \nocite{progbib1}%program
        \nocite{bib1}%other
        \nocite{confbib1}%conf
    \end{refsection}%
        %
        % Всё, что вне этих двух refsection, это refsection=0,
        %  * для диссертации - это нормальные ссылки, попадающие в обычную библиографию
        %  * для автореферата:
        %     * при usefootcite==0, ссылка корректно сработает только для источника из `external.bib`. Для своих работ --- напечатает "[0]" (и даже Warning не вылезет).
        %     * при usefootcite==1, ссылка сработает нормально. В авторской библиографии будут только процитированные в refsection=0 работы.
}


% При использовании пакета \verb!biblatex! будут подсчитаны все работы, добавленные
% в файл \verb!biblio/author.bib!. Для правильного подсчёта работ в~различных
% системах цитирования требуется использовать поля:
% \begin{itemize}
%         \item \texttt{authorvak} если публикация индексирована ВАК,
%         \item \texttt{authorscopus} если публикация индексирована Scopus,
%         \item \texttt{authorwos} если публикация индексирована Web of Science,
%         \item \texttt{authorconf} для докладов конференций,
%         \item \texttt{authorpatent} для патентов,
%         \item \texttt{authorprogram} для зарегистрированных программ для ЭВМ,
%         \item \texttt{authorother} для других публикаций.
% \end{itemize}
% Для подсчёта используются счётчики:
% \begin{itemize}
%         \item \texttt{citeauthorvak} для работ, индексируемых ВАК,
%         \item \texttt{citeauthorscopus} для работ, индексируемых Scopus,
%         \item \texttt{citeauthorwos} для работ, индексируемых Web of Science,
%         \item \texttt{citeauthorvakscopuswos} для работ, индексируемых одной из трёх баз,
%         \item \texttt{citeauthorscopuswos} для работ, индексируемых Scopus или Web of~Science,
%         \item \texttt{citeauthorconf} для докладов на конференциях,
%         \item \texttt{citeauthorother} для остальных работ,
%         \item \texttt{citeauthorpatent} для патентов,
%         \item \texttt{citeauthorprogram} для зарегистрированных программ для ЭВМ,
%         \item \texttt{citeauthor} для суммарного количества работ.
% \end{itemize}
% % Счётчик \texttt{citeexternal} используется для подсчёта процитированных публикаций;
% % \texttt{citeregistered} "--- для подсчёта суммарного количества патентов и программ для ЭВМ.

% Для добавления в список публикаций автора работ, которые не были процитированы в
% автореферате, требуется их~перечислить с использованием команды \verb!\nocite! в
% \verb!Synopsis/content.tex!.