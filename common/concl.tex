%% Согласно ГОСТ Р 7.0.11-2011:
%% 5.3.3 В заключении диссертации излагают итоги выполненного исследования, рекомендации, перспективы дальнейшей разработки темы.
%% 9.2.3 В заключении автореферата диссертации излагают итоги данного исследования, рекомендации и перспективы дальнейшей разработки темы.
\begin{enumerate}
  \item разработаны математические модели в виде экстремальной комбинаторной задачи и
  задачи ЦЛП для оптимального размещения базовых станций при
  проектировании беспроводных широкополосных сетей (БШС) с линейной топологией;
  \item предложен специальный алгоритм МВиГ для решения сформулированной
  экстремальной комбинаторной задачи;
  \item разработана итерационная процедура нахождения последовательности лучших
  решений для задачи размещения базовых станций в рамках комплексного
  проектирования БШС с линейной топологией;
  \item разработаны математические модели для задач проектирования БШС с ячеистой
  топологией;
  \item предложены модели прогнозирования оценок характеристик производительности сети с
  помощью методов машинного обучения для многофазной сети массового обслуживания.
\end{enumerate}

% \begin{enumerate}
%   \item построены математические модели в виде экстремальной комбинаторной задачи и задачи ЦЛП для оптимального размещения базовых станций при проектировании БШС с линейной топологией;
%   \item представлен алгоритм метода ветвей и границ для задачи размещения базовых станций с линейной топологией; 
%   \item разработана итерационная процедура нахождения последовательности лучших решений для задачи размещения базовых станций в рамках комплексного проектирования БШС с линейной топологией;
%   \item разработаны математические модели для задач проектирования БШС для покрытия множества рассредоточенныз объектов;
%   \item разработаны модели прогнозирования оценок характеристик производительности сети с помощью методов машинного обучения.
% \end{enumerate}
