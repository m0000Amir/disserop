%% Согласно ГОСТ Р 7.0.11-2011:
%% 5.3.3 В заключении диссертации излагают итоги выполненного исследования, рекомендации, перспективы дальнейшей разработки темы.
%% 9.2.3 В заключении автореферата диссертации излагают итоги данного исследования, рекомендации и перспективы дальнейшей разработки темы.

\begin{enumerate}
    \item Проведен анализ методики проектирования современных беспроводных широкополосных сетей. В рамках такой методики были исследованы проблемы синтеза топологии беспроводных сетей вдоль протяженных транспортных магистралей. 
    \item Представлена новая математическая модель в виде задачи целочисленного линейного программирования оптимального размещения базовых станций с линейной топологией.
    \item Представлена новая математическая модель задачи оптимального размещения БС в виде комбинаторной модели в экстремальной форме. 
    % Данная модель учитывает специфику задачи для нахождения оптимального решения.
    \item Для комбинаторной модели разработан новый специальный алгоритм типа ветвей и границ, учитывающий специфику решения задачи размещения базовых станций широкополосной сети вдоль протяженных транспортных магистралей, а также учитывающий ограничение на среднюю задержку в сети.  
    
    % Представлены результаты сравнения поиска оптимального решения с помощью МВиГ с алгоритмами решения задачи в общем виде.
    \item В рамках комплексного проектирования БШС представлена новая итерационная процедура нахождения последовательности лучших решений задачи оптимального размещения базовых станций для случая, когда найденное оптимальное решение не удовлетворяет  критериями функционирования БШС, проверяемых на следующих этапах проектирования.
    \item Предложена новая математическая модель в виде задачи частично целочисленного линейного программирования оптимального размещения базовых станций по критерию минимизации стоимости для покрытия множества рассредоточенных объектов. 
    \item Разработан программный комплекс для расчета задачи оптимального размещения базовых станций с помощью нового алгоритма типа ветвей и границ.
    \item Представлены результаты численных экспериментов, подтверждающие целесообразность использования сформулированных моделей оптимизации и разработанного комбинаторного алгоритма в практических целях. 
\end{enumerate}

% \begin{enumerate}
%     \item Был проведен анализ методики проектирования современных БШС. В рамках такой методики были исследованы проблемы синтеза топологии БШС вдоль протяженных участков: трубопроводные магистрали, протяженные автомобильные дороги. 
%     \item Была предложена математическая модель в виде задачи ЦЛП размещения БС с линейной топологией.
%     \item Была представлена математическая модель экстремальной задачи оптимального размещения БС в комбинаторной форме. 
%     % Данная модель учитывает специфику задачи для нахождения оптимального решения.
%     \item Для комбинаторной модели был разработан специальный алгоритм типа ветвей и границ. Представлены результаты сравнения поиска оптимального решения с помощью МВиГ с алгоритмами решения задачи в общем виде.
%     \item В рамках комплексного проектирования была представлена итерационная процедура нахождения последовательности лучших решений задачи оптимального размещения для случая, когда найденное оптимальное решение построение топологии сети не удовлетворяет некоторым критериями функционирования БШС, проверяемых на этапе моделирования процесса передачи данных.
%     \item Предложена математическая модель оптимального размещения БС для покрытия множества рассредоточенных объектов.
%     \item Разработан программный комплекс для расчета задачи оптимального размещения БС  с помощью предложенного алгоритма типа ветвей и границ.
% \end{enumerate}




% \begin{enumerate}
%   \item построены математические модели в виде экстремальной комбинаторной задачи и задачи ЦЛП для оптимального размещения базовых станций при проектировании БШС с линейной топологией;
%   \item представлен алгоритм метода ветвей и границ для задачи размещения базовых станций с линейной топологией; 
%   \item разработана итерационная процедура нахождения последовательности лучших решений для задачи размещения базовых станций в рамках комплексного проектирования БШС с линейной топологией;
%   \item разработаны математические модели для задач проектирования БШС для покрытия множества рассредоточенныз объектов;
%   \item разработаны модели прогнозирования оценок характеристик производительности сети с помощью методов машинного обучения.
% \end{enumerate}
